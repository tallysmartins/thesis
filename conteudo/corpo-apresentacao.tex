%!TeX root=../apresentacao.tex
%("dica" para o editor de texto: este arquivo é parte de um documento maior)
% para saber mais: https://tex.stackexchange.com/q/78101/183146

% Apague as duas linhas abaixo (elas servem apenas para gerar um
% aviso no arquivo PDF quando não há nenhum dado a imprimir) e
% insira aqui o conteúdo do seu trabalho. Tome por base o
% arquivo corpo-apresentacao.tex do diretório conteudo-exemplo
% para a definição do título, autoria etc. e estrutura da
% apresentação.

\input{extras/aviso-conteudo}
\avisoApresentacao
%%%%%%%%%%%%%%%%%%%%%%%%%%%%%%%%% METADADOS %%%%%%%%%%%%%%%%%%%%%%%%%%%%%%%%%%%%

\title[The shortened title]{Visualização Dinâmica Multiescala\\ de Dados do Tráfego
de Veículos}
% \subtitle{The (optional) subtitle}

\author[Tallys Martins]{Tallys Martins}

%\institute[USP]{\textbf{Workshop Name} \\ Computer Science Department \\ IME USP}
\institute[USP]{
 \textbf{Orientador:} Fabio Kon \\
 \textbf{Co-Orientador:} Higor Amario de Souza \\
 Computer Science Department \\ IME USP
}

\date{Março de 2019}

% Coloca a imagem no fundo da página de título
\bgimage{\includegraphics[width=\paperwidth]{fundo_predios_e_grafo}}

% Logotipos no rodapé da página de título
\logos{
  \hfil\hfil\includegraphics[width=.1\textwidth]{usp-logo}\hfil%
  \raisebox{-.0103\paperheight}{\includegraphics[height=.0932\paperheight]{interscity-logo}}\hfil%
  \raisebox{-.033\paperheight}{\includegraphics[width=.07\textwidth,trim=0 230 0 0,clip]{ime-logo}}\hfil\hfil
}

%\logos{
%  \hfil\hfil\includegraphics[width=.1\textwidth]{usp-logo}\hfil%
%  \raisebox{-.0103\paperheight}{\includegraphics[height=.0932\paperheight]{interscity-logo}}\hfil%
%  \raisebox{-.00517\paperheight}{\includegraphics[height=.057\paperheight]{cnpq-logo}}\hfil%
%  \raisebox{-.0342\paperheight}{\includegraphics[height=.1035\paperheight]{capes-logo}}\hfil%
%  \includegraphics[height=.044\paperheight]{fapesp-logo}\hfil\hfil
%}

% Usado para criar o qrcode com o endereço da apresentação
\presentationurl{http://interscity.org}

% Inclui o qrcode no sumário da apresentação
\includeqrcodeintoc

% O slide de sumário pode ser dividido em colunas; o parâmetro
% determina após qual o número da seção fazer a quebra de coluna
% (use zero para uma coluna ou simplesmente omita este comando).
\toccolumns{4}


%%%%%%%%%%%%%%%%%%%%%%%%%%%%%%%%%%%%%%%%%%%%%%%%%%%%%%%%%%%%%%%%%%%%%%%%%%%%%%%%
%%%%%%%%%%%%%%%%%%%%%%%%%%%% INÍCIO DA APRESENTAÇÃO %%%%%%%%%%%%%%%%%%%%%%%%%%%%
%%%%%%%%%%%%%%%%%%%%%%%%%%%%%%%%%%%%%%%%%%%%%%%%%%%%%%%%%%%%%%%%%%%%%%%%%%%%%%%%

% É complicado colocar uma imagem de fundo, os logos das agências e
% o conteúdo "normal" do slide de título sem que as coisas fiquem
% bagunçadas, então definimos um comando para gerar o slide de título
\customtitlepage

% Slide com o qrcode
\showqrcode

\begin{frame}{Agenda}
  \overview
\end{frame}

\section{Introdução}

\begin{frame}{Contexto}
  \begin{itemize}
    \item A visualização realmente visualiza
    \item Very good very nice
    \item[pula mijoleta]
    \item As a reaction, the free software movement was created
    \begin{itemize}
      \item Return to sharing (of source code) and to collaboration (exchange of ideas and team work)
      \item Formalization with the GNU project
      \item Only really possible when there are favourable conditions for source code exchange
      \begin{itemize}
        \item as highlighted by the growth that accompanied the Internet boom
      \end{itemize}
    \end{itemize}
  \end{itemize}

\end{frame}

\begin{frame}[plain]
  \includegraphics[width=\textwidth]{interscity-logo}
\end{frame}

\begin{frame}[standout]
  Data data data
\end{frame}

\begin{frame}{Objetivos}
  \begin{block}{Explorar o \emph{bundling}}
    \begin{itemize}
      \item Integration and Management of \alert{IoT} Devices
      \item Data Acquisition, Storing, and Processing
      \item Context-awareness
      \item City Resource Discovery
      \item Geolocation-based Services
      \item External data access
    \end{itemize}
  \end{block}
\end{frame}

\section{Conceitos}

\begin{frame}{Conceitos}
  \begin{columns}[t]
    \col
      \begin{coloredblock}{red!90!black}{Functional requirements}
        \begin{itemize}
          \item Integration and Management of IoT Devices
          \item Data Acquisition, Storing, and Processing
          \item Context-awareness
          \item City Resource Discovery
          \item Geolocation-based Services
          \item External data access
        \end{itemize}
      \end{coloredblock}

    \col
      \begin{coloredblock}{red!90!black}{Non-functional requirements}
        \begin{itemize}
          \item Interoperability
          \item Scalability
          \item Security
          \item Privacy
          \item Evolvability
          \item Adaptability
        \end{itemize}
      \end{coloredblock}
  \end{columns}
\end{frame}

\begin{frame}{Theorems and proofs}
  \pause
  \begin{theorem}[An example theorem]
    Theorem\dots
  \end{theorem}

  \pause
  \begin{example}[An example of an example]
    Example\dots
  \end{example}

  \pause
  \begin{proof}[An example proof]
    Proof\dots
  \end{proof}

  \pause
  \begin{definition}[An example definition]
    Definition\dots
  \end{definition}

  \pause
  \begin{proposition}[An example proposition]
    Proposition\dots
  \end{proposition}
\end{frame}

\section{Trabalhos Relacionados}

\begin{frame}{Trabalhos Relacionados}
\end{frame}

\section{Metodologia}

\begin{frame}{Metodologia}
\end{frame}

\section{Resultados}

\subsection{Validação e Análise}

\begin{frame}{Validação}
\end{frame}

\begin{frame}{Estudo de Caso}
\end{frame}

\section{Conclusion and Future works}

\begin{frame}{Conclusion and Future works}
\end{frame}

\section{References}

\begin{frame}[allowframebreaks]{References}
  \nocite{bronevetsky02, schmidt03:MSc, FSF:GNU-GPL, CORBA:spec, MenaChalco08, natbib, biblatex, eco:09}
  \printbibliography
\end{frame}

% Recapitulando
\begin{frame}{\insertshorttitle}
  \overview

  % \begin{center} acrescenta espaço vertical;
  % como possivelmente temos bem pouco espaço aqui,
  % vamos usar centering
  {%
    \centering\noindent%
    \url{https://gitlab.com/link-of-your-repository}\par
  }

\end{frame}

\showqrcode

\appendix

\begin{frame}{Extra info}
  \begin{itemize}
    \item It is often useful to have some extra slides addressing likely questions from the audience at the end of the presentation
    \item By putting them after the ``appendix'' command, they are not counted in the page count indicator
  \end{itemize}
\end{frame}
