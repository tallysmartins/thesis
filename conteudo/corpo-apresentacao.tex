%!TeX root=../apresentacao.tex
%("dica" para o editor de texto: este arquivo é parte de um documento maior)
% para saber mais: https://tex.stackexchange.com/q/78101/183146

% Apague as duas linhas abaixo (elas servem apenas para gerar um
% aviso no arquivo PDF quando não há nenhum dado a imprimir) e
% insira aqui o conteúdo do seu trabalho. Tome por base o
% arquivo corpo-apresentacao.tex do diretório conteudo-exemplo
% para a definição do título, autoria etc. e estrutura da
% apresentação.

\input{extras/aviso-conteudo}
\avisoApresentacao
%%%%%%%%%%%%%%%%%%%%%%%%%%%%%%%%% METADADOS %%%%%%%%%%%%%%%%%%%%%%%%%%%%%%%%%%%%

\title[The shortened title]{Visualização Dinâmica Multiescala\\ de Dados do Tráfego
de Veículos}
% \subtitle{The (optional) subtitle}

\author[Tallys Martins]{Tallys Martins}

%\institute[USP]{\textbf{Workshop Name} \\ Computer Science Department \\ IME USP}
\institute[USP]{
 \textbf{Orientador:} Fabio Kon \\
 \textbf{Co-Orientador:} Higor Amario de Souza \\
 Computer Science Department \\ IME USP
}

\date{Março de 2019}

% Coloca a imagem no fundo da página de título
\bgimage{\includegraphics[width=\paperwidth]{fundo_predios_e_grafo}}

% Logotipos no rodapé da página de título
\logos{
  \hfil\hfil\includegraphics[width=.1\textwidth]{usp-logo}\hfil%
  \raisebox{-.0103\paperheight}{\includegraphics[height=.0932\paperheight]{interscity-logo}}\hfil%
  \raisebox{-.033\paperheight}{\includegraphics[width=.07\textwidth,trim=0 230 0 0,clip]{ime-logo}}\hfil\hfil
}

%\logos{
%  \hfil\hfil\includegraphics[width=.1\textwidth]{usp-logo}\hfil%
%  \raisebox{-.0103\paperheight}{\includegraphics[height=.0932\paperheight]{interscity-logo}}\hfil%
%  \raisebox{-.00517\paperheight}{\includegraphics[height=.057\paperheight]{cnpq-logo}}\hfil%
%  \raisebox{-.0342\paperheight}{\includegraphics[height=.1035\paperheight]{capes-logo}}\hfil%
%  \includegraphics[height=.044\paperheight]{fapesp-logo}\hfil\hfil
%}

% Usado para criar o qrcode com o endereço da apresentação
\presentationurl{http://interscity.org}

% Inclui o qrcode no sumário da apresentação
\includeqrcodeintoc

% O slide de sumário pode ser dividido em colunas; o parâmetro
% determina após qual o número da seção fazer a quebra de coluna
% (use zero para uma coluna ou simplesmente omita este comando).
\toccolumns{4}


%%%%%%%%%%%%%%%%%%%%%%%%%%%%%%%%%%%%%%%%%%%%%%%%%%%%%%%%%%%%%%%%%%%%%%%%%%%%%%%%
%%%%%%%%%%%%%%%%%%%%%%%%%%%% INÍCIO DA APRESENTAÇÃO %%%%%%%%%%%%%%%%%%%%%%%%%%%%
%%%%%%%%%%%%%%%%%%%%%%%%%%%%%%%%%%%%%%%%%%%%%%%%%%%%%%%%%%%%%%%%%%%%%%%%%%%%%%%%

% É complicado colocar uma imagem de fundo, os logos das agências e
% o conteúdo "normal" do slide de título sem que as coisas fiquem
% bagunçadas, então definimos um comando para gerar o slide de título
\customtitlepage

% Slide com o qrcode
\showqrcode

\begin{frame}{Agenda}
  \overview
\end{frame}

\section{Introdução}

% O trabalho é sobre dados do tráfego, então nada mais justo do que começar
% explicando o que é o tráfego e que estamos interessados no trânsito
% Pode ressaltar que 40% da população mundial gasta pelo menos 1h por dia no trânsito
% E também pode falar que esses dados na cidade são coletados com os avanços
% nos sistemas de transporte
\begin{frame}{Dados de Tráfego}
  ``Tráfego em uma via é o fluxo ou a passagem de veículos e pedestres por um
caminho que pode ser em regiões urbanas, mares, no ar e até no subsolo''

\hfill \citep{Chen2015}

  \begin{figure}[h!]
    \centering
    \begin{subfigure}{0.20\textwidth}
      \centering
      \includegraphics[width=.4\textwidth]{black-plane.png}
    \end{subfigure}
    ~
    \begin{subfigure}{0.20\textwidth}
      \centering
      \includegraphics[width=.4\textwidth]{delivery-truck.png}
    \end{subfigure}
    ~
    \begin{subfigure}{0.20\textwidth}
      \centering
      \includegraphics[width=.4\textwidth]{sea-ship.png}
    \end{subfigure}
    ~
    \begin{subfigure}{0.20\textwidth}
      \centering
      \includegraphics[width=.4\textwidth]{one-man-walking.png}
    \end{subfigure}
  \end{figure}
\end{frame}

%\begin{frame}[plain]
%  \includegraphics[width=\textwidth]{interscity-logo}
%\end{frame}

% Aqui eu falo de mais algumas características sobre os dados do tráfego e
% como eles podem auxiliar no planejamento das cidades
% o que dá uma noção da complexidade de análise desses dados
\begin{frame}{Dados de Tráfego}
  \begin{itemize}
    \item Contém relações de deslocamento entre regiões
    \item Multidimensionais (posição, velocidade, aceleração)
    \item Volumosos (\emph{big data})
  \end{itemize}
\end{frame}

% Aqui eu falo que eles podem ser representados como um grafo e que essa representação
% pode ajudar na compreensão das suas características de uma maneira visual
% essa maneira é conhecida como leiaute/metáfora de linhas
\begin{frame}{Dados de Tráfego}
  Podem ser representados como um grafo onde os nós representam localidades e
as arestas representam o movimento entre dois pontos no espaço. Assim, temos
uma representação que ajuda a mapear visualmente as características e padrões
contidas nos dados
\end{frame}

% Porém essa representação começa a apresentar problemas quando o número
% de elementos aumenta (cruzamentos, sobreposição) como na figura do tráfego aéreo, fica difícil enxergar
% as conexões entre aeroportos, etc
\begin{frame}{Visualização do Tráfego}
  Mas esse tipo de representação começa a ter problemas de oclusão à medida
que a quantidade de elementos aumenta
\end{frame}

% Aqui eu introduzo o bundling como solução para simplificar a visualização
% desse tipo de representação agrupando-se as arestas similares.
% Com ela podemos ver a estrutura geral dos dados.
% Citar que existem maneiras diferentes de se atingir um "bundling"
\begin{frame}{Bundling}
  Uma técnica chamada bundling, ajuda a simplificar a visualização desse tipo
de representação. Ela ajuda a ter uma visão geral da estrutura dos dados
agrupando as arestas similares, como origens e destinos similares.
\end{frame}

\begin{frame}{Atualmente}
  No trânsito até encontramos algumas aplicações como Waze e google Maps, mas
que não dão uma visão geral como proporcionado pelo uso de bundling sobre
os dados do tráfego.
\end{frame}

%\begin{frame}[standout]
%  Bundling
%\end{frame}

\begin{frame}{Objetivos}
  \begin{block}{Explorar o \emph{bundling} na visualização dos fluxos de origem-destino}
    \begin{enumerate}
      \item Como o bundling pode ser usado para identificar os fluxos de origem
      e destino no trânsito em diferentes escalas?

      \item É possível utilizar o bundling para identificar padrões de fluxos
      de origem e destino no trânsito?

      \item O bundling é eficiente para gerar uma visualização de uma grande
      quantidade de dados do trânsito?
    \end{enumerate}
  \end{block}
\end{frame}

\begin{frame}{Desafios}
    \begin{enumerate}
      \item Selecionar um método de bundling

      \item Implementar método de bundling selecionado

      \item Elaborar cenários para validar nossa abordagem

      \item Executar cenários elaborados
    \end{enumerate}
\end{frame}

\section{Conceitos}

\begin{frame}{Bundling}
    Modelos e ADEB
\end{frame}


\section{Metodologia}

\begin{frame}{InterSCityPlotter}
\end{frame}

\begin{frame}{Conjuntos de Dados}
\end{frame}

\begin{frame}{Visualização em Si}
\end{frame}

\begin{frame}{Validação}
\end{frame}

\section{Considerações Finais}

\begin{frame}{Plano de Trabalho}
\end{frame}

\section{References}

\begin{frame}[allowframebreaks]{References}
  \nocite{bronevetsky02, schmidt03:MSc, FSF:GNU-GPL, CORBA:spec, MenaChalco08, natbib, biblatex, eco:09}
  \printbibliography
\end{frame}

% Recapitulando
\begin{frame}{\insertshorttitle}
  \overview

  % \begin{center} acrescenta espaço vertical;
  % como possivelmente temos bem pouco espaço aqui,
  % vamos usar centering
  {%
    \centering\noindent%
    \url{https://gitlab.com/link-of-your-repository}\par
  }

\end{frame}

\showqrcode

\appendix

\begin{frame}{Extra info}
  \begin{itemize}
    \item It is often useful to have some extra slides addressing likely questions from the audience at the end of the presentation
    \item By putting them after the ``appendix'' command, they are not counted in the page count indicator
  \end{itemize}
\end{frame}
