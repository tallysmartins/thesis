%!TeX root=../apresentacao.tex
%("dica" para o editor de texto: este arquivo é parte de um documento maior)
% para saber mais: https://tex.stackexchange.com/q/78101/183146

% Apague as duas linhas abaixo (elas servem apenas para gerar um
% aviso no arquivo PDF quando não há nenhum dado a imprimir) e
% insira aqui o conteúdo do seu trabalho. Tome por base o
% arquivo corpo-apresentacao.tex do diretório conteudo-exemplo
% para a definição do título, autoria etc. e estrutura da
% apresentação.

\input{extras/aviso-conteudo}
\avisoApresentacao
%%%%%%%%%%%%%%%%%%%%%%%%%%%%%%%%% METADADOS %%%%%%%%%%%%%%%%%%%%%%%%%%%%%%%%%%%%

\title[The shortened title]{Visualização Dinâmica Multiescala\\ de Dados do Tráfego
de Veículos}
% \subtitle{The (optional) subtitle}

\author[Tallys Martins]{Tallys Martins}

%\institute[USP]{\textbf{Workshop Name} \\ Computer Science Department \\ IME USP}
\institute[USP]{
 \textbf{Orientador:} Fabio Kon \\
 \textbf{Co-Orientador:} Higor Amario de Souza \\
 Computer Science Department \\ IME USP
}

\date{Março de 2019}

% Coloca a imagem no fundo da página de título
\bgimage{\includegraphics[width=\paperwidth]{fundo_predios_e_grafo}}

% Logotipos no rodapé da página de título
\logos{
  \hfil\hfil\includegraphics[width=.1\textwidth]{usp-logo}\hfil%
  \raisebox{-.0103\paperheight}{\includegraphics[height=.0932\paperheight]{interscity-logo}}\hfil%
  \raisebox{-.033\paperheight}{\includegraphics[width=.07\textwidth,trim=0 230 0 0,clip]{ime-logo}}\hfil\hfil
}

%\logos{
%  \hfil\hfil\includegraphics[width=.1\textwidth]{usp-logo}\hfil%
%  \raisebox{-.0103\paperheight}{\includegraphics[height=.0932\paperheight]{interscity-logo}}\hfil%
%  \raisebox{-.00517\paperheight}{\includegraphics[height=.057\paperheight]{cnpq-logo}}\hfil%
%  \raisebox{-.0342\paperheight}{\includegraphics[height=.1035\paperheight]{capes-logo}}\hfil%
%  \includegraphics[height=.044\paperheight]{fapesp-logo}\hfil\hfil
%}

% Usado para criar o qrcode com o endereço da apresentação
\presentationurl{http://interscity.org}

% Inclui o qrcode no sumário da apresentação
\includeqrcodeintoc

% O slide de sumário pode ser dividido em colunas; o parâmetro
% determina após qual o número da seção fazer a quebra de coluna
% (use zero para uma coluna ou simplesmente omita este comando).
\toccolumns{4}


%%%%%%%%%%%%%%%%%%%%%%%%%%%%%%%%%%%%%%%%%%%%%%%%%%%%%%%%%%%%%%%%%%%%%%%%%%%%%%%%
%%%%%%%%%%%%%%%%%%%%%%%%%%%% INÍCIO DA APRESENTAÇÃO %%%%%%%%%%%%%%%%%%%%%%%%%%%%
%%%%%%%%%%%%%%%%%%%%%%%%%%%%%%%%%%%%%%%%%%%%%%%%%%%%%%%%%%%%%%%%%%%%%%%%%%%%%%%%

% É complicado colocar uma imagem de fundo, os logos das agências e
% o conteúdo "normal" do slide de título sem que as coisas fiquem
% bagunçadas, então definimos um comando para gerar o slide de título
\customtitlepage

% Slide com o qrcode
\showqrcode

\begin{frame}{Agenda}
  \overview
\end{frame}

\section{Introdução}

\begin{frame}{Contexto}
  ``Tráfego em uma via é o fluxo ou a passagem de veículos e pedestres por um
caminho que pode ser em regiões urbanas, mares, no ar e até no subsolo''

\hfill \citep{Chen2015}
\end{frame}

\begin{frame}{Contexto}
  \textbf{Dados de tráfego}
  \begin{itemize}
    \item Têm crescido com o avanço de sistemas de transporte
    \item Complexos e volumosos 
  \end{itemize}
\end{frame}

\begin{frame}{Contexto}
  \textbf{Visualizar esses dados}
  \begin{itemize}
    \item Trazem novas abstrações para entender seu comportamento
    \item Relação entre deslocamentos entre regiões
  \end{itemize}
\end{frame}

%\begin{frame}[plain]
%  \includegraphics[width=\textwidth]{interscity-logo}
%\end{frame}

\begin{frame}[standout]
  Bundling
\end{frame}

\begin{frame}{Objetivos}
  \begin{block}{Explorar o \emph{bundling} na visualização dos fluxos de origem-destino}
    \begin{enumerate}
      \item Como o bundling pode ser usado para identificar os fluxos de origem e des-
tino no trânsito em diferentes escalas?

      \item É possível utilizar o bundling para identificar padrões de fluxos de origem
e destino no trânsito?

      \item O bundling é eficiente para gerar uma visualização de uma grande quanti-
dade de dados do trânsito?
    \end{enumerate}
  \end{block}
\end{frame}

\section{Conceitos}

\begin{frame}{Leiaute}
\end{frame}

\begin{frame}{Bundling}
\end{frame}

\begin{frame}{InterSCSimulator}
\end{frame}


\section{Trabalhos Relacionados}

\begin{frame}{Trabalhos Relacionados}
\end{frame}

\section{Metodologia}

\begin{frame}{Metodologia}
\end{frame}

\section{Considerações}

\begin{frame}{Validação}
\end{frame}

\begin{frame}{Plano de Trabalho}
\end{frame}

\section{References}

\begin{frame}[allowframebreaks]{References}
  \nocite{bronevetsky02, schmidt03:MSc, FSF:GNU-GPL, CORBA:spec, MenaChalco08, natbib, biblatex, eco:09}
  \printbibliography
\end{frame}

% Recapitulando
\begin{frame}{\insertshorttitle}
  \overview

  % \begin{center} acrescenta espaço vertical;
  % como possivelmente temos bem pouco espaço aqui,
  % vamos usar centering
  {%
    \centering\noindent%
    \url{https://gitlab.com/link-of-your-repository}\par
  }

\end{frame}

\showqrcode

\appendix

\begin{frame}{Extra info}
  \begin{itemize}
    \item It is often useful to have some extra slides addressing likely questions from the audience at the end of the presentation
    \item By putting them after the ``appendix'' command, they are not counted in the page count indicator
  \end{itemize}
\end{frame}
