<<<<<<< HEAD
%!TeX root=../tese.tex
%("dica" para o editor de texto: este arquivo é parte de um documento maior)
% para saber mais: https://tex.stackexchange.com/q/78101/183146

% Apague as duas linhas abaixo (elas servem apenas para gerar um
% aviso no arquivo PDF quando não há nenhum dado a imprimir) e
% insira aqui o conteúdo do resumo e abstract do seu trabalho
=======
\begin{resumo}{port}
  Nos últimos anos, houve um crescimento na geração e disponibilização de uma
variedade de dados de mobilidade, como carros, ônibus, aviões, embarcações,
cujo conteúdo revela a dinâmica de fluxo entre diferentes regiões espaciais. No
trânsito, a posição dos veículos é mapeada em registros 2D, geralmente latitude
e longitude, registrando sua trajetória ao longo do tempo. A análise dessas
trajetórias tem diversas aplicações, como detecção de padrões de
origem-destino, monitoramento do fluxo ou a observação de eventos externos,
como mudanças climáticas, e seus impactos no tráfego. Com isso, gestores de uma
cidade conseguem mais informações para auxiliar na tomada de decisão e
planejamento do trânsito. Recentemente, técnicas avançadas para a análise de
trajetórias surgiram, permitindo a visualização de grandes conjuntos de dados
em diferentes níveis de detalhes. Uma dessas técnicas, o \emph{bundling},  ajuda a
reduzir a oclusão na visualização de trajetórias através do agrupamento de
segmentos espacialmente próximos, possibilitando que uma representação visual
comunique de maneira mais clara e intuitiva os padrões geográficos de
origem-destino desses objetos. Neste trabalho exploraremos como o \emph{bundling} pode
ser usado na análise e identificação de padrões de deslocamento de fluxos
provenientes de diferentes regiões de uma cidade e como esses fluxos impactam
em ruas e avenidas com grandes congestionamentos. Para isso, iremos aplicar
algoritmos de \emph{bundling} que favorecem a análise multiescala para obter uma
visão macroscópica e microscópica dos fluxos de deslocamento. Além disso,
investigaremos como a técnica ajuda na comparação da estrutura do tráfego em
condições diversas, como em diferentes dias de semana, dias de chuva e na
ocorrência de eventos, como o fechamento de vias. Nós usaremos o simulador
InterSCSimulator para gerar dados de tráfego de carros, ônibus e metrô na
cidade de São Paulo. A simulação será feita com base na pesquisa de
origem-destino (OD) realizada pela Companhia do Metropolitano de São Paulo
(Metrô). Usaremos também dados reais da frota de ônibus de São Paulo para
comparação da visualização com os dados simulados. Esperamos que o uso de
\emph{bundling} para visualização do tráfego possa colaborar  para o melhor
entendimento do trânsito e dar suporte à tomada de decisão por gestores do
transporte urbano, que precisam destinar os limitados recursos para manutenção
das vias da cidade, o que afeta diretamente a qualidade de vida dos cidadãos.
\end{resumo}
>>>>>>> 3faf015... Adds chapters and content

\begin{resumo}{eng}
  Nos últimos anos, houve um crescimento na geração e disponibilização de uma
variedade de dados de mobilidade, como carros, ônibus, aviões, embarcações,
cujo conteúdo revela a dinâmica de fluxo entre diferentes regiões espaciais. No
trânsito, a posição dos veículos é mapeada em registros 2D, geralmente latitude
e longitude, registrando sua trajetória ao longo do tempo. A análise dessas
trajetórias tem diversas aplicações, como detecção de padrões de
origem-destino, monitoramento do fluxo ou a observação de eventos externos,
como mudanças climáticas, e seus impactos no tráfego. Com isso, gestores de uma
cidade conseguem mais informações para auxiliar na tomada de decisão e
planejamento do trânsito. Recentemente, técnicas avançadas para a análise de
trajetórias surgiram, permitindo a visualização de grandes conjuntos de dados
em diferentes níveis de detalhes. Uma dessas técnicas, o bundling,  ajuda a
reduzir a oclusão na visualização de trajetórias através do agrupamento de
segmentos espacialmente próximos, possibilitando que uma representação visual
comunique de maneira mais clara e intuitiva os padrões geográficos de
origem-destino desses objetos. Neste trabalho exploraremos como o bundling pode
ser usado na análise e identificação de padrões de deslocamento de fluxos
provenientes de diferentes regiões de uma cidade e como esses fluxos impactam
em ruas e avenidas com grandes congestionamentos. Para isso, iremos aplicar
algoritmos de bundling que favorecem a análise multiescala para obter uma
visão macroscópica e microscópica dos fluxos de deslocamento. Além disso,
investigaremos como a técnica ajuda na comparação da estrutura do tráfego em
condições diversas, como em diferentes dias de semana, dias de chuva e na
ocorrência de eventos, como o fechamento de vias. Nós usaremos o simulador
InterSCSimulator para gerar dados de tráfego de carros, ônibus e metrô na
cidade de São Paulo. A simulação será feita com base na pesquisa de
origem-destino (OD) realizada pela Companhia do Metropolitano de São Paulo
(Metrô). Usaremos também dados reais da frota de ônibus de São Paulo para
comparação da visualização com os dados simulados. Esperamos que o uso de
bundling para visualização do tráfego possa colaborar  para o melhor
entendimento do trânsito e dar suporte à tomada de decisão por gestores do
transporte urbano, que precisam destinar os limitados recursos para manutenção
das vias da cidade, o que afeta diretamente a qualidade de vida dos cidadãos.
\end{resumo}
