%!TeX root=../tese.tex
%("dica" para o editor de texto: este arquivo é parte de um documento maior)
% para saber mais: https://tex.stackexchange.com/q/78101/183146

% Apague as duas linhas abaixo (elas servem apenas para gerar um
% aviso no arquivo PDF quando não há nenhum dado a imprimir) e
% insira aqui o conteúdo do resumo e abstract do seu trabalho
\begin{resumo}{port}
  Nos últimos anos, houve um crescimento na geração e disponibilização de uma
variedade de dados de mobilidade de veículos urbanos, tais como carros, ônibus,
aviões, embarcações, cujo conteúdo revela a dinâmica de fluxo entre diferentes
regiões espaciais. A posição desses objetos é geralmente mapeada em registros
2D, como latitude e longitude, registrando sua trajetória ao longo do tempo. A
análise dessas trajetórias tem diversas aplicações, como detecção de padrões de
origem-destino, monitoramento do fluxo ou a observação de eventos externos, como
mudanças climáticas, e seus impactos no tráfego. Com isso, gestores de uma
cidade conseguem mais informações para auxiliar na tomada de decisão e
planejamento do trânsito. No entanto, é um desafio visualizar grandes
quantidades de dados de conjuntos de dados de mobilidade. Desenhar grandes
quantidades de trajetórias diretamente em um mapa geralmente causa uma  oclusão
de dados, prejudicando a análise visual. Exibir os múltiplos atributos que
acompanham essas trajetórias é um desafio ainda maior. Uma abordagem para
resolver esse problema é o \emph{bundling}, que agrupa trajetórias espacialmente
próximas em uma representação simplificada. Nesta dissertação, adaptamos uma
técnica recente de \emph{bundling} para visualizar e analisar grandes
conjuntos de dados de mobilidade urbana. Nosso estudo de caso é baseado na
pesquisa de viagens da Região Metropolitana de São Paulo, uma das áreas de
tráfego mais intenso do mundo. Os resultados mostram que o \emph{bundling} ajuda
a identificar e analisar vários padrões de mobilidade para diferentes atributos
de dados, como horários de pico, renda e modos de transporte.
\end{resumo}

\begin{resumo}{eng} 
In recent years, there has been an increase in the generation and availability
of a variety of urban vehicle mobility data such as cars, buses, airplanes,
boats and others.  This data reveals the flow dynamics between different spatial
regions. Objects positions are usually mapped into 2D records containing their
latitude and longitude over time. This type of data leads to the field of
trajectory analysis and has several applications, such as origin-destination
flows studies, recomendation systems and traffic monitoring, which can help
managers to gain insights to support decision making and traffic planning.
However, it is challenging to visualize huge amounts of data from mobility
datasets. Plotting raw trajectories on a map often causes data occlusion,
impairing the visual analysis. Displaying the multiple attributes that these
trajectories come with is an even larger challenge. One approach to solve this
problem is trail bundling, which groups motion trails that are spatially close
in a simplified representation. In this study, we augment a recent bundling
technique to support multi-attribute trail datasets for the visual analysis of
urban mobility. Our case study is based on the travel survey from the São Paulo
Metropolitan Area, which is one of the most intense traffic areas in the world.
The results show that bundling helps the identification and analysis of various
mobility patterns for different data attributes, such as peak hours, social
strata, and transportation modes.
\end{resumo}
