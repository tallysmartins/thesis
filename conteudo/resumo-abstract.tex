<<<<<<< HEAD
%!TeX root=../tese.tex
%("dica" para o editor de texto: este arquivo é parte de um documento maior)
% para saber mais: https://tex.stackexchange.com/q/78101/183146

% Apague as duas linhas abaixo (elas servem apenas para gerar um
% aviso no arquivo PDF quando não há nenhum dado a imprimir) e
% insira aqui o conteúdo do resumo e abstract do seu trabalho
=======
\begin{resumo}{port}
  Nos últimos anos, houve um crescimento na geração e disponibilização de uma
variedade de dados de mobilidade de veículos urbanos, tais como carros, ônibus, aviões, embarcações,
cujo conteúdo revela a dinâmica de fluxo entre diferentes regiões espaciais. No
trânsito, a posição dos veículos é mapeada em registros 2D, geralmente latitude
e longitude, registrando sua trajetória ao longo do tempo. A análise dessas
trajetórias tem diversas aplicações, como detecção de padrões de
origem-destino, monitoramento do fluxo ou a observação de eventos externos,
como mudanças climáticas, e seus impactos no tráfego. Com isso, gestores de uma
cidade conseguem mais informações para auxiliar na tomada de decisão e
planejamento do trânsito. Recentemente, técnicas avançadas para a análise de
trajetórias surgiram, permitindo a visualização de grandes conjuntos de dados
em diferentes níveis de detalhes. Uma dessas técnicas, o \emph{bundling},  ajuda a
reduzir a oclusão na visualização de trajetórias através do agrupamento de
segmentos espacialmente próximos, possibilitando que uma representação visual
comunique de maneira mais clara e intuitiva os padrões geográficos de
origem-destino desses objetos. Neste trabalho exploraremos como o \emph{bundling} pode
ser usado na análise e identificação de padrões de deslocamento de fluxos
provenientes de diferentes regiões de uma cidade e como esses fluxos impactam
em ruas e avenidas com grandes congestionamentos. Para isso, iremos aplicar
algoritmos de \emph{bundling} que favorecem a análise multiescala para obter uma
visão macroscópica e microscópica dos fluxos de deslocamento. Além disso,
investigaremos como a técnica ajuda na comparação da estrutura do tráfego em
na ocorrência de eventos atípicos, como o bloqueio de vias. Nós usaremos o simulador
InterSCSimulator para gerar dados do tráfego de veículos na
cidade de São Paulo e também dados reais da frota de ônibus de São Paulo para
comparação da visualização com os dados simulados. Esperamos que o uso de
\emph{bundling} para visualização do tráfego possa colaborar  para o melhor
entendimento do trânsito e dar suporte à tomada de decisão por gestores do
transporte urbano, que precisam destinar os limitados recursos para manutenção
das vias da cidade, o que afeta diretamente a qualidade de vida dos cidadãos.
\end{resumo}
>>>>>>> 3faf015... Adds chapters and content

\begin{resumo}{eng} 
In recent years, there has been an increase in the generation and availability
of a variety of urban vehicle mobility data such as cars, buses, airplanes,
boats and others.  This data reveals the flow dynamics between different
spatial regions. At the transit, vehicles positions are usually mapped into 2D
registers containing their latitude and longitude over time.  This type of data
lead us to the field of trajectory analysis and has several applications, such
as origin-destination flows studies, recomendation systems and traffic
monitoring, which can help managers to gain insights to support decision making
and traffic planning.  Recently, advanced techniques for the analysis of
trajectories appeared, making possible to visualize large datasets in different
levels of detail.  One of these techniques, called bundling, helps to reduce
the occlusion in the visualization of trajectories by grouping spatially close
elements to make geographical patterns of these objects clearer.  In this
research project we will explore the use of bundling in the analysis of
origin-destination flows in a city and how these flows impact streets and
avenues with great congestions.  We will apply bundling algorithms that favor
multiscale analysis to obtain a macroscopic and microscopic view of the traffic
flows. Besides that, we will investigate how our technique allows the
comparission of the traffic structure in the occurrence of atypical events,
such as street clojures. We will use the InterSCSimulator tool to generate
simulated traffic data from vehicles in the city of São Paulo. Also, we will
use data of bus trips from the São Paulo public transportation system. We hope
that the use of bundling to visualize the origin-destination flows may
collaborate for a better understanding of the traffic using real and simulated
data and support decision making by transport managers, which need to allocate
the limited resources for maintenance of the city infrastructure and provide
solutions to their citizens.
\end{resumo}
