\chapter{Referencial teórico}
\label{cap:referencial-teorico}

  Neste capítulo esclarecemos os conceitos teóricos necessários para o
entendimento do trabalho e a metodologia proposta. Inicialmente, abordamos
aspectos da visualização de dados do tráfego, que abrange uma variedade de
cenários conforme os objetivos, tipos de dado e diferentes representações
visuais da informação. Neste contexto, destacamos o \emph{bundling}, uma
técnica bastante usada na visualização de grandes quantidades de dados do
tráfego. Apresentamos os detalhes da técnica e uma breve consideração sobre os
avanços recentes nesta área. Por fim, considerando que neste trabalho faremos o
uso de dados simulados do tráfego de veículos, apresentamos o simulador
InterSCSimulator.

\section{Visualização de dados do tráfego}

  A análise de dados do tráfego tem um histórico antigo. \citet{Chen2015} fazem
um levantamento de várias pesquisas de análise e visualização de dados do
tráfego de pessoas, carros e embarcações. Eles apresentam uma taxonomia
conforme algumas características observadas, como o objetivo da visualização,
formato dos dados do tráfego e formas de apresentação de suas propriedades.
Nesta seção apresentamos essa taxonomia de forma breve, que é importante para
um melhor entendimento da área e dos esforços que têm sido feitos no campo da
visualização de dados do tráfego.

\subsection{Quanto ao Objetivo da Visualização}

  Segundo \citet{Chen2015}, os trabalhos e sistemas de software sobre análise e
visualização de dados do tráfego podem ser classificadas em quatro categorias
de acordo com seus objetivos e a tarefa a que servem.

\begin{description}
  \item[Monitoramento do tráfego:] esse tipo de aplicação
foca no monitoramento em tempo real para descobertas instantâneas de eventos no
tráfego, como câmeras de gravação ao vivo e sistemas de alerta.

  \item[Descoberta de padrões e clusterização:] para a descoberta de padrões e
tendências de mobilidade, algumas aplicações utilizam métodos para o
agrupamento visual de trajetórias ou novas abstrações dos dados.

  \item[Exploração e predição:] outras aplicações focam em fornecer mecanismos
de pesquisa e exploração dos dados para que os usuários investiguem os dados
que justifiquem situações do tráfego, como congestionamentos, ou ainda técnicas
de predição para prever essas situações (e.g. predizer que haverá
congestionamento devido a ocorrência de chuva).

  \item[Planejamento de rotas e recomendação:] sistemas de transporte
inteligentes contam essencialmente com recomendações de trajetos para usuários
do transporte, várias soluções existem com esse propósito, que podem envolver monitoramento
em tempo real ou análise histórica no processo de recomendação.
\end{description}

\subsection{Quanto ao Tipo de Coleta dos Dados}

  Dados do tráfego são tipicamente coletados a partir de sensores e dispositivos
eletrônicos. A estrutura desses conjuntos de dados dependem do modo de operação
dos sensores e dispositivos que registram a movimentação dos objetos. \citet{Chen2015}
os dividem em três classes:

\begin{description}
  \item[Baseado em pontos de interesse:] a posição de um objeto é gravada assim
que ele entra na área do sensor. Como uma câmera de vídeo que capta a
movimentação e orientação de um veículo assim que ele passa pela área de
monitoramento.

  \item[Baseado em ações:] as informações sobre um objeto podem estar associados
a certas atividades. O usuário de um aparelho celular por exemplo, tem sua
atividade e localização registrados pela rede GSM quando o mesmo efetua uma
chamada.

  \item[Baseado em sinais de dispositivos:] Um dispositivo de localização
carregado por um objeto constantemente grava e envia suas informações de
localização para uma central, como aparelhos de GPS instalados em um ônibus do
transporte público de uma cidade.  \end{description}

  Uma série de posições de localização ao longo do tempo formam a trajetória de
um objeto. Uma trajetória contém informações temporais, que permitem traçar uma
linha do tempo da movimentação, e informações espaciais, que grava a posição em
cada momento no tempo.    A frequência de armazenamento dessas informações também impacta a análise.
Capturar e armazenar essas informações com um intervalo de tempo muito pequeno
pode ser algo bastante custoso. Outro fator de impacto é a precisão das
informações que dependente altamente de questões relacionadas ao hardware
utilizado na coleta, o que pode requerer etapas de pré-processamento para
extrair dados inconsistentes da análise.

\subsection{Visualização das propriedades espaciais}

  As propriedades espaciais de localização são o principal componente de dados
de tráfego. Elas referem-se aos locais onde ações, incidentes e eventos
ocorrem. Diferentes níveis de agregação dessa informação levam a categorização
da visualização também em três classes, segundo \citet{Chen2015}: visualização
baseada em pontos (nenhuma agregação), visualização baseada em linhas
(agregação de primeira ordem), e agregação baseada em regiões (agregação de
segunda ordem).

\begin{description}
  \item[Visualização baseada em pontos:] visualizações baseadas em pontos
consideram as informações do tráfego como pontos discretos e usam sua forma
pura como representação, como por exemplo pontos em um mapa 2D. Essa técnica
mostra intuitivamente a posição de objetos em um certo momento no tempo. O
projeto \emph{Trains of Data}, ilustrado na Figura \ref{fig:trains-of-data},
utiliza esse tipo de visualização para representar a movimentação dos trens na
França. O tamanho dos pontos indica a quantidade de passageiros e a cor indica
se o trem está atrasado, sendo verde no horário e vermelho significa atraso.
Para uma visualização integral da trajetória eles ainda usam efeitos de
animação. Visualizações baseadas em pontos tipicamente representam cada ponto
de forma individual. Em casos onde há uma grande quantidade de pontos, o uso de
mapas de calor é indicado para a visualização da densidade. A vantagem desse
método é que ele permite observar os estados de cada objeto e sua distribuição
no espaço, como também explorar regiões da cidade que estejam mais ocupadas.
Por outro lado, ele é inapropriado para representação de informações contínuas,
como quantos veículos viajam de um determinado local para outro.

\begin{figure}[!htb]
  \centering
  \includegraphics[width=.6\textwidth]{../figuras/trains-of-data.jpg}
  \caption[Visualização baseada no mapa das ferrovias da França]{Posição dos trens às 19:43 na França com uma visualização baseada no mapa
das ferrovias. Fonte: \citet{Senseable2018}.}
  \label{fig:trains-of-data}
\end{figure}

  \item[Visualização baseada em linhas:] visualizações baseadas em linhas são
desenhadas para mostrar a trajetória de objetos, mapas de vias e estradas em
uma região, ou fluxos do tráfego em uma rede de transporte. As trajetórias e
fluxos são representados por linhas ou curvas e são escaladas ou coloridas de
acordo com suas propriedades (i.e. densidade, direção, velocidade).
\citet{Klein2013} apresenta um sistema iterativo para análise do tráfego aéreo
na França. Cada trajetória é representada por uma linha que conecta os pontos
de partida e chegada de cada voo, como pode ser visto na Figura
\ref{fig:air-traffic}.

\begin{figure}[!htb]
  \centering
  \includegraphics[width=1\textwidth]{../figuras/air-traffic.png}
  \caption[Visualização do tráfego aéreo na França]{Visualização do tráfego aéreo na França. Fonte: \citet{Klein2013}.}
  \label{fig:air-traffic}
\end{figure}

  A representação espacial das linhas pode ainda sofrer transformações
geométricas e topológicas que geram novas abstrações. Por exemplo,
\citet{Tarik2009} mostram uma proposta que transforma as trajetórias de um
leiaute espacial para um leiaute abstrato para visualizar a movimentação de
pessoas durante a fuga de uma explosão em um escritório.  A abordagem abstrata
é usada para mostrar mais efetivamente os padrões de movimentação de pessoas
que se movem ao mesmo tempo para as mesmas áreas. A Figura \ref{fig:tarik}
mostra essa abstração, com ela é possível notar a movimentação de indivíduos
antes da explosão, o que sugere possíveis suspeitos ou testemunhas do evento.
Além disso, as informações temporais são difíceis de representar no leiaute
espacial, enquanto no leiaute abstrato o eixo X denomina a sequência do tempo e
o eixo Y os locais no espaço.

\begin{figure}[ht!]
  \centering
  \begin{subfigure}[t]{0.45\textwidth}
    \centering
    \includegraphics[width=65mm]{../figuras/proximidade-espacial.png}
    \caption{Visualização espacial das trajetórias ao longo do tempo. \label{fig:viz-espacial}}
  \end{subfigure}
  ~
  \begin{subfigure}[t]{0.45\textwidth}
    \centering
    \includegraphics[width=75mm]{../figuras/proximidade-abstrata.png}
    \caption{Visualização abstrata da movimentação ao longo do tempo. \label{fig:viz-abstrata}}
  \end{subfigure}

  \caption[Visualização cartesiana vs abstrata da movimentação de pessoas em um
escritório]{Simulação de uma evacuação de um escritório depois de uma explosão.
(a) Visualização da movimentação da trajetória das pessoas no espaço. (b)
Visualização abstrata, baseada na proximidade.  Fonte: \citet{Tarik2009}.
\label{fig:tarik}}
\end{figure}

  No entanto, a medida que o número de objetos cresce, aumenta também os
problemas de oclusão, o que acaba afetando a estética da visualização e a
obtenção de informações sobre os dados, \citep{Zhou2013}. Uma forma de reduzir
a complexidade de análise de um grande conjunto  trajetórias é utilizando
outras formas de abstração dos dados, como uma visualização baseada em regiões.

  \item[Visualização baseada em regiões:] Esse tipo de abstração agrupa fluxos
de deslocamento com origem e destino similares em um nível de macro regiões,
geralmente determinado por divisões administrativas.  \citet{Zeng2013}
apresentam um diagrama em círculos para mostrar padrões de movimentação de
pessoas entre as regiões da cidade. O círculo representa a junção ou conexão
entre as diferentes regiões e o fluxo entre elas. A densidade do fluxo é medido
pela espessura, bem como a direção é destacada dentro do círculo, como é
ilustrado na Figura \ref{fig:interchange-circo}. Apesar de facilitar a
visualização, esse tipo de agregação abre mão das informações individuais de
cada trajetória em favor de uma visão condensada dos fluxos.

\begin{figure}[!htb]
  \centering
  \includegraphics[width=1\textwidth]{../figuras/region-based.png}
  \caption[Exemplo de visualização baseada em regiões do sistema de metrô na França]{Exemplo de visualização baseada em regiões: padrões regionais de movimentação no sistema de metrô na França. Fonte: \citet{Zeng2013}.}
  \label{fig:interchange-circo}
\end{figure}
\end{description}

\section{\emph{Bundling}}
\label{sec:bundling}

  Um grande problema na visualização de grafos e dados de trajetórias é a
oclusão visual a medida que a quantidade de elementos aumenta. Uma pequena
quantidade de dados já começa a apresentar problemas de sobreposição e
cruzamento de arestas, o que dificulta a obtenção de informação sobre os dados.
Uma maneira de contornar esse problema é através de filtros que determinam a
quantidade de itens na visualização. Como consequência do filtro perde-se a
visão global de todos os itens ao mesmo tempo, o que pode ser necessário para a
identificação das correlações e padrões entre eles. Uma outra abordagem é o uso
de agregações que geram novas abstrações dos dados. Na seção anterior mostramos
como o agrupamento em regiões projetado sobre um diagrama circular simplifica a
visualização das trajetórias entre as áreas da cidade.

 Uma solução que tem sido amplamente utilizada em várias pesquisas de
visualização de dados de tráfego é o uso de uma técnica chamada
\emph{bundling}. A técnica ajuda a simplificar o desenho da visualização
através da agregação espacial das arestas em conjuntos chamados
\emph{bundles}, similar a algoritmos de clusterização. Os \emph{bundles} são
definidos como um grupo de arestas similares, compatíveis o suficiente para
serem representadas por um corpo único e compacto \citep{Lhuillier2017}. A
compatibilidade é calculada a partir de uma função de similaridade usada para
determinar quais arestas devem fazem parte do mesmo agrupamento.
Imaginando então algumas viagens no trânsito, podemos agrupá-las pela sua
região de origem, destino, distância percorrida, direção ou até mesmo o meio de
transporte utilizado.  Dessa forma, o número de \emph{bundles} é ligeiramente
inferior ao número de trajetórias a serem desenhadas, ficando mais simples
compreender e visualizar a estrutura global, padrões e tendências entre grupos
de trajetórias que ligam áreas fortemente relacionadas \citep{Zhou2013}.  A
Figura \ref{fig:bundling-hierarquico} mostra o uso de uma técnica de
\emph{bundling} para visualização da hierarquia de módulos e arquivos de um
software onde as arestas mostram suas relações de dependência.

\begin{figure}[!htb]
  \centering
  \includegraphics[width=1\textwidth]{../figuras/hierarquical-edge-bundling.png}
  \caption[Bundling hierárquico na visualização de dados de software]{Bundling hierárquico na visualização de dados de software. Fonte: \citep{Holten2006}.}
  \label{fig:bundling-hierarquico}
\end{figure}

\subsection{Modelos de Bundling}

  O cálculo de um \emph{bundle} não possui uma definição precisa, e por isso há
uma grande variedade de algoritmos e abordagens para modelar problemas de
agrupamento de arestas com \emph{bundling}, que podem variar significativamente
de um para outro em questões de complexidade e aplicação dos algoritmos
\citep{Zhou2013}.  \emph{Force-directed edge bundling} (FDEB), apresentado em
\citet{Holten2009}, cria \emph{bundles} através da atração entre pontos de
controle colocados ao longo das arestas e é considerado um modelo baseado em
custo, já que tenta minimizar o valor de uma função de custo que representa a
força de atração entre as arestas.  \emph{Hierarchical Edge Bundling} (HEB)
agrupa as arestas baseadas na estrutura hierárquica do grafo para o cálculo dos
\emph{bundles} e por isso é considerado um modelo geométrico \citep{Holten2006}. Outra
classificação são os modelos baseados em imagem, como \emph{Skeleton-based edge
bundling} (SBEB) e \emph{Kernel Density Estimation-based Edge Bundling}
(KDEEB), que surgiram mais recentemente. Eles utilizam algoritmos de
clusterização para extrair a estrutura geral do grafo para então calcularem os
\emph{bundles}. O algoritmo SBED utiliza o esqueleto do grafo como guias para
criar \emph{bundles} com muitas ramificações \citep{Ersoy2011}, enquanto o
KDEEB utiliza um mapa de densidade do grafo para extrair sua estrutura no
espaço da imagem \citep{Hurter2012}. A Figura \ref{fig:subfigures} dá um
exemplo da variedade de possíveis saídas geradas por diferentes algoritmos de
\emph{bundling} (FDEB, SDEB e KDEEB) aplicados nos mesmos dados.  Nesse exemplo
são usados dados de uma semana do tráfego aéreo sobre os EUA.

\begin{figure}[!htb]
  \centering
  \begin{subfigure}{0.6\textwidth}
    \centering
    \includegraphics[width=1\textwidth]{../figuras/FDEB.png}
    \caption{FDEB}
    \label{fig:FDEB}
  \end{subfigure}

  \begin{subfigure}{0.6\textwidth}
    \centering
    \includegraphics[width=1\textwidth]{../figuras/SBEB.png}
    \caption{SBEB}
    \label{fig:SBEB}
  \end{subfigure}

  \begin{subfigure}{0.6\textwidth}
    \centering
    \includegraphics[width=1\textwidth]{../figuras/KDEEB.png}
    \caption{KDEEB}
    \label{fig:KDEEB}
  \end{subfigure}
  \caption[Diferentes algoritmos de bundling aplicadas nos mesmos dados]{Diferentes algoritmos de bundling aplicadas em dados do tráfego aéreo dos EUA - 235 nós, 2099 arestas. Fonte: \citep{Klein2013}.}
  \label{fig:subfigures}
\end{figure}

  A escolha do modelo geralmente parte da observação e experiências de
aplicação em problemas semelhantes, já que conhecer e testar diferentes
abordagens e suas variações não é uma tarefa simples. Por isso,
\citet{Lhuillier2017} recentemente propôs uma nova taxonomia para os modelos e
algoritmos de bundling dividindo-os com base no tipo de dados em que se deseja
aplicar a técnica. Eles justificaram que desta forma pesquisadores e usuários
da técnica pudessem focar no seu contexto de aplicação e não em detalhes
específicos de implementação algorítmica. Seu estudo dividiu os modelos
existentes inicialmente em dois grupos, conforme o tipo de dados, grafos ou
trajetórias. Então, mais detalhes sobre a direção (direcionado ou não),
dimensão (2D ou 3D) e dependência do tempo (dinâmico ou estático) refinam a
divisão dentro da sua taxonomia. 

\begin{figure}[!htb]
  \centering
  \includegraphics[width=160mm]{../figuras/estado-da-arte-crop.pdf}
  \caption{Taxonomia dos métodos de bundling baseado no tipo de dado. Fonte: \citet{Lhuillier2017}}
  \label{table:bundling-methods}
\end{figure}

 O algoritmo KDEEB, apresentado em \citet{Hurter2012}, é um eficiente algoritmo
de \emph{bundling} que demonstrou resultados em cerca de uma ordem de magnitude
a mais em velocidade em relação a métodos anteriores. \citet{Klein2013} aplicam
esse algoritmo na análise de grandes quantidades de dados do tráfego aéreo. Na
seção a seguir discutimos mais sobre as características desse modelo e como
nos beneficiamos dele em nossa análise do trânsito na cidade.

\subsection{Modelo Baseado em Imagem}

  Recentemente, uma classe de algoritmos de \emph{bundling} baseados em imagem
aumentaram as possibilidades de análise de dados do tráfego, conseguindo aumentar
a escala em milhares de dados. Esse tipo de algoritmo utiliza etapas de pré-processamento
no espaço da imagem para otimizar o cálculo dos \emph{bundles}. Além disso,
mapeiam naturalmente para implementações paralelas e podem fazer uso do poder
computacional de placas gráficas. O algoritmo KDEEB . . . . FINISH

\subsection{\emph{Bundling} em Dados Dinâmicos}

  Um grafo dinâmico é aquele no qual seus nós e arestas são dependentes do
tempo, ou seja, a cada momento \emph{t}, existe um grafo G(t) diferente a ser
explorado. Nesse contexto, dois diferentes tipos de grafos dinâmicos podem ser
considerados, grafos sequenciais e grafos de fluxo \citep{Hurter2013}. Um
grafo sequencial consiste em um conjunto de grafos $G^i = (V^i, E^i)$, onde
cada grafo $G^i$ é como um retrato no momento $i$ de um sistema que é
dependente do tempo. Grafos de fluxo, por outro lado, são definidos como um
conjunto de vértices $V$ e arestas $E$, onde as arestas são definidas pelo seu
tempo de vida e nós. Grafos de fluxo não possuem uma sequência pré-definida e
são tipicamente obtidos de fontes de dados online.

  O uso de \emph{bundling} nesse cenário é similar a outros métodos de
visualização de grafos dinâmicos. Para grafos sequenciais o \emph{bundling} é
aplicado a cada grafo $G^i$ de maneira estática. Já em grafos de fluxo, os
dados são divididos em janelas de tempo e o algoritmo é aplicado em cada
intervalo $\Delta t$. \cite{Hurter2014} mostram as operações de \emph{bundling} em
grafos dinâmicos e como o uso de animações pode ser eficiente na visualização
das mudanças em longas séries temporais.


%  \cite{Lhuillier2017} aponta que a utilização de \emph{bundling} em dados de
%trajetórias deve ser feita com cuidado, já que a distorção espacial pode afetar
%a informação de localidade. Muitas técnicas de \emph{bundling} lidam com esse problema
%oferecendo vários parâmetros para controle do nível de \emph{bundling} de forma
%que o resultado não fique tão discrepante do desenho original. Por exemplo,
%CUBu \citep{VanDerZwan2016}, um algoritmo que permite o controle em vários
%aspectos, como a resolução da imagem usada pelo estimador de densidade, raio do
%núcleo do estimador, número de pontos de controle ao longo da trajetória,
%número de iterações do \emph{bundling}, e outras configurações. Existe, no
%entanto, um desafio em encontrar parâmetros adequados. O controle do \emph{bundling}
%fica ainda mais difícil quando existem parâmetros acoplados, onde o mesmo
%resultado pode ser obtido alterando-se diferentes parâmetros para diferentes
%valores.

\section{Simulação do Tráfego de Veículos}

  O crescimento da população nas cidades ao redor do mundo trouxe também vários
desafios de gestão e controle do trânsito, especialmente em grandes cidades,
que precisam lidar com vários problemas de congestionamentos, acidentes,
poluição, controle do transporte público, entre outros. Uma maneira alternativa
para estudar e solucionar esses problemas é através do uso de ferramentas de
simulação. Essas ferramentas permitem criar cenários do tráfego de veículos e
ajudam nos estudos sobre seu comportamento e na elaboração de soluções para
serem aplicadas por gestores das cidades. Recentemente, novas ferramentas de
simulação com foco em escalabilidade foram desenvolvidas, permitindo testes e
experimentos em escala real de uma grande cidade, com milhões de veículos. O
InterSCSimulator é uma destas ferramentas e o detalhamos a seguir.

\subsection{InterSCSimulator}
\label{sec:interscsimulator}

\citet{mabs2017} apresentam o InterSCSimulator, um ferramenta de código livre,
capaz de executar simulações com mais de 4 milhões de veículos em uma grande
metrópole.  As simulações são feita em escala mesoscópica. Isso significa que
ela segue um modelo onde cada veículo do trânsito é simulado individualmente. O
simulador calcula para cada um deles as ações de movimentação durante sua
viagem.

\subsubsection{Componentes do InterSCSimulator}

  O simulador possui quatro componentes principais: o \textbf{cenário}, que
recebe os arquivos de entrada e cria o grafo da cidade e os primeiros veículos;
\textbf{o motor de simulação}, que executa os algoritmos e modelos da simulação
para gerar os arquivos com os resultados; uma componente de
\textbf{visualização do mapa}, que recebe o arquivo de saída com os resultados
e gera uma visualização da simulação baseada em pontos; e finalmente um
componente de \textbf{visualização de gráficos} que gera uma série de gráficos
com informações sobre o cenário simulado, como velocidade média dos veículos ao
longo do tempo e outros. A Figura \ref{fig:interscsimulator} ilustra a
interação entre os componentes e os arquivos de entrada e saída.

\begin{figure}[!htb]
  \centering
  \includegraphics[width=160mm]{../figuras/interscsimulator.png}
  \caption{Componentes do InterSCSimulator. Fonte: \citet{mabs2017} \label{fig:interscsimulator}}
\end{figure}

\subsubsection{Arquivos de entrada}
  O simulador utiliza três arquivos de entrada para definição do cenário, mas
apenas dois merecem ser detalhados, são eles \emph{map.xml} e \emph{trips.xml}.
O primeiro  contém a descrição da rede rodoviária da cidade e é construído com
dados obtidos de serviços como \emph{Open Street Maps}
(OSM)\footnote{\rurl{www.openstreetmap.org}}. O arquivo é composto por nós e
links, que representam cruzamentos e ruas, respectivamente. A Listagem
\ref{map.xml} mostra um exemplo de um arquivo \emph{map.xml} com 3 nós e 3
links.

\begin{lstlisting}[style=myxml, caption={Exemplo de arquivo map.xml que define a rede rodoviária da cidade. Fonte: \citet{mabs2017}}, label=map.xml]
<network>
  <nodes>
   <node id ="1" x="-46.65805" y="-23.58162"/>
   <node id ="2" x="-46.65828" y="-23.58342"/>
   <node id ="3" x="-46.65228" y="-23.59341"/>
  </nodes>
  <links>
    <link id="35985" from="1" to="2" length="100" free speed="40"/>
    <link id="35985" from="2" to="3" length="200" free speed="40"/>
    <link id="35985" from="3" to="1" length="80"  free speed="50"/>
  </links>
</network>
\end{lstlisting}

  O segundo arquivo de entrada, \emph{trips.xml}, descreve todas as viagens que devem ser
simuladas. Cada viagem deve informar os nós de origem e destino e o horário da
simulação em que ela será iniciada, como pode ser visto na Listagem
\ref{trips.xml}. A trajetória pode ainda ser fixada previamente ou calculada
pelo simulador em tempo de simulação, o qual fica responsável por determinar o
melhor caminho.

\begin{lstlisting}[style=myxml, caption={Exemplo de arquivo trips.xml que define a rede rodoviária da cidade. Fonte: \citet{mabs2017}}, label=trips.xml]

<scsimulator_matrix>
  <trip origin="247951669" destination="60641382"
    type="car" start_time="28801"/>
  <trip origin="60641382" destination="247951669"
    type="car" start_time="63001"/>
  <trip origin="4511105625" destination="2109902387"
    type="car" start_time="16201"/>
  <trip origin="247951669" destination="60641382"
    type="car" start_time="54001"/>
  <trip origin="246650787" destination="247951670"
    type="car" start_time="54001"/>
  <trip origin="247951670" destination="246650787"
    type="car" start_time="66601"/>
  <trip origin="246650787" destination="60641382"
    type="bus" start_time="54001"/>
</scsimulator_matrix>
\end{lstlisting}

  Cada veículo (carro ou ônibus) definido no arquivo \emph{trips.xml} é um ator
dentro da simulação.  Então, ele percorre um trajeto dos nós de origem até os
nós de destino através da rede rodoviária da cidade. Os veículos apresentam quatro
ações dentro da simulação: \emph{Start}, quando o tempo da simulação atinge o
tempo de início do veículo; \emph{Move}, quando a simulação atinge o tempo do
próximo movimento do veículo; \emph{Wait}, quando o veículo tem que esperar até
o próximo movimento; e \emph{Finish}, quando o veículo chega ao seu destino.

  Atualmente é utilizado um modelo de fluxo livre para calcular o tempo em que
um veículo gasta para percorrer um link: $tempo = tamanho\_do\_link/velocidade\_do\_veículo$.
Cada link mantém o número de veículos que estão na rua em um dado instante. Se o número
de carros no link é igual à sua capacidade, então nenhum outro veículo pode entrar naquele
link até que pelo menos um veículo saia, o que caracteriza um congestionamento
dentro da simulação. 

\subsubsection{Arquivos de saída}

O InterSCSimulator gera um arquivo XML contendo todos os eventos ocorridos
durante a simulação. A Listagem \ref{output.xml} mostra parte de um arquivo de
saída gerado durante uma simulação. É possível ver os eventos de
\emph{start\_trip}, \emph{move} e \emph{finish\_trip} para dois veículos com os
identificadores $2121$ e $2223$.

\begin{lstlisting}[style=myxml, caption={Exemplo de arquivo de saída output.xml com os eventos da simulação.}, label=output.xml]

<events version="1.0">
  <event time="4" type="start_trip" vehicle="2121" link="5243" />
  <event time="4" type="start_trip" vehicle="2223" link="1002" />
  <event time="11" type="move" vehicle="2223" link="4005" />
  <event time="31" type="move" vehicle="2121" link="4005" />
  <event time="38" type="move" vehicle="2223" link="2007" />
  <event time="52" type="finish_trip" vehicle="2121" link="4005" />
  <event time="52" type="finish_trip" vehicle="2223" link="5243" />
</events>
\end{lstlisting}
