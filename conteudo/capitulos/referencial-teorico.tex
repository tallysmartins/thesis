\chapter{Referencial Teórico e Trabalhos Relacionados}
\label{cap:referencial-teorico}

  A seguir nós discutimos conceitos e trabalhos relacionados sobre a visualização de
trajetórias com foco em dados de Origem-Destino.

\section{Visualização de Trajetórias}
\label{sec:dados-de-trajetorias}

\emph{Trajetórias} (aqui também nomeado como fluxos) são tipicamente conjuntos
de pontos no espaço $\mathbf{t} = \{\mathbf{x}_i\}$ capturados em momentos
consecutivos $t_i$ e que descrevem o movimento de um objeto ao longo do tempo.
Dados de Origem-Destino (OD) são um caso particular de trajetórias que contém
apenas o primeiro e o último ponto. Uma trajetória pode conter outras
informações, capturadas em cada passo $\mathbf{x_i}$, \emph{e.g.}, velocidade,
ou ainda atributos globais para a trajetória completa, \emph{e.g.}, tipo de
veículo.

Dados de trajetórias podem ser diretamente desenhados sobre mapas 2D,
e outros atributos representados através de elementos visuais como cor e textura.
As propriedades espaciais de localização, que referem-se
aos locais onde ações, incidentes e eventos ocorreram, é a principal característica
desse tipo de dado e pode ser agregada de diferentes formas, as quais \citet{Chen2015}
separa em três categorias: \emph{visualização baseada  em pontos} (nenhuma agregação), \emph{visualização baseada em linhas}
(agregação de primeira ordem), e \emph{agregação baseada em regiões} (agregação de segunda ordem).


Visualizações baseada em pontos (nenhuma agregação)
consideram as informações do tráfego como pontos discretos e usam sua forma
pura como representação em um leiaute espacial, como pontos individuais em um mapa 2D.
Essa tipo de visualização mostra a posição instantânea dos objetos em um certo momento
no tempo. A vantagem desse método é que ele mostra um retrato dos estados de cada objeto em um
determinado momento, permitindo observar sua distribuição no espaço e explorar regiões que estejam
mais ocupadas. Para entender mudanças contínuas ao longo do tempo, técnicas de animação podem
ser empregadas para mostrar uma transição entre os estados. Porém, com um grande número de partículas
- alguns milhares de pontos - esse tipo de visualização começa a sofrer
de um efeito conhecido como como movimento Browniano - onde o movimento dos pontos parecem ser movimentos
aleatórios - o que torna difícil identificar a trajetória dos objetos. O projeto \emph{Trains of Data},
ilustrado na Figura \ref{fig:trains-of-data}, utiliza esse tipo de visualização para
representar a movimentação dos trens na França. O tamanho dos pontos indica a quantidade de passageiros e a cor
sua assiduidade, verde se o trem está no horário e vermelho significa que está atrasado
em relação ao tempo estimado.

\begin{figure}[!htb]
  \centering
  \includegraphics[width=.6\textwidth]{../figuras/trains-of-data.jpg}
  \caption[Visualização baseada no mapa das ferrovias da França]{Posição dos trens às 19:43 na França com uma visualização baseada no mapa
das ferrovias. Fonte: \citet{Senseable2018}.}
  \label{fig:trains-of-data}
\end{figure}

Visualizações baseadas em linhas (agregação de primeira ordem) são
desenhadas para mostrar a trajetória de objetos, mapas de vias e estradas em
uma região, ou fluxos do tráfego em uma rede de transporte. As trajetórias e
fluxos são representados por linhas ou curvas e são escaladas ou coloridas de
acordo com suas propriedades (i.e. densidade, direção, velocidade) e mostram de forma
explicita a conexão entre diferentes localidades espaciais. \citet{Klein2014} apresenta
um sistema iterativo para análise do tráfego aéreo na França. As trajetórias são
representada por linhas que conectam os pontos de partida e chegada de cada voo,
como pode ser visto na Figura \ref{fig:air-traffic}.

\begin{figure}[!htb]
  \centering
  \includegraphics[width=1\textwidth]{../figuras/air-traffic.png}
  \caption[Visualização do tráfego aéreo na França]{Visualização do tráfego aéreo na França. Fonte: \citet{Klein2014}.}
  \label{fig:air-traffic}
\end{figure}

A representação espacial das linhas pode ainda sofrer transformações
geométricas e topológicas que geram novas abstrações. Por exemplo,
\citet{Tarik2009} mostram uma proposta que transforma as trajetórias de um
leiaute espacial (Figura \ref{fig:viz-espacial}) para um leiaute abstrato
(Figura \ref{fig:viz-abstrata}) para visualizar a movimentação de pessoas
durante a fuga de uma explosão em um escritório. A abordagem abstrata é usada
para mostrar mais efetivamente os padrões de movimentação de pessoas que se
movem ao mesmo tempo para as mesmas áreas. Na Figura \ref{fig:viz-abstrata} é
possível notar a movimentação de indivíduos antes da explosão (linhas azuis), o
que sugere possíveis suspeitos ou testemunhas do evento.  As informações
temporais são difíceis de representar no leiaute espacial, assim o leiaute
abstrato complementa a visualização dos dados mostrando os instantes de tempo
no eixo X e as informações espaciais no eixo Y.

% Existem trabalhos que mostram
% ainda visualizações em 3 dimensões onde as informações de tempo e espaço são
% mostradas ao mesmo tempo na imagem de um cubo, no entanto esse tipo de
% recurso acaba elevando bastante a complexidade de design e desenvolvimento
% da visualização, e por isso é menos utilizado.

\begin{figure}[ht!]
  \centering
  \begin{subfigure}[t]{0.45\textwidth}
    \centering
    \includegraphics[width=65mm]{../figuras/proximidade-espacial.png}
    \caption{Visualização espacial das trajetórias ao longo do tempo. \label{fig:viz-espacial}}
  \end{subfigure}
  ~
  \begin{subfigure}[t]{0.45\textwidth}
    \centering
    \includegraphics[width=75mm]{../figuras/proximidade-abstrata.png}
    \caption{Visualização abstrata da movimentação ao longo do tempo. \label{fig:viz-abstrata}}
  \end{subfigure}

  \caption[Visualização espacial vs abstrata da movimentação de pessoas]{Simulação de uma evacuação de um escritório depois de uma explosão.
(a) Visualização da movimentação da trajetória das pessoas no espaço. (b)
Visualização abstrata, baseada na proximidade.  Fonte: \citet{Tarik2009}.
\label{fig:tarik}}
\end{figure}

%  Algumas ideias buscam ainda sintetizar as informações espaciais e temporais
%em um cubo espaço-temporal, que utiliza o eixo Z para mostrar o tempo e os
%eixos X e Y para representar o espaço.

No entanto, a medida que o número de linhas cresce, aumenta também os
problemas de oclusão, o que acaba afetando a estética da visualização e a
obtenção de informações sobre os dados \citep{Zhou2013}. Visualizações baseadas
em regiões (agregação de segunda ordem) são uma forma de reduzir
a complexidade de análise de um grande conjunto trajetórias. Os dados 
são então agrupados em um nível de macro regiões, geralmente determinado por divisões administrativas.
\citet{Zeng2013} apresentam um diagrama em círculos para mostrar padrões de movimentação de
pessoas entre as regiões da cidade. O círculo representa a junção ou conexão
entre as diferentes regiões e o fluxo entre elas. A densidade do fluxo é medido
pela espessura, bem como a direção é destacada dentro do círculo, como é
ilustrado na Figura \ref{fig:interchange-circo}. Esse tipo de agrupamento
troca as informações individuais de cada trajetória  (como na representação de linhas e pontos)
em favor de uma visão condensada, tornando possível a exploração de maiores quantidades de dados, porém
sob uma macro visão da informação.

\begin{figure}[!h]
  \centering
  \includegraphics[width=0.97\textwidth]{../figuras/region-based.png}
  \caption[Exemplo de visualização baseada em regiões do sistema de metrô na França]{Exemplo de visualização baseada em regiões: padrões regionais de movimentação no sistema de metrô na França. Fonte: \citet{Zeng2013}.}
  \label{fig:interchange-circo}
\end{figure}

Visualizar conjuntos de trajetórias com dezenas de milhares de fluxos ou mais é um grande desafio.
Além dos problemas de oclusão para representar tantos objetos em uma única imagem, há também o desafio
na visualização de seus múltiplos atributos como velocidade, direção, tipo de veículo,
posição. Uma maneira de contornar os problemas de escalabilidade visual
é através de filtros que determinam a quantidade de itens na visualização.
Porém, como consequência do filtro, perde-se a visão global de todos os itens ao mesmo tempo,
o que pode ser necessário para a identificação das correlações e padrões entre eles. 
Uma outra técnica conhecida como \emph{bundling} busca resolver
tais problemas de escalabilidade visual, trazendo um outro tipo de abstração sobre visualizações
baseadas em linhas, que agrupa trajetórias espacialmente próximas, e opcionalmente,
que possuam atributos similares. 

Técnicas de \emph{bundling} têm sido utilizadas em várias pesquisas sobre dados de movimentação, como
estudos sobre migração de aves, \citep{Anita2017}, identificação de padrões no tráfego aéreo, \citep{Klein2014},
visualização da movimentação de embarcações marítimas \citep{Willems2009}, entre outros. Na seção a seguir
explicamos como o \emph{bundling} pode ser aplicado para simplificar a visualização de grandes conjuntos
de dados de trajetórias, similares aos dados de OD de nossa pesquisa.

%  fez uma análise da movimentação de embarcações marítimas também explorando
% técnicas de \emph{bundling}. Separadamente, \citep{Blascheck2017} mostra como a técnica foi
% empregada para simplificar a visualização de dados de monitoramento
% da visão para inferir padrões de leitura. 

% Técnicas de \emph{bundling} têm sido utilizadas em várias pesquisas sobre dados de movimentação.
% \citep{Anita2017} criou uma visualização com \emph{bundling} para estudar características da migração de aves.
% \citep{Klein2014} desenvolveu uma visualização dinâmica para mostrar variações do tráfego aéreo ao longo do tempo e
% \citep{Willems2009} fez uma análise da movimentação de embarcações marítimas também explorando
% técnicas de \emph{bundling}. Separadamente, \citep{Blascheck2017} mostra como a técnica foi
% empregada para simplificar a visualização de dados de monitoramento
% da visão para inferir padrões de leitura. 

%  Lhuillier et al. [8] compiled a survey on the state of
% the art on bundling techniques. This survey distinguishes between bundling graph
% data (straight-line drawings of graph layouts) and trail data (such as our OD data),
% highlighting the best methods for both cases.ss

\section{\emph{Bundling}}
\label{sec:bundling}
O \emph{bundling} ajuda a simplificar o desenho da visualização
através da agregação espacial das trajetórias em conjuntos chamados
\emph{bundles}. Os \emph{bundles} são definidos como um grupo de trajetórias similares, compatíveis o suficiente para
serem representadas por um corpo único e compacto \citep{Lhuillier2017}. A
compatibilidade é calculada a partir de uma função de similaridade usada para
determinar quais objetos devem fazem parte do mesmo agrupamento.
Imaginando então algumas viagens no trânsito, é possível agrupá-las pela sua
região de origem, destino, distância percorrida, direção ou até mesmo o meio de
transporte utilizado. Dessa forma, a informação agrupada em \emph{bundles} é ligeiramente
inferior ao número de trajetórias a serem desenhadas, ficando mais simples
compreender e visualizar a estrutura global, padrões e tendências entre grupos
de trajetórias que ligam áreas fortemente relacionadas \citep{Zhou2013}. A
Figura \ref{fig:bundling-hierarquico} mostra o uso de uma técnica de
\emph{bundling} para visualização da hierarquia de módulos e arquivos de um
software onde as linhas mostram as relações de dependência entre os itens. A ligação
acontece quando um módulo acessa variáveis e/ou funções definidos em outras partes
do software.

\begin{figure}[!htb]
  \centering
  \includegraphics[width=.45\textwidth]{../figuras/hbundling.png}
  \caption[Bundling hierárquico na visualização de dados de software]{Bundling hierárquico na visualização de dados de software. As cores mapeiam
o sentido do acesso, verde é a origem, vermelho é o alvo. Fonte: \citep{Holten2006}.}
  \label{fig:bundling-hierarquico}
\end{figure}


% Lhuillier et al. [8] compiled a survey on the state of
% the art on bundling techniques. This survey distinguishes between bundling graph
% data (straight-line drawings of graph layouts) and trail data (such as our OD data),
% highlighting the best methods for both cases.

% \citepLhuillier2017

O cálculo de um \emph{bundle} não possui uma definição precisa, e por isso há
uma grande variedade de algoritmos e abordagens para modelar problemas de
agrupamento de arestas com \emph{bundling}, que podem variar significativamente
de um para outro em questões de complexidade e aplicação dos algoritmos
\citep{Zhou2013}.  \emph{Force-directed edge bundling} (FDEB), apresentado em
\citet{Holten2009}, cria \emph{bundles} através da atração entre pontos de
controle colocados ao longo das arestas e é considerado um modelo baseado em
custo, já que tenta minimizar o valor de uma função de custo que representa a
força de atração entre as arestas.  \emph{Hierarchical Edge Bundling} (HEB)
agrupa as arestas baseadas na estrutura hierárquica do grafo para o cálculo dos
\emph{bundles} e por isso é considerado um modelo geométrico \citep{Holten2006}. Outra
classificação são os modelos baseados em imagem, como \emph{Skeleton-based edge
bundling} (SBEB) e \emph{Kernel Density Estimation-based Edge Bundling}
(KDEEB), que surgiram mais recentemente. Eles utilizam algoritmos de
clusterização para extrair a estrutura geral do grafo para então calcularem os
\emph{bundles}. O algoritmo SBED utiliza o esqueleto do grafo como guias para
criar \emph{bundles} com muitas ramificações \citep{Ersoy2011}, enquanto o
KDEEB utiliza um mapa de densidade do grafo para extrair sua estrutura no
espaço da imagem \citep{Hurter2012}. A Figura \ref{fig:subfigures} dá um
exemplo da variedade de possíveis saídas geradas por diferentes algoritmos de
\emph{bundling} (FDEB, SDEB e KDEEB) aplicados nos mesmos dados.  Nesse exemplo
são usados dados de uma semana do tráfego aéreo sobre os EUA.

\begin{figure}[h!]
  \centering
  \begin{subfigure}{0.44\textwidth}
    \centering
    \includegraphics[width=1\textwidth]{../figuras/FDEB.png}
    \caption{FDEB}
    \label{fig:FDEB}
  \end{subfigure}

  \begin{subfigure}{0.44\textwidth}
    \centering
    \includegraphics[width=1\textwidth]{../figuras/SBEB.png}
    \caption{SBEB}
    \label{fig:SBEB}
  \end{subfigure}

  \begin{subfigure}{0.44\textwidth}
    \centering
    \includegraphics[width=1\textwidth]{../figuras/KDEEB.png}
    \caption{KDEEB}
    \label{fig:KDEEB}
  \end{subfigure}
  \caption[Diferentes algoritmos de \emph{bundling} aplicadas nos mesmos dados]{Diferentes algoritmos de \emph{bundling} aplicadas em dados do tráfego aéreo dos EUA - 235 nós, 2099 arestas. Fonte: \citep{Klein2014}.}
  \label{fig:subfigures}
\end{figure}

  A escolha do modelo geralmente parte da observação e experiências de
aplicação em problemas semelhantes, já que conhecer e testar diferentes
abordagens e suas variações não é uma tarefa simples. Por isso,
\citet{Lhuillier2017} recentemente propôs uma nova taxonomia para os modelos e
algoritmos de \emph{bundling} dividindo-os com base no tipo de dados em que se deseja
aplicar a técnica. Eles justificaram que desta forma pesquisadores e usuários
da técnica pudessem focar no seu contexto de aplicação e não em detalhes
específicos de implementação algorítmica. Seu estudo dividiu os modelos
existentes inicialmente em dois grupos, conforme o tipo de dados, grafos ou
trajetórias. Então, mais detalhes sobre a direção (direcionado ou não),
dimensão (2D ou 3D) e dependência do tempo (dinâmico ou estático) refinam a
divisão dentro da sua taxonomia.


\begin{figure}[!htb]
  \centering
  \includegraphics[width=\textwidth]{../figuras/estado-da-arte.pdf}
  \caption[Taxonomia dos métodos de \emph{bundling} baseado no tipo de dado]{Taxonomia dos métodos de \emph{bundling} baseado no tipo de dado. Fonte: \citet{Lhuillier2017}}
  \label{table:bundling-methods}
\end{figure}


Podemos então seguir a taxonomia para selecionar uma método de
\emph{bundling} que seja mais adequado para a análise de fluxos no trânsito da
cidade. \emph{Attribute-Driven Edge Bundling} (ADEB) é um algoritmo que segue um
modelo baseado em imagem e é apresentado dentro desta taxonomia como um das
opções para a análise de dados de trajetórias direcionadas. Discutimos na
próxima seção as características desse modelo e como podemos nos beneficiar
dele em nosso estudo.

\subsection{Modelo Baseado em Imagem}
\label{sec:modelo-imagem}

  Uma das etapas do \emph{bundling} é a identificação das trajetórias
similares, que requerem que cada uma delas seja comparada a todos os outras
para descobrir quais são as mais próximas, afim de agrupá-las em
\emph{bundles}.  Esse cálculo, no entanto, é bastante custoso e pode demorar
minutos em cenários com milhares de trajetórias.

\emph{Kernel Density Estimation Edge Bundling} (KDEEB) é um algoritmo proposto
por \citet{Hurter2012}. Eles observaram que ao aplicar uma operação de \emph{bundling} $B$ qualquer sobre um
grafo $G$, o resultado são áreas de maior densidade (dentro dos \emph{bundles})
e áreas de menor densidade fora deles, em comparação ao grafo original. Assim,
essa operação \emph{bundling} $B$ pode ser moldada em função de uma operação
sobre a densidade dos pontos do grafo, similar ao que faz o algoritmo de
clusterização \emph{Mean Shift} \citep{Comaniciu2012}. A partir disso, a heurística
desse algoritmo de \emph{bundling} consiste em computar repetidamente o
gradiente dos pontos em relação à uma função de densidade, e então movendo-os
para regiões mais densas apontadas pelo gradiente. O maior benefício do método
é sua implementação paralela com o uso do poder computacional de placas
gráficas (GPUs), o que levou a ganhos de desempenho de uma ordem de magnitude
em relação a métodos anteriores. KDEEB representou uma abertura para uma área
chamada \emph{image-based bundling}, ou métodos baseados em imagem, onde $B$ é
implementado via operações de processamento de imagem, diferentemente de
métodos puramente geométricos existentes até então. O algoritmo consiste nos
seguintes passos:

\begin{enumerate}
  \item Converte o grafo $G$ em um mapa de densidade usando uma função de
densidade. A função utilizada é um estimador baseado em núcleos, geralmente um
estimador Gaussiano ou Epanechnikov.

  \item Computar o gradiente da função de densidade para cada ponto/nó do
grafo. O cálculo do gradiente indica a direção onde há maior quantidade
de nós/arestas aglomeradas no grafo.

  \item Mover os nós na direção do gradiente (áreas mais densas). Essa etapa é
suavizada movendo-se os nós pela norma do gradiente, ou seja, em apenas uma unidade
a cada iteração. Isso evita que as arestas dêem grandes saltos.

  \item Corrigir distorções causadas pela movimentação dos nós com filtro
Laplaciano (opcional).  O passo anterior pode causar pequenas distorções, como
por exemplo, a sobreposição de nós. Esta é uma forma de se corrigir essas
distorções sem que se perca a estrutura geral do grafo.

  Repetir a partir do passo 1 até a convergência do algoritmo, o que leva cerca de 8..10
iterações.
\end{enumerate}

%\citet{Hurter2012} apresentam ainda uma maneira para limitar os \emph{bundles}
%com relação a obstáculos presentes no caminho das arestas. Um obstáculo
%qualquer, representado por seus limites espaciais pode ser contornado
%alterando-se a função de densidade, que sofre uma degradação dentro da área
%geométrica do obstáculo. Com isso, o gradiente age como uma força de repulsão
%das arestas que cruzam o obstáculo e faz com que as arestas sejam movidas
%na direção pra fora de suas bordas.

  {\emph{Attribute-Driven Edge Bundling} (ADEB) é uma extensão do algoritmo
KDEEB, proposta por \citet{ZegarraFlores2016}, e que permite a separação das
arestas conforme sua direção. Essa informação é obtida percorrendo cada aresta
a partir do ponto de origem até o destino e calculando para cada uma delas um
vetor unitário que aponta a sua direção, que é mapeado para o
ângulo desse mesmo vetor. Então, cria-se um mapa de densidade direcionado, onde são
consideradas apenas as arestas que vão na mesma direção (mesmo ângulo). Esse
algoritmo herda todos os benefícios de desempenho e robustez apresentados pelo
KDEEB, o que o faz um método adequado para a análise do tráfego de veículos no
trânsito da cidade.

\subsection{\emph{Bundling} em Dados Dinâmicos}

  Um grafo dinâmico é aquele no qual seus nós e arestas são dependentes do
tempo, ou seja, a cada momento \emph{t}, existe um grafo G(t) diferente a ser
explorado. Nesse contexto, dois diferentes tipos de grafos dinâmicos podem ser
considerados, grafos sequenciais e grafos de fluxo \citep{Hurter2013}. Um
grafo sequencial consiste em um conjunto de grafos $G^i = (V^i, E^i)$, onde
cada grafo $G^i$ é como um retrato no momento $i$ de um sistema que é
dependente do tempo. Grafos de fluxo, por outro lado, são definidos como um
conjunto de vértices $V$ e arestas $E$, onde as arestas são definidas pelo seu
tempo de vida e nós. Grafos de fluxo não possuem uma sequência pré-definida e
são tipicamente obtidos de fontes de dados online.

  O uso de \emph{bundling} nesse cenário é similar a outros métodos de
visualização de grafos dinâmicos. Para grafos sequenciais o \emph{bundling} é
aplicado a cada grafo $G^i$ de maneira estática. Já em grafos de fluxo, os
dados são divididos em janelas de tempo e o algoritmo é aplicado em cada
intervalo $\Delta t$, que podem ser colocados lado a lado para comparação em
uma técnica conhecida como \emph{small multiples}. Uma outra maneira de se
visualizar a dinâmica é com o uso de animações, como mostram
\citet{Hurter2014}. A vantagem desse método sobre o anterior é que ele se
adequa melhor à análise de longas séries temporais.

-----------------------------

  Os conceitos apresentados servem de base para a nossa proposta de
visualização dos dados do tráfego de veículos no trânsito. O processo de
formulação da proposta inclui a definição do tipo de visualização que iremos
utilizar, obter domínio sobre os dados do trânsito e suas propriedades, para
posteriormente chegarmos a uma análise dos fluxos de origem e destino em
diferentes níveis de detalhe a partir de um modelo de \emph{bundling}, além de
investigar formas de destacar essas propriedades dentro da visualização. No
próximo capítulo apresentamos uma revisão da literatura sobre alguns trabalhos
que utilizam \emph{bundling} e outras técnicas para análise de dados de
tráfego.
