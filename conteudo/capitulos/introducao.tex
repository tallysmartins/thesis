%% ------------------------------------------------------------------------- %%
\chapter{Introdução}
\label{cap:introducao}

A mobilidade urbana é um conceito que está associado ao comportamento de fluxos
de deslocamento de pessoas e objetos dentro das cidades. Com o rápido
desenvolvimento de sistemas de transporte e crescimento das cidades, esses
fluxos têm se tornado cada vez mais complexos e volumosos. É estimado que cerca
de 40\% da população passa pelo menos uma hora no trânsito todos os dias
\citep{Zhang2011}, o que afeta diretamente sua qualidade de vida e também gera
impactos econômicos negativos.

Cada vez mais a coleta e análise de dados de mobilidade tem sido feita para
apoiar o planejamento de sistemas de transporte mais eficientes. Esses dados
podem ser gerados de diferentes fontes, como GPS, câmeras de tráfego, e
pesquisas censitárias. Dentro desse contexto, nós focamos nosso estudo em
dados de mobilidade da região metropolitana de S\~ao Paulo.

A cada dez anos, desde 1967, a Companhia Metropolitano de São Paulo (Metrô), que gerencia o
sistema de metrô da cidade, conduz um estudo de mobilidade chamado
\emph{Pesquisa Origem-Destino} (OD).  Nesse estudo, são coletados dados das
atividades diárias dos cidadãos em um dia típico da semana, o que resulta num
panorama do comportamento de mobilidade da população na região. Esses dados são
utilizados pela Companhia do Metrô para entender padrões de deslocamento na
região. Esses padrões guiam novas políticas públicas, como criação de novas
linhas de transporte de metrô e ônibus, e ajudam no monitoramento da evolução
do transporte público e privado.

A pesquisa OD contem dados de locomoção diários, com informações de origem e
destino de viagens feitas por uma amostra da população, que é posteriormente
extrapolada para toda a cidade. No último ano em que a pesquisa foi feita, em
2017, foram contabilizados aproximadamente 42 milhões de viagens em um dia
típico. A pesquisa também coleta vários dados socioeconômicos como idade,
gênero, renda familiar e tipos de transporte utilizados. 

Formas de visualização desses dados facilitam essa tarefa, uma vez que uma
representação visual pode comunicar padrões de deslocamento de forma mais
intuitiva \citep{Liu2013}. 




%Formas de visualização desses dados facilitam essa tarefa, uma vez que uma representação
%visual pode comunicar seus padrões geográficos e direcionais de forma mais intuitiva
%\citep{Liu2013}.
%
%% problema
%  De maneira abstrata, fluxos de tráfego podem ser interpretados como um grafo
%direcionado, onde os vértices (nós) representam localidades espaciais e as
%arestas representam o fluxo e a ligação entre as diferentes localidades
%\citep{Anita2017}. A visualização desses fluxos é uma tarefa complexa que
%apresenta desafios de escalabilidade computacional e visual \citep{Klein2014}.
%Quando uma representação de linhas é utilizada, mesmo conjuntos de dados com
%poucos fluxos começam a apresentar uma certa oclusão, com vários cruzamentos e
%sobreposição entre os elementos da visualização, o que torna difícil sua
%interpretação. A oclusão acaba por limitar a quantidade de elementos que podem
%ser vistos ao mesmo tempo e prejudica a identificação de possíveis padrões e
%relações presentes nos dados, como por exemplo o movimento de uma grande massa
%de veículos de uma região A para uma região B durante o mesmo período. A Figura
%\ref{fig:exemplo-oclusao} ilustra as trajetórias de aviões durante uma semana
%sobre os Estados Unidos. O grande número de voos gera uma oclusão que torna
%impossível identificar as conexões entre as regiões e aeroportos.
%
%\begin{figure}[!htb]
%  \centering
%  \includegraphics[width=100mm]{../figuras/cluttered-map.png}
%  \caption[Trajetórias de voos nos Estados Unidos]{Trajetórias de voos nos Estados Unidos no período de uma semana. Fonte \citet{Zhou2013}.}
%  \label{fig:exemplo-oclusao}
%\end{figure}
%
%  Algumas aplicações existentes no domínio do trânsito, como Google
%Maps\footnote{\rurl{maps.google.com}} e
%Waze\footnote{\rurl{www.waze.com/pt-BR}}, que sugerem para motoristas qual
%caminho tomar até um determinado destino, e
%OlhoVivo\footnote{\rurl{olhovivo.sptrans.com.br}}, que  mostra as rotas feitas
%por ônibus na cidade de São Paulo e suas posições instantâneas, se empenham em
%mostrar informações pontuais do tráfego e não na sua estrutura geral. Não é
%possível, por exemplo, saber de onde vem os veículos que formam o fluxo em uma
%determinada via ou região. O Google Maps, por exemplo, sugere um trajeto entre
%uma origem e um destino inseridos pelo usuário e traça o percurso em um mapa.
%Sua visualização destaca também as áreas com maior densidade de fluxo com
%diferentes cores, sendo laranja trechos com fluxo moderado e vermelho sendo
%trechos de maior fluxo, indicando que há um congestionamento. A Figura
%\ref{fig:gmaps} ilustra uma visualização de uma rota sugerida pelo Google Maps.
%
%\begin{figure}[!htb]
%  \centering
%  \includegraphics[width=\textwidth]{../figuras/maps.pdf}
%  \caption{Trajeto entre dois pontos visto com o Google Maps \label{fig:gmaps}}
%\end{figure}
%
%  Para lidar com os problemas da oclusão, algumas técnicas têm sido
%desenvolvidas por comunidades de visualização e da cartografia para a o estudo
%de grandes conjuntos de dados de tráfego, como por exemplo, clusterização de
%vértices \citep{Schaeffer2007, Andrienko2011}, matrizes \citep{Elmqvist2008}
%e grides \citep{JoWood2010}. Essas técnicas, no entanto, acabam usando
%abstrações diferentes da representação baseada em linhas. Um outro tipo de
%abstração pode ser obtido com uma técnica chamada \emph{bundling}, que tem sido
%bastante utilizada na visualização massiva de dados do tráfego que utilizam a
%representação de fluxos por linhas \citep{Zhou2013}. Algoritmos de
%\emph{bundling} agrupam arestas similares e próximas, permitindo uma
%representação única e compacta que reduz o número de arestas e a oclusão no
%desenho. A Figura \ref{fig:exemplo-bund} ilustra a aplicação da técnica de
%\emph{bundling}.  Quanto mais à esquerda, mais oclusão, e à direita, o
%resultado final.
%
%\begin{figure}[!htb]
%  \centering
%  \includegraphics[width=1\textwidth]{../figuras/bundle-ex.png}
%\caption[Exemplo de aplicação do \emph{bundling}]{Exemplo de aplicação do \emph{bundling}. Fonte: \citet{Hurter2012}.}
%  \label{fig:exemplo-bund}
%\end{figure}
%
%  Uma variedade de métodos de \emph{bundling} têm sido apresentados por
%pesquisadores, como métodos geométricos, hierárquicos e baseados em imagem
%\citep{Lhuillier2017}. Recentemente, novos métodos viabilizaram a visualização
%de grandes conjuntos de dados e em diferentes níveis de detalhes, como em
%\citet{Klein2014}, que utilizam \emph{bundling} na visualização de milhares de
%dados do tráfego aéreo. Eles mostram como a técnica destaca rotas comuns nos
%voos e o grau de conexão entre diferentes aeroportos. O método utilizado
%permite ainda uma abordagem multinível de acordo com parâmetros que
%possibilitam o controle do \emph{bundling} em função da densidade de pontos no
%espaço. Isso significa que é possível visualizar padrões globais entre grandes
%áreas no mapa, que contemplem muitos pontos, e também padrões locais sobre
%áreas menores com poucos pontos. Os autores também apresentam uma visualização
%dinâmica para mostrar as mudanças no tráfego ao longo do tempo, agrupando os
%voos em pequenos intervalos de tempo $\Delta t$ e aplicando o \emph{bundling}
%nos dados de cada intervalo.
%
%  Neste trabalho, focamos na visualização do fluxo de veículos no trânsito e as
%relações de origem-destino (OD) que eles representam. Apresentamos uma
%visualização dos dados utilizando \emph{bundling} para explorar os fluxos de
%deslocamentos no trânsito da cidade e como eles impactam em vias com grandes
%congestionamentos. Fazemos um novo estudo sobre visualização de dados do
%tráfego projetando uma visualização para mostrar os fluxos em diferentes níveis
%de detalhe. Utilizamos dados do tráfego de ônibus na cidade de São Paulo e
%também dados gerados pelo InterSCSimulator, um simulador de cidades
%inteligentes apresentado por \citet{mabs2017}.
%
%\section{Objetivos e Contribuições}
%  O objetivo desta pesquisa é explorar como o \emph{bundling} ajuda na
%observação de padrões nos fluxos de deslocamento no tráfego de veículos de uma
%grande cidade. Para isso, fazemos uma análise sobre as propriedades temporais e
%espaciais dos dados do tráfego, sendo a direção e a densidade os principais
%atributos que utilizaremos para responder às seguintes questões de pesquisa:
%
%\begin{itemize}
%  \item[\textbf{Q1)}] \textbf{Como o \emph{bundling} pode ser usado para
%identificar os fluxos de origem e destino no trânsito em diferentes escalas?} -
%Queremos investigar o uso do \emph{bundling} para visualizar os fluxos no
%tráfego em várias escalas, até no nível das ruas. Com isso exploraremos
%a visualização das origens e destinos dos fluxos em uma via, para saber, por exemplo
%quais pontos de origem e destino são maiores responsáveis por um congestionamento.
%
%  \item[\textbf{Q2)}] \textbf{É possível utilizar o \emph{bundling}  para
%identificar padrões de fluxos de origem e destino no trânsito?} - O
%\emph{bundling} acaba por dar uma visualização da estrutura geral do tráfego,
%que por sua vez, não é a mesma ao longo de todo o dia. Mais veículos trafegam
%nas ruas em horários de pico, os itinerários do transporte público mudam,
%acidentes fecham ruas, e vários outros fatores afetam a dinâmica do trânsito.
%Queremos testar essa estrutura geral em diferentes momentos do dia para saber
%como o \emph{bundling} destaca essas mudanças ou padrões na visualização do
%tráfego.
%
%  \item[\textbf{Q3)}] \textbf{O \emph{bundling} é eficiente para gerar uma
%visualização de uma grande quantidade de dados do trânsito?} - O número de
%veículos circulando nas ruas de uma grande cidade é variavelmente superior a
%outros contextos, como o tráfego aéreo. São Paulo, por exemplo, possui mais de
%8 milhões de veículos, segundo o Departamento Estadual do
%Trânsito\footnote{\rurl{www.detran.sp.gov.br}} (Detran). Queremos entender os
%desafios de escalabilidade computacional do \emph{bundling} na visualização de uma
%grande quantidade de dados do trânsito medindo o tempo gasto para se gerar a visualização.
%\end{itemize}
%
% Para isso, apresentamos o InterSCityPlotter, uma ferramenta livre e de código
%aberto, para a análise de dados do tráfego de veículos que utiliza
%\emph{bundling} para visualização dos dados do trânsito.  Exploramos o
%\emph{bundling} em três aspectos: (1) abordagem multinível para visualizar
%padrões globais (i.e. na escala da cidade), e locais (i.e. bairros, quadras e
%ruas); (2) abordagem dinâmica para visualização de mudanças no padrão de fluxos
%ao longo do tempo; (3) sua utilização com uma grande quantidade de dados do
%trânsito em escala real de uma grande cidade, com mais de 4 milhões de
%veículos.  Apresentamos também um conjunto de técnicas para mapear os atributos
%de direção e densidade dos dados dentro de uma visualização sobre um mapa.
%
%  Avaliaremos os resultados da visualização de maneira qualitativa e
%quantitativa. Do ponto de vista qualitativo avaliaremos o que foi alcançado com
%a visualização em relação às questões \textbf{Q1} e \textbf{Q2}
%levantadas, fazendo considerações sobre os resultados obtidos e suas
%limitações. Além disso, compararemos nossa proposta em relação a outros
%trabalhos sobre visualização de dados de movimentação que utilizam
%\emph{bundling}. Do ponto de vista quantitativo será avaliado o desempenho da
%solução dado o tempo de processamento para gerar a visualização dos conjuntos
%de dados utilizados, relacionado à questão \textbf{Q3}.
%
%  Esperamos que a solução a ser criada ajude na identificação das relações
%entre as regiões da cidade que recebem maior fluxo, como esse fluxo se
%distribui ao longo do tempo e como ele impacta em vias com grandes
%congestionamentos. Desta forma, esta pesquisa potencialmente oferece as
%seguintes contribuições científicas (CC) e contribuições técnicas (CT):
%
%\begin{itemize}
%  \item \textbf{CC} - avaliação do \emph{bundling} em multiníveis para análise de
%padrões globais e locais do tráfego
%
%  \item \textbf{CC} - avaliação da escalabilidade do \emph{bundling} na visualização de uma grande
%quantidade de dados do tráfego de veículos
%
%  \item \textbf{CC} - avaliação do \emph{bundling} dinâmico para análise de mudanças ao longo
%e a ocorrência de eventos atípicos
%
%  \item \textbf{CT} - construção de uma ferramenta para visualização dos dados do tráfego
%que será integrada ao InterSCSimulator
%\end{itemize}
%
%Esta pesquisa está sendo desenvolvida como parte do projeto InterSCity
%\footnote{\rurl{interscity.org}} - INCT of the Future Internet for Smart
%Cities. O objetivo do projeto é desenvolver pesquisas multidisciplinares em
%infraestruturas de software para cidades inteligentes, produzindo contribuições
%técnicas e científicas para a comunidade \citep{Daniel2016}. O simulador InterSCSimulator foi
%desenvolvido no contexto do projeto e nós nos apoiamos em seus resultados como
%forma de impulsionar esta pesquisa no campo da visualização de dados do
%tráfego. Desta forma, o desenvolvimento desta pesquisa de mestrado seguirá as
%diretrizes do projeto InterSCity para produzir resultados que sejam
%reprodutíveis, bem como soluções de software livre que possam ser mantidas
%adiante pela comunidade do projeto.
