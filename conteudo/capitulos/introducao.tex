%% ------------------------------------------------------------------------- %%
\chapter{Introdução}
\label{cap:introducao}

O rápido  desenvolvimento de sistemas de transporte e crescimento das cidades
tem sido acompanhado de fluxos de deslocamento cada vez mais complexos e
volumosos, aumentando também os desafios para as questões de mobilidade urbana.
\citet{Zhang2011} mostram em seu estudo uma estimativa de que cerca de 40\% da
população mundial passa pelo menos uma hora no trânsito todos os dias.
Os impactos negativos de uma infraestrutura de transporte
ineficiente se traduzem em perdas econômicas, como mostra um outro estudo,
elaborado por \citet{ricardo:18}, que somente no Brasil, os gargalos em mobilidade geram
perdas anuais de cerca de R\$ $7,33$ bilhões de reais. Tais dados mostram como há
bastante espaço para se buscar melhorias para os sistemas de transporte, de
maneira a aumentar a qualidade de vida dos cidadãos e reduzir custos.

Para isso, a tecnologia tem sido amplamente empregada como um agente
facilitador de coleta e disponibilização de várias fontes de informação que
permitem cidadãos, pesquisadores, empresas e agentes públicos a realizarem
estudos para entender problemas de mobilidade urbana e buscarem soluções.
Várias tipos de dados podem ser coletados de diferentes fontes, como sensores
de GPS, câmeras de tráfego, e mesmo coletados manualmente. Na administração
pública, essas fontes de informação podem ser utilizadas para prover serviços
melhores para a população, melhorar a gestão da infraestrutura urbana e dar
suporte para a tomada de decisão baseada em evidências. Dentro desse
contexto, nós focamos nosso estudo em dados de mobilidade urbana da Região
Metropolitana de S\~ao Paulo (RMSP).

A cada dez anos, desde 1967, a Companhia Metropolitano de São Paulo (Metrô),
que gerencia o sistema de metrô da cidade, conduz um estudo censitário de
mobilidade chamado \emph{Pesquisa Origem-Destino} (OD). A pesquisa OD é feita
através de entrevistas aos cidadãos e coleta dados sobre suas vidas e
atividades de deslocamento em um dia típico de trabalho. O resultado é um amplo
panorama social e de mobilidade da população na RMSP. A última pesquisa OD
(2017) mostra que ocorrem cerca de 42 milhões de deslocamentos nas 24 horas de
um dia comum de trabalho. Além das informações de origem (O) e destino (D), a
pesquisa coleta informações socioeconômicas dos cidadãos e também outros dados
sobre os deslocamentos, como o modo de transporte utilizado, o motivo da
viagem, idade, gênero, renda familiar. Com isso, a pesquisa OD se traduz num
grande conjunto de dados com múltiplos atributos.

Embora a pesquisa OD mencionada seja bem abrangente e precisa, 
os gestores precisam de ferramentas adequadas para analisar essa grande quantidade de
dados multivariados. O uso de planilhas e gráficos com informações agregadas
ajudam a compreendê-los, mostrando, por exemplo, o número de usuários 
do sistema de transporte público ao longo dos anos, ou o número de homens \emph{vs.} mulheres
que saem diariamente a trabalho. Técnicas de visualização, como mapas de densidade,
podem ajudar a responder questões geo-espaciais, como encontrar regiões que concentram
os maiores fluxos de mobilidade durante o dia ou quais os pares de origem e destino
mais comuns. No entanto, considerando a grande quantidade de dados (e atributos) gerados
nas cidades, traduzir dados geo-localizados em imagens significativas não é uma tarefa simples.
Nesse contexto, buscamos responder a seguinte questão de pesquisa (QP):

\begin{itemize}
  \item \textbf{Como oferecer uma visualização de grandes quantidades de
dados de mobilidade de uma grande região metropolitana?}
\end{itemize}

De maneira geral buscamos em nosso estudo uma proposta de solução que: também
forneça informações visuais claras para gestores urbanos, pesquisadores e o
grande público; permita a visualização clara de milhões de fluxos de mobilidade
em uma única imagem; seja ajustável, permitindo facilmente a visualização de
diferentes parâmetros.

Um estudo recente sobre métodos de visualização de fluxos de mobilidade indica
um uso comum de visualizações baseadas em linhas para estudar a estrutura desses
dados, o que geralmente inclui descoberta de padrões e agrupamentos
\citep{Chen2015}. Cores e texturas podem ser utilizados para dar informações
adicionais sobre os objetos estudados, como sua velocidade, direção, meio de
transporte utilizado, etc. Porém, desenhar diretamente linhas das trajetórias em
um mapa torna-se pouco efetivo em cenários com uma grande quantidade de dados.
Uma visualização das 42 milhões de trajetórias, como os registrados pela última
pesquisa OD, resulta em uma imagem completamente ofuscada pela sobreposição e
cruzamento dos fluxos, como mostramos mais adiante na
Figura~\ref{fig:cluttered-graph}. Para isso, técnicas de \emph{bundling}
aprimoram as visualizações baseadas em linha e podem representar grandes volumes
de dados de trajetórias.

Neste trabalho nós propomos o uso de um conjunto de técnicas de visualização
baseadas em \emph{bundling} para explorar as características de uma grande
quantidade de dados de mobilidade urbana. Para isso, nós adaptamos uma técnica
que revelou resultados interessantes na visualização de trajetórias em outros
cenários, como o do tráfego aéreo, para o nosso contexto. Nosso estudo permitiu
a visualização dos dados em diferentes níveis de granularidade espaço temporal,
ajudando a encontrar padrões de mobilidade a respeito do sistema de transporte
de S\~ao Paulo e da infraestrutura da cidade. Em contraste a outros trabalhos
que utilizam técnicas similares, nossa pesquisa ainda se destaca pela grande
quantidade de dados e atributos analisados. Para isso, apresentamos também uma
metodologia para reduzir o tamanho do conjunto de dados com impacto mínimo na
sua significância estatística.

Desenvolvemos esta pesquisa no contexto do consórcio
InterSCity\footnote{\url{http://interscity.org}} - INCT da Internet do Futuro
para Cidades Inteligentes. O objetivo do projeto InterSCity é desenvolver
pesquisas multidisciplinares em infraestrutura de software para cidades
inteligentes, produzindo contribuições científicas e técnicas para a comunidade
\citep{Daniel2016}. O desenvolvimento desta dissertação de mestrado seguiu as
diretrizes dos projetos do InterSCity com o objetivo de produzir resultados
científicos reprodutíveis e de alto impacto, bem como software 
que pode ser mantido posteriormente pela comunidade do projeto.