\chapter{Considerações Finais}
\label{cap:plano-de-trabalho}

\section{Resultados Alcançados}
  
\section{Esforço Técnico}

\section{Trabalhos Futuros}


%   Bibliotecas abertas com implementações de algoritmos de \emph{bundling} ainda
% não são algo bem estabelecido, diferente de algoritmos de clusterização, que
% possuem vários pacotes disponíveis (e.g. Sklearn, Theano, Keras, TensorFlow). 


%  Um outro aspecto técnico importante a ser considero, é que os trabalhos
%recentes sobre as técnicas de \emph{bundling} apresentam abordagens com
%técnicas para paralelismo e uso de recursos gráficos em placas de vídeo
%dedicadas (GPUs), como é o caso de \citet{Hurter2012} e \citet{ADEB}. Dada a
%nossa análise prévia dos trabalhos, consideramos isso se encaixa como uma
%otimização da nossa proposta, e que por questões de tempo e complexidade, não
%farão parte de nosso escopo.  Uma discussão sobre acelerações de vídeo é
%apresentada em \citet{}. O uso de tecnologias como Cuda, OpenCL, requerem
%hardware específico e conhecimento específico sobre essas tecnologias. Eles
%propõe o uso de uma solução mais genérica propõe uma solução mais genérica que
%usa tecnologia OpenGL, amplamente disponível em diversos dispositivos
%computacionais, e utilizada também na biblioteca Geoplotlib. O custo benefício,
%é que a diferença no cálculo do \emph{bundling} para um grafo com 2000 nós é
%relativamente grande. Mas, dadas as limitações de tempo na elaboração deste
%trabalho, este é um tópico que pode ser abordado em trabalhos futuros.




