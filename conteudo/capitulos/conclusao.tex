\chapter{Considerações Finais}
\label{cap:plano-de-trabalho}

Neste trabalho, exploramos o uso do \emph{bundling} para criar
visualizações de vários atributos dos dados de mobilidade urbana da Região
Metropolitana de São Paulo. Nossas análises sobre as características da pesquisa
OD 2017 mostram que o \emph{bundling} pode ser usado para identificar e comparar
diferentes padrões de mobilidade implícitos em diferentes subconjuntos de dados
e subconjuntos de atributos disponíveis. Combinando adequadamente a filtragem
(para reduzir a quantidade de dados e/ou atributos a serem explorados) com o
\emph{bundling} (para simplificar as visualizações criadas e reduzir a oclusão
visual) e com os canais visuais disponíveis (opacidade, cor, direção),
destacamos diferentes padrões no conjunto de dados OD17 que não teriam sido
facilmente obtido por ferramentas clássicas de mineração e visualização de
dados.

Nossos resultados sobre a visualização das trajetórias com \emph{bundling}
destacaram sua estrutura centralizada sobre a área estudada da RMSP. Além disso,
essa estrutura condiz com a infraestrutura metroviária e ferroviária de São
Paulo. Embora não tenha sido uma surpresa, a correlação sugere que nossos
parâmetros foram bem ajustados para a visualização na escala metropolitana.
Nossa metodologia para reduzir a complexidade do conjunto de dados de 42 milhões
de viagens para menos de um milhão, e também nossa adaptação de um
\emph{framework} de uso geral (\emph{CUBu}) para agrupar subconjuntos específicos de
dados e/ou atributos foram os pontos principais que tornaram essa análise
possível. Desta forma conseguimos responder a nossa questão de pesquisa, pois
conseguimos obter um método de visualização de grandes conjuntos de dados de
mobilidade sob diferentes perspectivas e explorar seus múltiplos atributos.

Como trabalho futuro, pretendemos explorar melhorias no uso e usabilidade de
visualizações agregadas com \emph{bundling}. Do ponto de vista da visualização,
melhorias no mapa para mostrar as divisões das regiões no topo das trajetórias,
conforme proposto em \citet{Klein2014}, podem ajudar a identificar melhor as
conexões entre as regiões. Além disso, seria interessante testar uma abordagem
diferente para transmitir a informação de densidade dos \emph{bundles} alterando
a espessura das trajetórias proporcionalmente ao fator de expansão de cada
registro da OD17 em vez de usar cores, de forma semelhante a
\citet{lhuillier-fft:17}. Desta maneira não haveria a necessidade de replicar as
trajetórias como fizemos e, portanto, isso reduziria significativamente o
tamanho do conjunto de dados. Essas são etapas importantes para permitir o uso
de \emph{bundling} para análise em tempo real.

Da perspectiva da aplicação da técnica de \emph{bundling}, existem ainda muitas
outras possibilidades para análise da mobilidade urbana usando dados da própria
pesquisa OD e ainda a possibilidades de agregar outros conjuntos de dados, como
aqueles de empresas privadas de mobilidade, dispositivos IoT e sistemas de
compartilhamento de bicicletas. Há ainda espaço para se fazer uma análise
temporal mais profunda comparando a evolução da mobilidade urbana utilizando
dados das pesquisas OD anteriores. Por último, mas não menos importante, também
seria interessante complementar os resultados de nosso estudo com a avaliação de
usuários reais, como gestores de tráfego, urbanistas e planejadores urbanos,
para verificar a utilidade das visualizações com \emph{bundling}.

% \section{Esforço Técnico}

% As colaborações deste trabalho envolveram também resultados técnic




