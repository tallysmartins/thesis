\chapter{Considerações preliminares}
\label{cap:plano-de-trabalho}

  Considerando os aspectos teóricos da pesquisa, investigamos as condições
necessárias para o projeto de uma visualização com o uso de \emph{bundling} no
estudo dos fluxos de origem e destino no trânsito. Consolidamos um conhecimento
geral da área de visualização de dados do tráfego, trazendo os vários aspectos
da literatura sobre os tipos de visualização, detalhes sobre a aquisição dos
dados e suas propriedades, bem como uma perspectiva das diferentes formas de
agregação e visualização.  Propomos também um modelo de \emph{bundling}, o qual
acreditamos possuir os recursos necessários para a nossa análise, e com isso,
apostamos na viabilidade de nossa pesquisa.  Do ponto de vista técnico, também
avançamos em algumas questões, a começar pela aquisição dos dados. Já
encontram-se de nossa posse os dados simulados, restando ainda fazer a coleta
dos dados reais do tráfego de ônibus. A ferramenta InterSCityPlotter já começou
a ser implementada e possui algumas funcionalidades, como pode ser visto no
repositório no Gitlab. Nosso maior desafio será na implementação do algoritmo
de \emph{bundling} e dos recursos interativos que propomos dentro da
visualização, como zoom, alteração de parâmetros, e outros.  A seguir,
apresentamos o nosso cronograma, que contabiliza um total de 5 meses,
finalizando em Setembro deste ano.

\begin{figure}
  \centering

  \begin{ganttchart}{2019-03}{2019-9}
    \gantttitlecalendar{year,month=shortname} \ganttnewline

    \ganttgroup[progress=35]{Desenvolvimento}{2019-3}{2019-5} \ganttnewline
    \ganttbar[progress=30]{Tratamento dos dados}{2019-3}{2019-3}\ganttnewline
    \ganttbar[progress=0]{Implementação do Bundling}{2019-4}{2019-4} \ganttnewline
    \ganttbar[progress=0]{Implementação dinâmica}{2019-5}{2019-5} \ganttnewline

    \ganttbar[progress=0]{Análise dos Resultados}{2019-6}{2019-6} \ganttnewline

    \ganttgroup[progress=0]{Escrita da Dissertação}{2019-6}{2019-9} \ganttnewline
    \ganttbar[progress=0]{Escrita}{2019-6}{2019-8} \ganttnewline
    \ganttbar[progress=0]{Revisão}{2019-9}{2019-9} \ganttnewline

    \ganttmilestone{Defesa}{2019-9}
  \end{ganttchart}

  \caption{Cronograma.\label{fig:gantt}}
\end{figure}
