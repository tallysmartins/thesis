\chapter{Considerações Finais}
\label{cap:plano-de-trabalho}

  Construímos até aqui os pilares teóricos e técnicos que irão sustentar nossa
caminhada até a conclusão dos objetivos de nosso trabalho, cujo foco é
investigar o uso de \emph{bundling} no estudo de fluxos de origem e destino na
cidade. Para atingirmos nossos
objetivos, trouxemos o conhecimento sobre práticas e algoritmos
considerados o estado da arte na área de análises de dados de tráfego. Do ponto
vista técnico.... Abordamos
a seguir, algumas considerações a respeito do que produzimos até aqui.

\section{Esforço Técnico}

  Bibliotecas abertas com implementações de algoritmos de \emph{bundling} ainda
não são algo bem estabelecido, diferente de algoritmos de clusterização, que
possuem vários pacotes disponíveis (e.g. Sklearn, Theano, Keras, TensorFlow). 


%  Um outro aspecto técnico importante a ser considero, é que os trabalhos
%recentes sobre as técnicas de \emph{bundling} apresentam abordagens com
%técnicas para paralelismo e uso de recursos gráficos em placas de vídeo
%dedicadas (GPUs), como é o caso de \citet{Hurter2012} e \citet{ADEB}. Dada a
%nossa análise prévia dos trabalhos, consideramos isso se encaixa como uma
%otimização da nossa proposta, e que por questões de tempo e complexidade, não
%farão parte de nosso escopo.  Uma discussão sobre acelerações de vídeo é
%apresentada em \citet{}. O uso de tecnologias como Cuda, OpenCL, requerem
%hardware específico e conhecimento específico sobre essas tecnologias. Eles
%propõe o uso de uma solução mais genérica propõe uma solução mais genérica que
%usa tecnologia OpenGL, amplamente disponível em diversos dispositivos
%computacionais, e utilizada também na biblioteca Geoplotlib. O custo benefício,
%é que a diferença no cálculo do \emph{bundling} para um grafo com 2000 nós é
%relativamente grande. Mas, dadas as limitações de tempo na elaboração deste
%trabalho, este é um tópico que pode ser abordado em trabalhos futuros.

\subsection{Validação da Visualização}

 Percebemos que um aspecto delicado em pesquisas de visualização de dados é a
validação da própria visualização. \citet{Telea2018} cita que a definição de
critérios objetivos que possam ser aplicados para comparar diferentes propostas
ainda é uma questão em aberto. Para complementar a etapa de validação é
possível fazer uma comparação mais detalhada com algum dos trabalhos
relacionados à nossa pesquisa, por exemplo, analisando se ambas visualizações
destacam os mesmos padrões. A partir disso, teríamos uma base melhor para
comparação dos resultados, em complemento de uma avaliação puramente feita
sobre a nossa visualização especificamente.  \citet{Guo2011} é o único trabalho
que avaliamos e que disponibilizou abertamente os dados utilizados e também sua
ferramenta.  Pretendemos verificar a viabilidade desta comparação com este ou
outros trabalhos de visualização de dados de origem e destino do tráfego, o que
seria um interessante complemento para esta pesquisa ou mesmo para trabalhos
futuros.


\section{Resultados Alcançados}

\section{Trabalhos Futuros}

