\chapter{Considerações Preliminares e Plano de Trabalho}
\label{cap:plano-de-trabalho}

  Construímos até aqui, os pilares teóricos e técnicos que irão sustentar nossa
caminhada até a conclusão dos objetivos de nosso trabalho, cujo foco é
investigar o uso de \emph{bundling} no estudo de fluxos de origem e destino na
cidade com dados reais e simulados do trânsito. Para atingirmos nossos
objetivos, trouxemos o conhecimento sobre práticas e algoritmos
considerados o estado da arte na área de análises de dados de tráfego. Do ponto
vista técnico, moldamos a arquitetura inicial da ferramenta InterSCityPlotter
com os componentes básicos que já permitem criar visualizações simples. A
seguir apontamos algumas considerações sobre decisões tomadas dentro do projeto
e o estado em que nos encontramos no presente momento.


\section{Esforço Técnico}

  Do ponto de vista técnico, estamos nos preparamos para uma implementação
própria, dado que não encontramos bibliotecas e implementações do algoritmo de
\emph{bundling} que possam ser reaproveitadas dentro da nossa proposta. Isso
leva a alguns desafios de implementação, sobre ferramentas e linguagens a serem
utilizadas, que diante do nosso conhecimento prévio, convergiu para o uso da
linguagem Python e da biblioteca Geoplotlib. Em testes preliminares com algumas
outras ferramentas para criação de visualizações sobre mapas (Bokeh e xd),
julgamos que essa opção nos dá maior flexibilidade e controle da implementação,
além de um bom suporte via sua documentação. 

Um outro aspecto técnico importante a ser considero, é que os trabalhos recentes sobre
as técnicas de \emph{bundling} apresentam abordagens com técnicas para
paralelismo e uso de recursos gráficos em placas de vídeo dedicadas (GPUs),
como é o caso de \citet{Hurter2012} e \citet{ADEB}. Dada a nossa análise prévia
dos trabalhos, consideramos isso se encaixa como uma otimização da nossa proposta, e que
por questões de tempo e complexidade, não serão incluídas, a princípio, em
nosso escopo.  Uma discussão sobre acelerações de vídeo são apresentadas por
\citet{}, que cita a dependência de hardware nesse tipo de implementação e
propõe uma solução mais genérica que usa tecnologia OpenGL, amplamente
disponível em diversos dispositivos computacionais, e utilizada também na
biblioteca Geoplotlib. 

\section{Escopo}
  
O escopo de nossa proposta parte da implementação de um algoritmo de
\emph{bundling} da literatura, aquisição e tratamento de dados do trânsito,
realização de experimentos e análise de resultados.  A Tabela \ref{table:scope}
mostra, no geral, os aspectos que cobriremos em nossa pesquisa e outras
atribuições que mapeamos para futuras evoluções sobre a visualização de dados
do tráfego em cima de nossa proposta

Escalabilidade, real time, other bundling algorithms

\subsection{Validação da Visualização}
  Consideramos este um aspecto crítico no campo da visualização, como afirmado
por \citet{Telea2018} e outros autores. No entanto, estudamos ainda a comparação
com a visualização de outros trabalhos, mas através do mesmo conjunto de dados.
\citet{Guo2011} é o único que disponibilizou abertamente os dados utilizados
e também sua ferramenta. Embora ainda não tivemos tempo de verificar a viabilidade
desta comparação, esta seria um interessante complemento para a nossa proposta.
Para isso, utilizaríamos os mesmos dados e faríamos uma comparação visual a respeito
das conclusões obtidas em cada visualização. Uma outra ideia que consideramos, porém
descartamos, é a aplicação de User Studies. Muitas vezes é difícil definir um estudo
compreensível e com uma boa quantidade de usuários. Apenas x e Y utilizam esse tipo
de validação, e de maneira bem restrita, com poucos usuários. Todavia, vemos que
isso não afeta de modo geral a validação de nossos resultados, e esperamos que isso
possa ser melhor repensado e viabilizado em pesquisas futuras.

\section{Plano de Trabalho}

 A Figura \ref{fig:gantt}, apresenta o nosso cronograma, com as atividades
previstas para os próximos 5 meses, demonstrando nossa perspectiva em finalizar
este projeto de pesquisa em Setembro de 2019.  Nosso maior desafio será na
implementação do algoritmo de \emph{bundling} e dos recursos interativos que
propomos dentro da visualização, como zoom, alteração de parâmetros, e outros.
Pretendemos finalizar esta etapa de desenvolvimento em Maio, para em seguida,
partirmos para a execução dos experimentos.


\begin{figure}[!htb]
  \centering

  \begin{ganttchart}{2019-03}{2019-9}
    \gantttitlecalendar{year,month=shortname} \ganttnewline

    \ganttgroup[progress=10]{Desenvolvimento}{2019-3}{2019-5} \ganttnewline
    \ganttbar[progress=15]{Aquisição e tratamento dos dados}{2019-3}{2019-3}\ganttnewline
    \ganttbar[progress=0]{Implementação da visualização}{2019-4}{2019-5} \ganttnewline

    \ganttgroup[progress=0]{Experimentos}{2019-6}{2019-7} \ganttnewline
    \ganttbar[progress=0]{Execução dos experimentos}{2019-6}{2019-6} \ganttnewline
    \ganttbar[progress=0]{Análise dos resultados}{2019-6}{2019-7} \ganttnewline

    \ganttgroup[progress=0]{Escrita da dissertação}{2019-7}{2019-9} \ganttnewline
    \ganttbar[progress=0]{Escrita}{2019-7}{2019-8} \ganttnewline
    \ganttbar[progress=0]{Revisão}{2019-9}{2019-9} \ganttnewline

    \ganttmilestone{Defesa}{2019-9}
  \end{ganttchart}

  \caption{Cronograma.\label{fig:gantt}}
\end{figure}

