\chapter{Considerações Preliminares e Plano de Trabalho}
\label{cap:plano-de-trabalho}

  Construímos até aqui os pilares teóricos e técnicos que irão sustentar nossa
caminhada até a conclusão dos objetivos de nosso trabalho, cujo foco é
investigar o uso de \emph{bundling} no estudo de fluxos de origem e destino na
cidade com dados do trânsito. Para atingirmos nossos
objetivos, trouxemos o conhecimento sobre práticas e algoritmos
considerados o estado da arte na área de análises de dados de tráfego. Do ponto
vista técnico, moldamos a arquitetura inicial da ferramenta InterSCityPlotter
com os componentes básicos que já permitem criar visualizações simples. Abordamos
a seguir, algumas considerações a respeito do que produzimos até aqui.

\section{Esforço Técnico}

  Bibliotecas abertas com implementações de algoritmos de \emph{bundling} ainda
não são algo bem estabelecido, diferente de algoritmos de clusterização, que
possuem vários pacotes disponíveis (e.g. Sklearn, Theano, Keras, TensorFlow). Dessa forma, seguiremos com uma
implementação própria dentro de nossa proposta, o que de certa forma tem suas
vantagens, já que assim temos uma maior flexibilidade e controle da implementação
e das tecnologias utilizadas. 

 Dada a nossa experiência prévia e alguns
testes com algumas ferramentas de visualização na linguagem Python
(Geoplotlib\footnote{\rurl{github.com/andrea-cuttone/geoplotlib}},
Folium\footnote{\rurl{github.com/python-visualization/folium}} e
Dash\footnote{\rurl{github.com/plotly/dash}}), optamos pelo o uso da
biblioteca Geoplotlib, apresentado em \citet{Andrea2016}. Essa biblioteca
permite a criação de visualizações com vários recursos de linhas, pontos,
funções para manipulação de polígonos, latitudes, longitudes, e outras
facilidades para o desenho de objetos sobre mapas. Ela conta ainda com algumas
visualizações pré-estabelecidas, como mapas de calor, mapas de densidade,
histogramas e outras.

 Nossos esforços de implementação até aqui resultaram no
desenvolvimento da arquitetura básica do InterSCityPlotter. A partir dessa
implementação, é possível criar visualizações simples, como as que apresentamos
nas Figuras \ref{fig:simulated-traffic} e \ref{fig:rastro}. Valorizamos esse
esforço como forma de validar os recursos e limitações das tecnologias
escolhidas, e consideramos que estamos no caminho certo para cumprirmos os
nossos objetivos.

  Um outro aspecto técnico importante é sobre a aquisição dos dados para os
experimentos. Apresentamos em nossa metodologia que iremos utilizar dados reais
gerados pelo simulador InterSCSimulator e também dados reais de um dia do
tráfego de ônibus na cidade de São Paulo. Quanto aos dados simulados,
precisamos obter a execução da simulação na ocorrência de eventos atípicos, no
caso o fechamento de vias.  Este cenário está sendo elaborado no trabalho do
colega de mestrado Dylan Guedes, cuja pesquisa irá utilizar o InterSCSimulator
para processamento de dados em tempo real com ferramentas de \emph{big data}.
Quanto aos dados reais dos ônibus da cidade, estes ainda serão coletados
diretamente da API do sistema da prefeitura. Contudo, há a possibilidade de
conseguir esses dados com a empresa
Scipopulis\footnote{\rurl{www.scipopulis.com/}} que trabalha com a coleta e
análise de informações do transporte público nas cidades.

%  Um outro aspecto técnico importante a ser considero, é que os trabalhos
%recentes sobre as técnicas de \emph{bundling} apresentam abordagens com
%técnicas para paralelismo e uso de recursos gráficos em placas de vídeo
%dedicadas (GPUs), como é o caso de \citet{Hurter2012} e \citet{ADEB}. Dada a
%nossa análise prévia dos trabalhos, consideramos isso se encaixa como uma
%otimização da nossa proposta, e que por questões de tempo e complexidade, não
%farão parte de nosso escopo.  Uma discussão sobre acelerações de vídeo é
%apresentada em \citet{}. O uso de tecnologias como Cuda, OpenCL, requerem
%hardware específico e conhecimento específico sobre essas tecnologias. Eles
%propõe o uso de uma solução mais genérica propõe uma solução mais genérica que
%usa tecnologia OpenGL, amplamente disponível em diversos dispositivos
%computacionais, e utilizada também na biblioteca Geoplotlib. O custo benefício,
%é que a diferença no cálculo do \emph{bundling} para um grafo com 2000 nós é
%relativamente grande. Mas, dadas as limitações de tempo na elaboração deste
%trabalho, este é um tópico que pode ser abordado em trabalhos futuros.

%\section{Objetivos}
%  
%O objetivo de nossa proposta parte da investigação do \emph{bundling} na visualização
%dos fluxos de origem e destino no trânsito. Para atingir esse objetivo, várias
%etapas serão executadas, como a implementação de um algoritmo de
%\emph{bundling} da literatura, aquisição e tratamento de dados do trânsito,
%realização de experimentos e análise de resultados. Neste cenário, focamos
%nas questões de pesquisa levantadas no capítulo 1, mas que também estamos
%cientes de outros aspectos cientificamente interessantes, mas que são adjacentes
%neste ponto da pesquisa,  porém não deixamos de mapeá-los, como listamos a seguir:
%
%A Tabela \ref{table:scope}
%mostra, no geral, os aspectos que cobriremos em nossa pesquisa e outras
%atribuições que mapeamos para futuras evoluções sobre a visualização de dados
%do tráfego em cima de nossa proposta
%
%Escalabilidade: Esta questão está intimamente ligada ao uso de tecnologias de paralelismo
%em placas gráficas, como apresentamos na seção anterior. Seria interessante uma
%investigação mais aprofundada sobre este tópico.
%
%Tempo Real: Nossa proposta está voltada para análise de dados em lotes, ou seja,
%recebemos um conjunto de arquivos do tráfego, processamos e construímos a visualização.
%Uma das áreas de pesquisa futuras citadas por \cite{Fulano} é justamente a construção
%de soluções capazes de analisar grandes quantidades de dados em tempo real, o que
%poderia estar voltado para uma integração com o simulador InterSCSimulator como
%um caminho alternativo de pesquisa.
%
%Outros algoritmos de \emph{bundling}: 
%
%Escalabilidade, real time, other bundling algorithms

\subsection{Validação da Visualização}

 Percebemos que um aspecto delicado em pesquisas de visualização de dados é a
validação da própria visualização. \citet{Telea2018} cita que a definição de
critérios objetivos que possam ser aplicados para comparar diferentes propostas
ainda é uma questão em aberto. Para complementar a etapa de validação é
possível fazer uma comparação mais detalhada com algum dos trabalhos
relacionados à nossa pesquisa, por exemplo, analisando se ambas visualizações
destacam os mesmos padrões. A partir disso, teríamos uma base melhor para
comparação dos resultados, em complemento de uma avaliação puramente feita
sobre a nossa visualização especificamente.  \citet{Guo2011} é o único trabalho
que avaliamos e que disponibilizou abertamente os dados utilizados e também sua
ferramenta.  Pretendemos verificar a viabilidade desta comparação com este ou
outros trabalhos de visualização de dados de origem e destino do tráfego, o que
seria um interessante complemento para esta pesquisa ou mesmo para trabalhos
futuros.

\section{Plano de Trabalho}

 A Figura \ref{fig:gantt}, apresenta o nosso cronograma, com as atividades
previstas para os próximos 8 meses, demonstrando nossa perspectiva em finalizar
este projeto de pesquisa em Outubro de 2019.

  Na etapa inicial de desenvolvimento estaremos focados em desenvolver as
funcionalidades da ferramenta InterSCityPlotter, de forma que ela suporte de
maneira consistente e padronizada tanto as entradas do simulador, quanto os
dados reais que iremos coletar. E também, iremos finalizar a implementação e
testes do componente de visualização (\emph{VisualizationLayers}), que precisa
de uma boa definição de suas interfaces para facilitar a construção de novos
tipos de visualização no futuro. Após esse desenvolvimento inicial, nosso
segundo maior desafio será na implementação do algoritmo de \emph{bundling} e
dos recursos interativos que propomos dentro da nossa visualização, como zoom,
alteração de parâmetros, e outros.  Paralelamente às etapas de desenvolvimento da visualização propriamente
dita, há um esforço técnico também com a obtenção e tratamento dos dados, mas
que demanda um percentual menor do tempo de trabalho. A princípio já possuímos
uma amostra de dados suficientes para prosseguir no projeto enquanto concluímos
a aquisição completa dos conjuntos de dados de forma paralela. Lembramos que os
dados da simulação com eventos atípicos serão produzidos na pesquisa do colega
de mestrado Dylan Guedes. Pretendemos finalizar toda a etapa de desenvolvimento
até Maio, para em seguida, partirmos para a execução dos experimentos em um
ciclo de 2 meses, o qual prevemos também uma parcela de esforço para
identificar e aplicar melhorias na ferramenta que não foram detectadas na etapa
anterior. Por fim, seguimos para a fase final de documentação dos resultados em
formato da dissertação de mestrado e também em artigo para conferência na área
de visualização, os quais pretendemos finalizar em Outubro.

\begin{figure}[!htb]
  \centering

  \begin{ganttchart}{2019-02}{2019-11}
    \gantttitlecalendar{year,month=shortname} \ganttnewline

    \ganttbar[progress=15]{Desenvolvimento da visualização}{2019-3}{2019-5} \ganttnewline
    \ganttbar[progress=15]{Coleta e tratamento dos dados}{2019-3}{2019-4}\ganttnewline
    \ganttbar[progress=0]{Execução dos experimentos}{2019-6}{2019-7} \ganttnewline
    \ganttbar[progress=0]{Análise dos resultados}{2019-8}{2019-8} \ganttnewline
    \ganttbar[progress=0]{Escrita da Dissertação}{2019-8}{2019-10} \ganttnewline
    \ganttbar[progress=0]{Escrita de Artigo}{2019-9}{2019-10} \ganttnewline
    \ganttmilestone{Defesa}{2019-10}
  \end{ganttchart}

  \caption{Cronograma.\label{fig:gantt}}
\end{figure}


\section{Resultados Esperados}

  Esperamos que ao fim desta pesquisa nos possamos obter de forma geral uma
maior clareza das possibilidades e limitações do uso de \emph{bundling} na
visualização dos fluxos de origem e destino no trânsito da cidade. Presumimos que
nossa abordagem de visualização das macro trajetórias e sub trajetórias, sob
diferentes escalas, seja capaz de avaliar o uso da técnica para análise de
padrões globais e locais do tráfego. Acreditamos também que o uso de dados
simulados possa contribuir com uma observação diferenciada sobre o
comportamento da visualização com \emph{bundling} em diferentes cenários do
trânsito, trazendo benefícios para as áreas da visualização e de
simulação. Por fim, desejamos que a ferramenta de visualização que
estamos desenvolvendo possa colaborar ainda para pesquisas futuras dentro do
projeto InterSCity e que a ciência possa continuar a florescer a partir desta
semente plantada.
