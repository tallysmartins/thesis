\chapter{Trabalhos Relacionados}
\label{cap:trabalhos-relacionados}

  A visualização de dados de tráfego tem um amplo histórico, nossa revisão traz
alguns dos trabalhos que mostram os avanços nessa área e também relacionados ao
\emph{bundling}. As pesquisas que julgamos relevantes tocam as áreas de
visualização da informação - \emph{InfoViz}, cujo foco é desvendar novos
métodos para visualizar uma informação, e também a área de análise visual
aplicada - \emph{Visual Analytics}, que se relaciona mais a sistemas e soluções
utilizadas em ambientes do mundo real. Ambos trabalhos trazem contribuições em
diferentes níveis para a nossa pesquisa. Grande parte do conhecimento em
de visualização do tráfego e também sobre \emph{bundling} pode ser encontrado em três
trabalhos recentes que fizeram um levantamento geral sobre os esforços na área.


  \citet{Telea2018} apresenta uma revisão geral das áreas de visualização
científica e de processamento de imagem, apontando métodos utilizados nessas
áreas para que beneficiaram a visualização de grandes grafos multivariados e
complexos. Os novos métodos para simplificação de dados, então chamados de
baseados em imagem, geraram novas técnicas de \emph{bundling} escaláveis
visualmente e computacionalmente.  O algoritmo Attributed-Driven Edge Bundling,
que utilizaremos, também se beneficia desses métodos. Eles apresentam ainda
alguns desafios no campo da visualização de grandes grafos com \emph{bundling},
como a falta de meios estabelecidos para avaliação da qualidade do desenho e
também a limitação na quantidade máxima de atributos que podem ser
representados ao mesmo tempo na imagem. Até o presente momento, desconhecemos
soluções para esses problemas.

  \citet{Lhuillier2017} traz um estudo ainda mais aprofundado sobre o
\emph{bundling}. Os autores sugerem uma definição formal sobre a operação de
\emph{bundling} em um conjunto de dados, e a partir daí definem o arcabouço em
volta dos objetivos e limitações de uso das diversas técnicas e algoritmos
existentes na literatura, dando uma visão geral sobre o estado da arte na área
e seu desenvolvimento em vários segmentos como grafos, mapas de fluxo,
coordenadas paralelas, campos de tensores e outras aplicações. Dentre suas
contribuições está uma nova taxonomia para a ajudar pesquisadores e usuários a
selecionarem os algoritmos de \emph{bundling} com base no tipo de dado.
Apresentamos essa taxonomia na Figura \ref{table:bundling-methods} na Seção
\ref{sec:modelos-de-bundling}. A partir dela, selecionamos o algoritmo ADEB
para a nossa pesquisa.

  \citet{Andrienko2017Visual}, \textbf{Visual Analytics of mobility and
Transportation: State of the Art and Further Research Directions:} Analisa
vários trabalhos de modo geral de análise de dados do tráfego e os divide em 6
módulos quanto ao contexto. Ele faz várias considerações sobre a tipologia dos
dados no contexto de análises de trajetórias, objetivos. 

  \citet{Chen2015} traz ainda faz uma revisão de vários trabalhos de análise do tráfego
e os reuni em uma tabela com base no tipo de tarefa a que se propõe a resolver,
tipo de dados, linguagem de programação e principal funcionalidade. Extraímos
de seu estudo os trabalhos que estão intimamente ligados à análise de
fluxos de origem-destino e seus padrões de deslocamento sobre o território. Ainda
complementamos a lista com outros trabalhos posteriores e os julgamos conforme
algumas propriedades a qual pretendemos cobrir com a nossa proposta.

  Investigamos também outros trabalhos que buscam especificamente analisar
fluxos de origem-destino em dados do tráfego, seja utilizando \emph{bundling}
ou não. Dentre essa lista incluem-se propostas de novas técnicas de
visualização e de sistemas analíticos que implementam técnicas existentes.
Analisamos esses trabalhos nos seguintes aspectos, nível de detalhes da
visualização (multi escala), utilização de dados do trânsito, quantidade de
dados e uso de \emph{bundling}. Boa parte desses trabalhos foram adquiridos
a partir do levantamento feito em \citet{Chen2015}.

\begin{table}[]
\begin{tabular}{|c|c|c|c|c|c|}
\hline
\textbf{Trabalho} & \textbf{Qtd. Pontos} & \textbf{Dados do Trânsito} & \textbf{\emph{Bundling}} & \textbf{Multi Escala} & \textbf{Simulação} \\ \hline
\citet{Zeng2013}  &          ?           & x                          & \checkmark               & \checkmark    & x                          \\ \hline
\citet{Andrienko2017} &      ?           & \checkmark                 &  x                       & \checkmark    & x                          \\ \hline
\citet{Anita2017} &          ?           & x                          &  \checkmark              &   ?           & x                          \\ \hline
\citet{Landersberg2016} &    ?           & x                          &  ?                       &   ?           & x                          \\ \hline
\citet{Klein2013} &          ?           & x                          & \checkmark               &   ?           & x                          \\ \hline
Nossa Proposta    &   31 milhões         & \checkmark                 & \checkmark               & \checkmark    & \checkmark                 \\ \hline

\end{tabular}
\caption{Análise dos Trabalhos Relacionados}
\end{table}

  \citet{Zeng2013} e \citet{Andrienko2017} apresentam sistemas interativos para
a visualização de fluxos de origem e destino em dados geoespaciais gerais.
Ambos os trabalhos contribuem com novos tipos de leiaute radial que codificam a
direção, intensidade e distância percorrida, das trajetórias. \citet{Zeng2013}
ainda faz o uso de \emph{bundling} para agrupar as linhas dentro do anel
radial. \citet{Anita2017} apresenta uma nova técnica de \emph{bundling},
resultado de uma otimização do algoritmo FDEB, criado pela primeira vez por
\citet{Selassie2011}. Para isso, utilizam uma etapa de clusterização aplicada
previamente nos dados, e posteriormente aplicam o \emph{bundling} em cada
cluster. O benefício da técnica é que ela pode ser utilizada em conjunto com
outros métodos de \emph{bundling}. \citet{Landersberg2016} apresenta também
apresenta uma técnica de \emph{bundling} similar ao algoritmo KDEEB, mas utilizando
o algoritmo de clusterização DBSCAN no cálculo do mapa de densidades. Além disso,
compõe a visualização das linhas das trajetórias agrupadas em círculos
posicionados nos centroides de cada região analisada, diminuindo a quantidade
de cruzamentos das arestas visualizadas.  \citet{Klein2013}

\begin{description}
  \item[Design:] Os trabalhos encontrados utilizam um design diferente do que propomos a seguir.
  Muitos deles fazem análises multinível, e com leiaute em linhas ou em circular, mas com outros tipos de dado
  e em outros contextos, como avião, navios e pessoas.

  \item[Tipo de Dado:] Os trabalhos de bundling encontrados utilizam bundling
  apenas com as informações de origem-destino e não todos os dados da trajetória.
  Fulano de tal faz o uso de mapas de tensores com dados mais granulares, mas
  utiliza uma outra abordagem no design da visualização, o que difere do contexto
  deste trabalho.

  \item[Escala:] Poucos trabalhos utilizam uma grande quantidade de dados. A visualização
  proposta, apesar de não estar focada em escalabilidade computacional, irá usar
  uma grande quantidade de dados do trânsito, para lidar com aspectos de escalabilidade
  visual.
\end{description}

