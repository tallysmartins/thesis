\chapter{Trabalhos Relacionados}
\label{cap:trabalhos-relacionados}

  A técnica de agrupamento de arestas na visualização de grafos, redes e trajetórias
tem sido utilizada em vários contextos. Neste capítulo apresentamos alguns trabalhos
que utilizam Bundling para a visualização de trajetórias. Lembramos que a definição
de trajetórias está ligada à codificação de informações espaciais intrínseca aos
dados, como na movimentação de veículos ou em gráficos de coordenadas paralelas.
De forma geral, tais trabalhos proporcionaram uma base para compor um conhecimento maior
sobre a área.

\begin{description}

  \item[Visualization of vessel movements:]
  Esse trabalho mostra a dinâmica do tráfego de embarcações. \citep{Willems2009}

  \item[Visualizing Interchange Patterns in Massive Movement Data:] esse
trabalho apresenta um diagrama circular que mostra a direção do fluxo entre
diferentes nós que simbolizam um ponto de interesse. Ele utiliza como estudo de
caso estações de metrô de Singapura com dados sigilosos do sistema de
transporte. \citep{Zeng2013} 

\item[Dynamic Multi Scale Visualization of Flight Data:]
 Esse trabalho do apresenta como visualizar uma grande quantidade
de dados do tráfego aéreo. \citep{Klein2014}

\item[Untangling origin-destination flows in geographic information systems:]
 Esse trabalho mostra a implementação do bundling como plugin de uma ferramenta GIS.
O algoritmo apresentado ainda recebe uma modificação ao algoritmo de bundling FDEB. \citep{Anita2017}

\item[Divided Edge Bundling for Directional Network Data:]
  Esse trabalho mostra uma abordagem para diferenciar no bundling a direção do movimento.
\citep{Selassie2011}
\end{description}

\begin{description}
  \item[Design:] Os trabalhos encontrados utilizam um design diferente do que propomos a seguir.
  Muitos deles fazem análises multinível, e com leiaute em linhas ou em circular, mas com outros tipos de dado
  e em outros contextos, como avião, navios e pessoas.

  \item[Tipo de Dado:] Os trabalhos de bundling encontrados utilizam bundling
  apenas com as informações de origem-destino e não todos os dados da trajetória.
  Fulano de tal faz o uso de mapas de tensores com dados mais granulares, mas
  utiliza uma outra abordagem no design da visualização, o que difere do contexto
  deste trabalho.

  \item[Escala:] Poucos trabalhos utilizam uma grande quantidade de dados. A visualização
  proposta, apesar de não estar focada em escalabilidade computacional, irá usar
  uma grande quantidade de dados do trânsito, para lidar com aspectos de escalabilidade
  visual.
\end{description}

\begin{table}[]
\caption{Comparação entre visualizações baseadas em mapas}
\begin{tabular}{|c|c|c|l|c|c|l|l|}
\hline
\textbf{Trabalho} & \textbf{Granularidade do dado} & \textbf{Tipo de Visualização} & /Tipo de Dado & \textbf{Direção} & \textbf{Dinâmica} & Nível de Detalhe & Interativa \\ \hline
                  & OD                             & Baseada em linhas             & Avião         & Não              &                   & Multi nível      &            \\ \hline
                  & OD                             & Mapa de OD                    & Carro         & Sim              &                   &                  &            \\ \hline
                  & OD                             & Diagramas Circulares          &               &                  &                   &                  &            \\ \hline
                  & Full                           & Tensor Fields                 &               & sim              & não               & único            &            \\ \hline
\end{tabular}
\end{table}

%  Por razões críticas, a movimentação de aeronaves no tráfego aéreo é monitorada
%  com grande cuidado. Sistemas avançados de localização registram em tempo real
%  a posição de aeronaves que sobrevoam o planeta. Esses sistemas podem ainda
%  conter outras informações como direção, altitude, velocidade, temperatura, pressão
%  e diversos outros atributos ao longo do trajeto. O tráfego aéreo é de certa
%  forma similar à movimentação de veículos sobre as vias da cidade e visualizar
%  essas informações traz também grandes desafios.
%
%  cite{Alex}, mostra um trabalho onde ele descreve uma visualização que utiliza
%  técnicas baseadas em imagem para análise exploratória de uma grande massa
%  de dados do tráfego. Sua análise inclui a utilização de três técnicas, bundling,
%  density maps e animações. Ele argumenta que, com essas técnicas, é possível visualizar
%  uma grande quantidade de dados, tanto a posição instantânea dos aviões como também
%  sua dinâmica ao longo do tempo, sem uma grande oclusão da visualização. 
%  Dentre os resultados apresentados está a detecção de \textit{outliers}, padrões
%  e congestionamentos durante vôos.
%
%  
%  O seu sistema possui alguns parâmetros configuráveis que permitem uma exploração dos dados que
%  trazem vários insights para visualização de padrões e outliers nos dados.
%
%-> Highlights:
%  - exploração das informações disponíveis e detecção de outliers
%  - visão de grandes áreas do tráfego e durante longos períodos (e.g. dados do mundo durante 1 mês)
%  - adaptação de várias técnicas baseadas em imagem (bundling, animation, density maps) para visualizar
%  padrões ao longo do tempo
%  - visualização com pouca oclusão
%  - análise em "tempo real" com processamento na GPU
%  - real world datasets
%
%-> Limitações
%  - Ainda gera alguma oclusão, como no dataset do mundo todo
%  - Software não disponível
%  - não implementa queries para filtros (e.g. aviões de altitude maior que X)
%  - Suas trilhas apresentam apenas três atributos (velocidade, direção, altura)

