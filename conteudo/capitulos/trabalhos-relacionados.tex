\chapter{Trabalhos Relacionados}
\label{cap:trabalhos-relacionados}

  A visualização de dados de tráfego tem um amplo histórico, nossa revisão traz
alguns dos trabalhos que mostram os avanços nessa área e também relacionados ao
\emph{bundling}. As pesquisas que julgamos relevantes tocam as áreas de
visualização da informação - \emph{InfoViz}, cujo foco é desvendar novos
métodos para visualizar uma informação, e também a área de análise visual
aplicada - \emph{Visual Analytics}, que se relaciona mais a sistemas e soluções
utilizadas em ambientes do mundo real. Ambos trabalhos trazem contribuições em
diferentes níveis. Inicialmente apresentamos três trabalhos recentes que
fizeram um levantamento geral sobre os esforços nas áreas de visualização do
tráfego e também sobre \emph{bundling}, que complementam o trabalho usado
como referência na Seção \ref{sec:viz-trafego}. Trazemos também 7 outros artigos
que apresentam propriamente técnicas de visualização de dados do tráfego,
utilizando ou não \emph{bundling}, e também possíveis sistemas que as
implementam.

  \citet{Telea2018} apresenta uma revisão geral das áreas de visualização
científica e os métodos de processamento de imagem utilizados nesse campo para
mostrar seus benefícios na visualização de grandes grafos multivariados e
complexos. Os novos métodos, então chamados de baseados em imagem, ajudam a
resolver os problemas de oclusão e também aumentam a escalabilidade
computacional com técnicas que usam o poder de processamento paralelo de placas
gráficas (GPUs), o que está relacionado ao algoritmo de \emph{bundling} que
utilizaremos. Eles apresentam ainda alguns desafios ainda existentes no
campo, como a falta de meios estabelecidos para avaliação da qualidade do
desenho e também a limitação na quantidade máxima de atributos que podem ser
representados ao mesmo tempo na visualização de um grafo.

  \citet{Lhuillier2017} traz um estudo mais aprofundado sobre o
\emph{bundling}. Os autores sugerem uma definição formal sobre a operação de
\emph{bundling} em um grafo, e a partir daí definem o arcabouço em volta dos
objetivos e limitações de uso das diversas técnicas de \emph{bundling}
existentes na literatura, dando uma visão geral sobre o estado da arte na área
e seu desenvolvimento em vários segmentos como grafos, mapas de fluxo,
coordenadas paralelas, campos de tensores e outras aplicações. Dentre suas
contribuições está uma nova taxonomia para a ajudar pesquisadores e usuários a
selecionarem os algoritmos de \emph{bundling} com base no tipo de dado.
Apresentamos essa taxonomia na Figura \ref{table:bundling-methods} na Seção
\ref{sec:modelos-de-bundling}. A partir dela, selecionamos o algoritmo
\emph{Attributed-Driven Edge Bundling} (ADEB) para a nossa pesquisa.

  \citet{Andrienko2017Visual}, \textbf{Visual Analytics of mobility and Transportation: State of the Art and
Further Research Directions:} Analisa vários trabalhos de modo geral de análise
de dados do tráfego e os divide em 6 módulos quanto ao contexto. Ele faz várias
considerações sobre a tipologia dos dados no contexto de análises de trajetórias,
objetivos. 

   \citet{Zeng2013} e \citet{Andrienko2017} apresentam uma visualização para o
estudo de fluxos de origem e destino de propósito geral. Sua abordagem utiliza
um leiaute radial que codifica a direção, intensidade e distância percorrida,
capaz de informar principais fluxos e padrões de deslocamento sobre o
território observado. O leiaute radial, no entanto, deixa de codificar vários
dos aspectos espaciais presentes em uma visualização baseada em linhas,
principalmente quando se deseja identificar mudanças de direção em regiões
específicas.

  \citet{Selassie2011}, apresenta o Divided Edge Bundling (DEB), o primeiro
algoritmo conhecido capaz de diferenciar a direção dos fluxos. Recentemente,
\citet{Anita2017} apresentam uma otimização deste algoritmo para aumentar a
velocidade do processo de \emph{bundling}. Para isso, utilizam uma etapa de
clusterização aplicada previamente nos dados, e posteriormente aplicam o
\emph{bundling} em cada cluster, uma técnica que pode ser estendida para outros
métodos de \emph{bundling}.  Outra importante contribuição desse trabalho é que
a sua implementação é uma das poucas que conhecemos feita de maneira aberta, e
foi desenvolvida como uma extensão da ferramenta GQIS, para análises
geoespaciais.  Percebemos a falta bibliotecas disponíveis que implementam esse
tipo de algoritmo, e por isso, nossa solução será totalmente aberta.

  \citet{Landersberg2016}, \textbf{MobilityGraphs: Visual Analysis of Mass Mobility Dynamics via Spatio-Temporal Graphs and Clustering:} apresenta um
estudo com clusterização DBSCAN parecido com o KDEEB. MAs a visualização combina
um leiaute de linhas com bolas nos centróides. Eles conseguem ver os fluxos,
direção, etc. São também multi nível, com opção de zoom, bem legal. Fazem estudo de caso em X dataset e conseguem mostrar Y.

  \citet{Klein2013}, \textbf{Real Time Multi Scale Visualization of Flight Data:} Esse é o dos aviões, sem mais.

  \textbf{Visualizing Interchange Patterns in Massive Movement Data:} esse
trabalho apresenta um diagrama circular que mostra a direção do fluxo entre
diferentes nós que simbolizam um ponto de interesse. Ele utiliza como estudo de
caso estações de metrô de Singapura com dados sigilosos do sistema de
transporte.

  \citet{Selassie2011}, \textbf{Divided Edge Bundling for Directional Network
Data:} esse algoritmo evolui o FDEB pra ser direcional.

  \citet{Willems2009}, \textbf{Visualization of vessel movements:} Esse trabalho mostra a dinâmica do tráfego de embarcações. 


\begin{description}
  \item[Design:] Os trabalhos encontrados utilizam um design diferente do que propomos a seguir.
  Muitos deles fazem análises multinível, e com leiaute em linhas ou em circular, mas com outros tipos de dado
  e em outros contextos, como avião, navios e pessoas.

  \item[Tipo de Dado:] Os trabalhos de bundling encontrados utilizam bundling
  apenas com as informações de origem-destino e não todos os dados da trajetória.
  Fulano de tal faz o uso de mapas de tensores com dados mais granulares, mas
  utiliza uma outra abordagem no design da visualização, o que difere do contexto
  deste trabalho.

  \item[Escala:] Poucos trabalhos utilizam uma grande quantidade de dados. A visualização
  proposta, apesar de não estar focada em escalabilidade computacional, irá usar
  uma grande quantidade de dados do trânsito, para lidar com aspectos de escalabilidade
  visual.
\end{description}

\begin{table}[]
\caption{Comparação entre visualizações baseadas em mapas}
\begin{tabular}{|c|c|c|l|c|c|l|l|}
\hline
\textbf{Trabalho} & \textbf{Granularidade do dado} & \textbf{Tipo de Visualização} & /Tipo de Dado & \textbf{Direção} & \textbf{Dinâmica} & Nível de Detalhe & Interativa \\ \hline
                  & OD                             & Baseada em linhas             & Avião         & Não              &                   & Multi nível      &            \\ \hline
                  & OD                             & Mapa de OD                    & Carro         & Sim              &                   &                  &            \\ \hline
                  & OD                             & Diagramas Circulares          &               &                  &                   &                  &            \\ \hline
                  & Full                           & Tensor Fields                 &               & sim              & não               & único            &            \\ \hline
\end{tabular}
\end{table}

%  Por razões críticas, a movimentação de aeronaves no tráfego aéreo é monitorada
%  com grande cuidado. Sistemas avançados de localização registram em tempo real
%  a posição de aeronaves que sobrevoam o planeta. Esses sistemas podem ainda
%  conter outras informações como direção, altitude, velocidade, temperatura, pressão
%  e diversos outros atributos ao longo do trajeto. O tráfego aéreo é de certa
%  forma similar à movimentação de veículos sobre as vias da cidade e visualizar
%  essas informações traz também grandes desafios.
%
%  cite{Alex}, mostra um trabalho onde ele descreve uma visualização que utiliza
%  técnicas baseadas em imagem para análise exploratória de uma grande massa
%  de dados do tráfego. Sua análise inclui a utilização de três técnicas, bundling,
%  density maps e animações. Ele argumenta que, com essas técnicas, é possível visualizar
%  uma grande quantidade de dados, tanto a posição instantânea dos aviões como também
%  sua dinâmica ao longo do tempo, sem uma grande oclusão da visualização. 
%  Dentre os resultados apresentados está a detecção de \textit{outliers}, padrões
%  e congestionamentos durante vôos.
%
%  
%  O seu sistema possui alguns parâmetros configuráveis que permitem uma exploração dos dados que
%  trazem vários insights para visualização de padrões e outliers nos dados.
%
%-> Highlights:
%  - exploração das informações disponíveis e detecção de outliers
%  - visão de grandes áreas do tráfego e durante longos períodos (e.g. dados do mundo durante 1 mês)
%  - adaptação de várias técnicas baseadas em imagem (bundling, animation, density maps) para visualizar
%  padrões ao longo do tempo
%  - visualização com pouca oclusão
%  - análise em "tempo real" com processamento na GPU
%  - real world datasets
%
%-> Limitações
%  - Ainda gera alguma oclusão, como no dataset do mundo todo
%  - Software não disponível
%  - não implementa queries para filtros (e.g. aviões de altitude maior que X)
%  - Suas trilhas apresentam apenas três atributos (velocidade, direção, altura)

