\chapter{Trabalhos Relacionados}
\label{cap:trabalhos-relacionados}

  A visualização de dados de tráfego tem um amplo histórico. Nossa revisão traz
alguns dos trabalhos e avanços recentes nessa área. As pesquisas que julgamos
relevantes tocam as áreas de visualização da informação - \emph{InfoViz}, cujo
foco é desvendar novos métodos para visualizar uma informação, como
\emph{bundling}, e também a área de análise visual aplicada - \emph{Visual
Analytics}, que se relaciona mais a sistemas e soluções utilizadas em ambientes
do mundo real. Ambos trabalhos trazem contribuições em diferentes níveis para a
nossa pesquisa. Grande parte do conhecimento em visualização do tráfego e
também sobre \emph{bundling} pode ser encontrado em quatro trabalhos recentes
que fizeram um levantamento geral sobre os esforços na área.

  \citet{Telea2018} apresenta uma revisão geral das áreas de visualização
científica e de processamento de imagem, apontando métodos utilizados nessas
áreas para que beneficiaram a visualização de grandes grafos multivariados e
complexos. Os novos métodos para simplificação de dados, então chamados de
baseados em imagem, geraram novas técnicas de \emph{bundling} escaláveis, que
permitiram a visualização de grandes grafos, com diversos atributos e
milhares de nós. O algoritmo \emph{Attributed-Driven Edge Bundling} (ADEB), que utilizaremos,
é um dos algoritmos que vêm dessa linhagem e se beneficia desses métodos. 

  \citet{Lhuillier2017} traz um estudo ainda mais aprofundado sobre o
\emph{bundling}. Os autores sugerem uma definição formal sobre a operação de
\emph{bundling} em um conjunto de dados, e a partir daí listam as características,
objetivos e limitações de uso das várias técnicas e algoritmos
existentes na literatura. O resultado é uma visão geral sobre o estado da arte na área
e seu desenvolvimento em vários segmentos como grafos, mapas de fluxo,
coordenadas paralelas, campos de tensores e outras aplicações. Dentre suas
contribuições está uma nova taxonomia para a ajudar pesquisadores e usuários a
selecionarem os algoritmos de \emph{bundling} com base no tipo de dado.
Apresentamos essa taxonomia na Figura \ref{table:bundling-methods} na Seção
\ref{sec:modelos-de-bundling}. A partir dela, selecionamos o algoritmo ADEB
para a nossa pesquisa.

  Já \citet{Andrienko2017Visual} e \citet{Chen2015} avaliam uma série de outros
trabalhos que fazem análises de dados do tráfego e de mobilidade em geral e
discutem questões sobre os tipos de dados, técnicas de visualização utilizadas
e principalmente o objetivo das propostas. Os trabalhos listados abrangem
vários contextos, como análise de incidentes no tráfego, monitoramento de
veículos em tempo real, detecção de congestionamentos, sugestão de rotas e
outras atividades executadas por usuários e especialistas de transporte. (O que
eles concluem?)

  Além dos estudos citados anteriormente, reunimos um conjunto de propostas
cujo o foco está na análise de fluxos de origem-destino em dados do tráfego
para visualização de tendências, fluxos dominantes e seus atributos, como
direção, velocidade, distâncias percorridas e outros.  \citet{Zeng2013} e
\citet{Andrienko2017}  apresentam sistemas interativos para a visualização de
fluxos de origem e destino em dados geoespaciais gerais, logo se encaixam nesse
grupo. Eles propõem novos tipos de leiaute radial para visualizar o tráfego, e
codificam a direção, intensidade e distância percorrida pelos objetos que se
locomovem. \citet{Zeng2013} menciona o uso de \emph{bundling} para agrupar as
linhas dentro do anel de seu leiaute radial (Figura
\ref{fig:interchange-circo}), mas não se aprofundam no assunto. 

 \citet{Landersberg2016} usa uma representação em linhas do tráfego sobre um
mapa e apresenta algoritmos de clusterização para agrupar o tráfego por regiões
e também no tempo, reduzindo a oclusão do desenho. Sua análise é focada em destacar diferentes momentos onde há
mudanças significativas nos fluxos. O desafio de sua técnica é estabelecer métricas
que capturam esses momentos.  Eles apresentam um sistema interativo e
verificam seu método com dados de \emph{tweets} geolocalizados na cidade de Londres e
também de redes de celulares. \citet{Klein2014} faz uma análise do tráfego aéreo
da França para detectar as conexões entre os diferentes aeroportos, pontos de
congestionamentos e permitir uma exploração com base em vários atributos, como
direção e altitude dos voos, além de uma janela de tempo que permite navegar
entre diferentes instantes do tráfego. Eles utilizam o algoritmo de
\emph{bundling} KDEEB para reduzir a oclusão na visualização das trajetórias
e apresentam uma interpolação das trajetórias sem \emph{bundling} para visualizar
sua direção, já que o KDEEB não inclui atributos como direção no processo de \emph{bundling}.

\citet{Ferreira2013} fazem uma análise de origem e destino de 500 mil viagens
de taxis feitas em um dia na cidade de Nova Iorque. Sua visualização é baseada
em pontos que destacam as origens e destinos das viagens com cores diferentes.
O resultado é similar a um mapa de densidades que mostra a distribuição das
viagens sobre a cidade. Eles também implementam uma estratégia robusta de
organização dos dados na memória para permitir que os usuários façam buscas e
apliquem filtros em uma grande quantidade de dados de maneira interativa. 

\citet{Anita2017} apresenta uma nova técnica de \emph{bundling}, resultado de
uma otimização do algoritmo FDEB, criado por \citet{Selassie2011}. Para isso,
utilizam uma etapa de clusterização aplicada previamente nos dados, e
posteriormente aplicam o \emph{bundling} em cada cluster. A vantagem da sua
abordagem é que ela pode ser utilizada em conjunto com outros métodos de
\emph{bundling}. \citet{Kim2018} utilizam campos de tensores para construir
um mapa de fluxo de dados geoespaciais analisando apenas as informações
estatísticas da distribuições dos pontos ao longo do tempo. Sua abordagem, no entanto
implica uma série de restrições estatísticas nos dados, como número mínimo 
de amostras e distribuição uniforme dos dados ao longo do tempo.

  Em nossa revisão, buscamos obter um panorama geral das lacunas que podem ser
preenchidas em relação ao uso de \emph{bundling} na visualização multi escala
de veículos no trânsito. Observamos vários aspectos de cada trabalho, como
o uso de recursos para visualização de padrões globais e locais do tráfego, o tipo de
objeto do tráfego analisado e tamanho do conjunto de dados, atentando-nos com interesse
especial em dados do trânsito e também o uso de \emph{bundling}. Concluímos que os dados de
trânsito ainda são pouco explorados, principalmente em trabalhos com
\emph{bundling}. A Tabela \ref{table:trabalhos} mostra os trabalhos que avaliamos
quanto ao uso de dados do trânsito, \emph{bundling} e o leiaute da visualização.

%O maior problema que observamos é a variedade de trabalhos que utilizam
%técnicas e conjuntos de dados diferentes sem comparar de fato os benefícios que
%eles trazem aos usuários finais, confirmando o que diz
%\cite{Andrienko2017Visual}, que os trabalhos de visualização são muito
%distantes do grupo de transportes e que as soluções nem sempre endereçam
%problemas reais. Seria interessante fazer experimentos comparativos de cada uma
%dessas visualizações que possuem objetivos em comum na análise dos fluxos de
%origem-destino na cidade. Desenhar tal experimento, no entanto é um grande
%desafio dadas as muitas variáveis subjetivas. Muitas delas, no entanto, abordam
%vantagens do ponto de vista computacional, que poderiam ser melhor verificados
%no que tange a aspectos de escalabilidade e flexibilidade quanto ao tipo de
%dados de tráfego suportado.

\begin{table}[htb!]
\begin{tabular}{|M{0.58}|M{0.11}|M{0.115}|M{0.09}|}
\hline
\textbf{Trabalho}       & \textbf{Dados do Trânsito} & \textbf{\emph{Bundling}} & \textbf{Leiaute}  \\ \hline
Nossa proposta          & \checkmark                 & \checkmark               &          linhas   \\ \hline
\citet{Kim2018}         & x                          &  x                       &          linhas   \\ \hline
\citet{Andrienko2017}   & \checkmark                 &  x                       &          radial   \\ \hline
\citet{Anita2017}       & x                          & \checkmark               &          linhas   \\ \hline
\citet{Landersberg2016} & x                          &  x                       &          linhas   \\ \hline
\citet{Klein2014}       & x                          & \checkmark               &          linhas   \\ \hline
\citet{Ferreira2013}    & \checkmark                 &  x                       &          pontos   \\ \hline
\citet{Zeng2013}        & x                          & \checkmark               &          radial   \\ \hline

\end{tabular}
\caption{Análise dos trabalhos relacionados quanto ao uso de dados do trânsito, uso de \emph{bundling} e leiaute da visualização. \label{table:trabalhos}}
\end{table}

