\chapter{Metodologia}
\label{cap:metodologia}
 Este capítulo descreve a nossa proposta de visualização para exploração dos
fluxos no tráfego de veículos. O objetivo principal da visualização é centrado
nas propriedades espaciais e temporais dos dados para entender como são os
fluxos de origem e destino pela cidade ao longo do tempo. Buscamos um maior
nível de detalhes através de uma abordagem multinível para explorar padrões
globais e locais do tráfego, com diferentes níveis de agregação temporais e
espaciais.

 Apresentamos inicialmente a ferramenta que iremos utilizar para responder às
questões de pesquisa levantas no Capítulo \ref{cap:introducao}, e que
relembramos a seguir. Damos uma visão geral dos seus componentes e dos
artefatos de entrada e saída que irão resultar na visualização e possibilitar a
análise dos dados. A ferramenta faz parte do ecossistema de soluções para
cidades inteligentes que são desenvolvidas no contexto do projeto
InterSCity\footnote{\rurl{interscity.org}}. Em seguida detalhamos os conjuntos
de dados que iremos utilizar e suas características, e por fim, explicamos as
técnicas que pretendemos aplicar para destacar as propriedades dos dados dentro
de uma visualização para responder às nossas questões:

Como podemos oferecer uma visualização de grandes
quantidades de dados de mobilidade de uma região
metropolitana:

\begin{itemize}
  \item \textbf{Q1:} Como podemos oferecer uma visualização de grandes quantidades
de dados de mobilidade de uma região metropolitana?

\end{itemize}


\section{Conjuntos de Dados}

Falar das ODS, falar quais ODS vamos utilizar


\subsection{Processamento dos dados}

 - Expansão das viagens
 - Limpeza
 - Filtragem das features de interesse
 - Features derivadas
 - Transformação dos dados para o formato do CUBU

\section{Visualização}

  Explorar todas as possibilidades seria um trabalho extensivo. Focamos
então em algumas classes de problemas dentro da visualização de dados
com bundling para explorar o seu potencial na análise de fluxos de origem e
destino na cidade.

\subsection{Agregação Espacial dos Dados}

1) Escalar o agrupamento conforme o zoom e resolução diferentes e dados diferentes (tunning não muito óbvio)

	- Número de pontos

	- Resolução da imagem

	- tamanho do kernel

	- Glanuralidade dos dados

  \begin{itemize}
    \item x) Aplicar bundling com alta resolução e alto numero de pontos, kernel grande e kernel pequeno
    \item x) Aplicar bundling com alta resolução e baixo numero de pontos, kernel grande e kernel pequeno
    \item x) Aplicar bundling com baixa resolução e alto numero de pontos, kernel grande e kernel pequeno
    \item x) Agrupar os dados na mão em regiões e aplicar o bundling em macro regiões. Qual a alguma diferença?
    \item x) Mostrar o bundling dos dados dos centróides e dos dados específicos em alguns modais (metrô, a pé, bike, taxi)
  \end{itemize}

	- Explicar que não há uma métrica, mas dissertar sobre qual fica melhor visualmente

	- Vantagens e desvantagens em relação às visualizações da OD em diferentes escalas

\subsection{Explorando outras propriedades}

- Aplicar o bundling para ver regiões densas, conforme mostrados na OD

  \begin{itemize}
    \item x) Mostrar as features de cores padrão (distância, direção, densidade)
    \item x) Mostrar as features de cores implementadas (tipo de transporte, modo de transporte, renda, gênero)
    \item x) Mostrar que tem pessoas que andam de metrô em locais que não tem metrõ (adicionar mapa)
    \item x) Mostrar como os ônibus metropolitanos se interceptam com os de são paulo
    \item x) Mostrar que os filtros de dados por horário pra mostrar que há mais fluxo
       no horário de pico, que os onibus escolares saem todos pela manhã, mas que também temos
      pessoas que saem de madrugada para trabalhar
    \item x) Mostrar a renda e correlacionar distância do centro com renda
    \item x) Mostrar correlação de renda com carro (pobre anda de carro?)
    \item x) Derivar um atributo do quão o horário de pico afeta e usar como VALUE

    \item x) Quem é baixa renda e não vão ao destino do centro, ele fica no local dele ou vai para lugares absurdos? (Excluir o Centro)
    \item x) Tentar extrair alguma ideia do mestrado da Haydee sobre genero
    % Tentar mostrar umas duas ou 3 coisas para mostrar pra o pessoal da CET
  \end{itemize}

	- Quais vantagens e desvantagens em relação ao que existe na OD (substitui alguma coisa do relatório da OD?)

	- Qual o nível de detalhes conseguimos extrair alguma informação?

  
\subsection{Estratégias de bundling}

2) Mostrar a mesma grandeza de diferentes maneiras

- Grandezas: Tempo/Tipo de transporte/Direção
- Estratégias: Fazer bundle de tudo e aplicando filtro E cores vs separado,
diferenciar por cores no bundling de tudo, e filtrar as arestas no tempo (manhã
e tarde)

  \begin{itemize}
    \item x) Fazer bundling de todos os dados de transporte e filtrar por modo
    \item x) Fazer bundling de todos os dados de transporte e colorir por modo
    \item x) Fazer bundling dos modos de transporte separados
    \item x) Fazer bundling dos dados em direção opostas
  \end{itemize}

	- Vantagens e desvantagens em relação aos gráficos da OD?

	- Qual os níveis de detalhes conseguimos obter?


\subsection{Evolução Temporal ao Longo dos anos}

4) Evolução temporal. Desafios: Correspondência de regiões

  \begin{itemize}
    \item x) Aplicar bundling de tudo: filtrar por ano
    \item x) Aplicar bundling separado em cada ano
    \item x) Aplicar bundling na diferença de um ano pro outro
    \item x) Observar viagens de Taxi não convencional de um ano pro outro
    \item x) Observar viagens de linhas novas de metro de um ano pro outro
  \end{itemize}

	Que vantagens/diferenças consigo ver em relação aos relatórios da OD?
	- Falar de bundling estático e dinâmico, mas que é future work...


\section{Avaliação da Visualização} 
