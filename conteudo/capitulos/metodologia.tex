\chapter{Metodologia}
\label{cap:metodologia}

Para explorar a visualização dos fluxos de mobilidade urbana utilizamos em nosso
estudo dados públicos da pesquisa OD17. Estes dados representam milhões de viagens
que ocorrem diariamente em um dia normal de trabalho sobre a Região Metropolitana
de São Paulo (RMSP). Parametrizamos e adaptamos o \emph{CUBu} para explorar várias propriedades
desses dados com diferentes usos do \emph{bundling} em nossas visualizações. A seguir
detalhamos cada uma das etapas que seguimos.

\section{Representação dos Dados}

Os dados da pesquisa OD que usamos como entrada para o \emph{bundling} são
descritos por uma tabela com seis colunas: ID da viagem, modo de transporte,
horário de partida, horário de chegada, coordenadas de origem e coordenadas de
destino. Extraímos esses dados da OD17: O ID da viagem é o identificador obtido
diretamente do conjunto de dados OD17 e representa uma entrada única no conjunto
de dados. O modo de transporte é armazenado como um número inteiro no intervalo
de 1 a 17 e corresponde a cada um dos diferentes modos presentes na pesquisa OD.
Os horários de partida e chegada são armazenado como um número de ponto
flutuante no intervalo de [0.0 - 23:59]. Por último temos as coordenadas de
origem e destino, o principal atributo do conjunto de dados. Em se tratando de
coordenadas, diversos sistemas podem ser utilizados; em nossa pesquisa
transformamos todas as coordenadas para o sistema de latitude / longitude para a
correta geo-localização das viagens no mapa, que também utiliza este formato.

Um outro atributo importante é o fator de expansão de cada registro da OD (apresentado na Seção~\ref{sec:pesquisa-od}),
o qual é utilizado para extrapolação estatística da quantidade real de viagens.
Desta forma, dado o registro de uma viagem da pesquisa OD17 contendo um fator de expansão $E$, criamos $E$
cópias desse registro para utilizar como entrada da nossa visualização com \emph{bundling}. Isso produz um conjunto de dados completo de 42
milhões de viagens. Para todas essas viagens replicadas, mantemos seu ID original
obtido da pesquisa OD17 que a gerou. Dessa forma, podemos rastrear quais viagens
agrupadas correspondem a um registro da OD17. Além desses dados básicos que apresentamos, atributos
extras da pesquisa OD17 podem ser adicionados, como o motivo da viagem (ver
Seção~\ref{sec:pesquisa-od}) e também dados pessoais (idade, renda). A pesquisa
OD é bastante rica e pode produzir um conjunto de dados com mais de dez atributos por viagem.
A Tabela~\ref{table:data-input} mostra um exemplo das viagens geradas com duas
viagens replicadas a partir de um registro da OD com ID 50 e fator de expansão $E=2$.

% Using the siunitx package to align and round
% numbers (columns of type "S")
\begin{table}[!htb]
  \small
  \newcommand{\hdr}[1]{\bfseries#1}
  \sisetup{
    mode=text,
    round-mode=places,
    round-precision=6,
    table-figures-integer=2,
    table-figures-decimal=6,
    table-number-alignment=left,
    table-text-alignment=left,
    table-space-text-post=nn,
  }
  \centering
  \caption{Formato de dados utilizados como entrada para o \emph{bundling}.\label{table:data-input}}
  \begin{tabular}{>{\footnotesize}l>{\footnotesize}c>{\footnotesize}l>{\footnotesize}l>{\footnotesize}S>{\footnotesize}S>{\footnotesize}S>{\footnotesize}S}
    \toprule
    \multirow{2}[2]{*}{\hdr{ID}} & \multirow{2}[2]{*}{\hdr{Modo}} & \hdr{Saída} & \hdr{Chegada} & \multicolumn{2}{c}{\hdr{Origem}}   & \multicolumn{2}{c}{\hdr{Destino}}\\
     &  & \hdr{hora} & \hdr{hora} & \hdr{Longitude}   & \hdr{Latitude}    & \hdr{Longitude}     & \hdr{Latitude}\\
    \midrule
    50         & 1   & 6.45 & 7.10  & -46.62809376987491 & -23.551691865840347 & -47.00348104352116 & -23.39356328288028\\
    50         & 1   & 6.45 & 7.10  & -46.62809376987491 & -23.551691865840347 & -47.00348104352116 & -23.39356328288028\\
    51         & 4   & 8.30 & 9.05  & -47.00187231236886 & -23.39846860627696  & -47.00348104352116 & -23.39356328288028\\
    \bottomrule
  \end{tabular}
\end{table}

\section{Processamento dos dados}

\section{Filtragem e recorte dos dados}

\section{Parâmetros do \emph{bundling}}
  
\section{Melhorias na visualização}

