\chapter{Metodologia}
\label{cap:metodologia}
 Este capítulo descreve a nossa proposta de visualização para exploração dos
fluxos no tráfego de veículos. O objetivo principal da visualização é centrado
nas propriedades espaciais e temporais dos dados para entender como são os
fluxos de origem e destino pela cidade ao longo do tempo. Buscamos um maior
nível de detalhes através de uma abordagem multinível para explorar padrões
globais e locais do tráfego, com diferentes níveis de agregação temporais e
espaciais.

 Apresentamos inicialmente a ferramenta que iremos utilizar para responder às
questões de pesquisa levantas no Capítulo \ref{cap:introducao}, e que
relembramos a seguir. Damos uma visão geral dos seus componentes e dos
artefatos de entrada e saída que irão resultar na visualização e possibilitar a
análise dos dados. A ferramenta faz parte do ecossistema de soluções para
cidades inteligentes que são desenvolvidas no contexto do projeto
InterSCity\footnote{\rurl{interscity.org}}. Em seguida detalhamos os conjuntos
de dados que iremos utilizar e suas características, e por fim, explicamos as
técnicas que pretendemos aplicar para destacar as propriedades dos dados dentro
de uma visualização para responder às nossas questões:

\begin{itemize}
  \item \textbf{Q1:} Como o \emph{bundling} pode ser usado para identificar os
fluxos de origem e destino no trânsito em diferentes escalas?

  \item \textbf{Q2:} É possível utilizar o \emph{bundling}  para
identificar padrões de fluxos de origem e destino no trânsito?

  \item \textbf{Q3:} O \emph{bundling} é eficiente para gerar uma
visualização de uma grande quantidade de dados do trânsito?
\end{itemize}

\section{Implementação da Visualização}
  Poucos algoritmos de \emph{bundling} são implementados em bibliotecas de uso
aberto. A maioria dos trabalhos pesquisados não disponibilizam a fonte para as
suas ferramentas e implementação dos seus algoritmos, e ainda que os algoritmos
estivessem disponíveis seria necessário implementar a ferramenta para
visualizar os dados da forma como queremos. Optamos então por partir de uma
implementação própria para a visualização dos dados, apoiando-nos em
bibliotecas já existentes para exploração de dados geoespaciais.

  Apresentamos o InterSCityPlotter, uma ferramenta para análise de dados do
tráfego que permite a criação de múltiplas visualizações, como visualizações
estáticas (Figura \ref{fig:simulated-traffic}) e visualizações baseadas em
pontos pontos (Figura \ref{fig:rastro}).  O objetivo da ferramenta é dar
suporte a estudos com dados do trânsito gerados pelo InterSCSimulator e também
por outras fontes, com o propósito de abranger análises de dados geoespaciais
que representam deslocamentos de um ponto de origem até um ponto de destino. O
diagrama da Figura \ref{fig:interscityplotter} mostra uma visão geral dos
componentes da ferramenta os quais damos uma breve descrição em seguida.

\begin{figure}[!htb]
  \centering
  \includegraphics[width=1\textwidth]{../figuras/interscityplotter.pdf}
  \caption{Visão geral da arquitetura da ferramenta InterSCityPlotter.}
  \label{fig:interscityplotter}
\end{figure}

\begin{description}
\item[\emph{ProjectManager}:] a exploração de dados do tráfego começa com a
criação de um novo projeto de visualização a partir de um conjunto de dados de
entrada. Esse componente é responsável pela criação, exclusão e outras ações de
gerenciamento dos projetos. Para utilizar a ferramenta, é necessário criar um
novo projeto apontando o conjunto de dados em um formato CSV. Os dados irão
passar por algumas etapas de pré-processamento definidos no componente
\emph{Pipelines} para que estejam prontos para visualização junto com os
arquivos do projeto criado.

  \item[\emph{Pipelines}:] esse componente isola as etapas de pré-processamento
necessárias para uso dos dados nas visualizações, como por exemplo limpeza e
checagem de inconsistências, derivação de novos atributos, etc.  Outras ações
também são definidas nesse módulo, como por exemplo, uma análise inicial que
contabiliza o número de trajetórias, quantidade média de pontos por trajetória
e número total de pontos no conjunto de dados é feita assim que se cria um
projeto. Essas informações são salvas junto com os dados do projeto.

  \item[VisualizationLayers:] o componente \emph{VisualizationLayers} baseia-se
na biblioteca Geoplotlib, apresentada em \citet{Andrea2016}, que fornece
recursos para criação de visualizações de dados geoespaciais. Ela fornece todo
o arcabouço em baixo nível para renderização de imagens a partir desses dados e
possui também algumas visualizações pré-definidas para projetar pontos e linhas
em um mapa. Os recursos do Geoplotlib serão estendidos dentro do
InterSCityPlotter para a implementação de diferentes tipos de visualizações
(Layers) customizadas. A visualização com \emph{bundling} que propomos será
construída como uma nova \emph{layer}.

  \item[CLI:] o módulo CLI possui a lógica para interação com a ferramenta
através da linha de comando. Essa interação se resume atualmente a comandos de
gerenciamento dos projetos (criação, exclusão) e execução de uma das
visualizações já desenvolvidas. A documentação completa pode ser vista no
repositório da ferramenta, disponível no
Github\footnote{\rurl{github.com/tallysmartins/interscity-plotter}}.
\end{description}

  A ferramenta se encontra em processo de desenvolvimento. Os módulos
\emph{Project Manager}, \emph{Pipelines} e CLI já estão parcialmente
implementados e já é possível criar um novo projeto a partir de dados do
simulador, que posteriormente passam por um primeiro pré-processamento para
contabilizar algumas estatísticas, como número de veículos totais, média de
pontos por viagem, tempo total da simulação e outros. Ainda restam implementar
todo o módulo \emph{VisualizationLayers} e também alguns pré-processamentos do
módulo \emph{Pipelines} que iremos utilizar para tratamento dos dados de
entrada e derivação de novos atributos.

\section{Conjuntos de Dados}
   Os dados do tráfego que utilizaremos possuem o registro da posição e o tempo
ao longo de toda a trajetória, desde a origem até seu destino. Outros atributos
como velocidade, direção, aceleração, podem ser derivados a partir dos dados
brutos. Cada trajetória $T$ de um veículo pode ser modelada como uma sequência
de pontos $p_i$ que contém a sua posição, o instante de tempo e um vetor $a$
contendo outros $n$ possíveis atributos. Em nosso estudo, esse vetor guarda a
direção, dada pelo ângulo da reta entre os pontos de origem e destino de cada
trajetória.

\begin{center}
$T = \left\langle p_i = ((latitude, longitude) \in \mathbb{R}^2, t \in \mathbb{R}^+, a^n \in \mathbb{R})_i \right\rangle$
\end{center}

  Duas fontes de dados com informações do tráfego de veículos na cidade de São
Paulo serão usadas neste trabalho. A primeira é uma API pública disponibilizada
pela prefeitura da cidade e que fornece dados da movimentação dos ônibus. A
outra fonte é o simulador InterSCSimulator, o qual usaremos para obter dados
do tráfego.

\subsection{Tráfego dos Ônibus de São Paulo} São Paulo possui uma frota de
cerca de 15 mil ônibus que circulam diariamente nas vias da cidade e fazem mais
de 70 mil viagens por dia em mais de 2 mil linhas diferentes. A posição dos
ônibus durante seu percurso é registrada a cada 45 segundos com aparelhos de
GPS e disponibilizada pela Secretaria de Transporte Público
(SPTrans)\footnote{\rurl{www.sptrans.com.br/desenvolvedores/APIOlhoVivo/Documentacao.aspx?1}}
através de API pública. As buscas dos dados são feitas pelo código da linha,
que retorna naquele momento informações de todos os veículos daquela linha em
operação nas ruas, retornando o código do veículo, horário da coleta, posição
(latitude e longitude) e se o ônibus possui suporte para pessoas com
deficiência. A lista de todas as linhas existentes também podem ser adquiridas
pela API.  A Listagem \ref{olhovivo.json} mostra um exemplo de uma consulta na
API para a linha de ônibus 2023. Mais detalhes podem ser vistos na documentação
da API que fornece os dados.

\begin{lstlisting}[style=myxml, caption={Parte da resposta obitida para a linha 2023}, label=olhovivo.json]
{
 "hr"=>"14:25", // Hora da requisição
 "vs"=> [       // Lista de ônibus em circulaçao
   {
    "p"=>"82324", // Identificador do veículo
    "a"=>true,    // Acessível a deficientes
    "ta"=>"%*2019-01-30T16:25:05Z*)", // Hora da coleta dos dados em formato UTC
    "py"=>-23.562566125000004, // Latitude
    "px"=>-46.72267612500001   // Longitude
   }]
}
\end{lstlisting}

Para a visualização do tráfego serão coletados os dados de um dia do tráfego de
ônibus. Embora não representem a totalidade do tráfego, esses dados compõe uma
grande parte da movimentação que ocorre na cidade ao longo do dia e servem como
base para comparação com a visualização dos dados simuladas. Pretendemos
explorar em nossa análise os padrões de movimentação no trânsito, como regiões
de maior fluxo e como ele se comporta em diferentes momentos do dia, como
horários de pico.

\subsection{Tráfego Simulado de Veículos}
  O uso de dados simulados nos trazem dois aspectos importantes levantados em
nossas questões \textbf{Q2} e \textbf{Q3}. O primeiro é a possibilidade de
obtermos cenários diferentes do trânsito com alterações na simulação, já o
segundo aspecto é capacidade de obtermos uma grande quantidade de dados do
tráfego de veículos no trânsito.

  Utilizaremos uma simulação do tráfego de carros e ônibus na cidade de São
Paulo com cerca de 585 mil veículos. O cenário foi criado e disponibilizado por
\citet{santana2018courb} e gerou um arquivo de saída com mais de 31 milhões de
eventos de movimentação dos veículos. Para construir o arquivo de entrada
\emph{trips.xml} com as mais de 585 mil viagens, eles utilizaram uma matriz de
Origem-Destino (OD) feita pela Companhia do Metropolitano de São Paulo
(Metrô)\footnote{Pesquisa Origem-Destino - \rurl{goo.gl/DNM8in}} que pesquisou
como os cidadãos se locomovem pela cidade e usaram isso para determinar a
movimentação dos carros.  Além disso, utilizaram também dados dos itinerários
dos ônibus fornecidos pela SPTrans para estabelecer viagens de ônibus com
horários, paradas e frequência mais realistas.

  A pesquisa OD catalogou 160 mil amostras de viagens realizadas em um dia de
semana da cidade de São Paulo. Ela possui dados como a origem, o destino, a
hora de início, o modo de transporte e um fator a extrapolação estatística dos
dados usado para generalizar os valores da amostragem para toda a cidade.
Destas viagens, cerca de 26 mil são feitas de carro, como mostram
\citet{santana2018courb}. Utilizando o fator máximo de extrapolação da pesquisa
é possível chegar a uma quantia de mais de 4 milhões de viagens de carro e
ônibus em um dia, mas é possível realizar simulações menores reduzindo esse
fator. Posteriormente, fazemos alterações no arquivo \emph{map.xml} para
alterar o comportamento da simulação e obter um cenário diferente, que
comparamos com o cenário original. Detalhamos essas alterações na Seção
\ref{sec:vis-eventos}.

\subsection{Pré-processamento dos dados}

  Os dados coletados estão sujeitos a erros, e por isso, algumas etapas de
pré-processamento são necessárias. O módulo de \emph{Pipelines} da ferramenta
InterSCityPlotter é o local destinado a armazenar as rotinas que fazem essas
manipulações dos dados, tanto de limpeza quanto de derivação de novos
atributos. Mapeamos as seguintes rotinas para tratamento dos dados:

\begin{description}
  \item[Viagens da simulação não finalizadas:] isso pode ocorrer
quando o tempo da simulação acabado e uma viagens ainda está no meio do percurso,
essas viagens devem ser identificadas e removidas da análise.

  \item[Dados inconsistentes com a trajetória:] os dados dos ônibus fornecidos
pela prefeitura podem conter inconsistências devido a erros nos sensores e/ou na transmissão,
como por exemplo, pontos fora das vias. Esses dados devem ser identificados e removidos da análise.

  \item[Rotas circulares:] alguns ônibus têm rotas circulares, ou seja, o ponto
de partida e de chegada são o mesmo. Deve-se observar esses casos e seus efeitos
na visualização, para se necessário, removê-los.

  \item[Anomalias:] podem ocorrer que algumas trajetórias apresentem uma distância
muito pequena, ou pontos esparsos. Essas possíveis anomalias devem ser identificadas
e removidas da análise.

  \item[Derivação da direção:] esse atributo será derivado a partir dos pontos
de origem e destino de cada trajetória e será dado por um ângulo $\theta$ no
plano cartesiano dado pelas latitudes e longitudes do globo.
\end{description}

  Outras necessidades de pré-processamento podem surgir ao longo da pesquisa.
Estes são apenas itens já identificados até o momento, e que receberão atenção
durante o processo de implementação da visualização.

\section{Visualização do Tráfego de Veículos}

  Apresentamos a seguir as técnicas para exploração das propriedades temporais
e espaciais dos dados em diferentes níveis de detalhes. Algumas técnicas se
inspiram em soluções apresentadas em outros trabalhos relacionados e algumas
outras surgiram a partir das necessidades desta pesquisa.

\subsection{Leiaute da Visualização}
  
  Utilizaremos em nossa proposta uma projeção das trajetórias em formato de
linhas sobre um mapa.  Como apresentamos na Seção \ref{sec:prop-espaciais},
esse tipo de leiaute espacial mostra de maneira mais intuitiva as relações de
localidade dos dados. Por outro lado, as propriedades temporais são também de
grande importância em nossa análise dos fluxos, sendo um aspecto que
possibilita identificar mudanças e eventos que alteram o comportamento do
trânsito durante um certo período.  Neste sentido, permitimos controle da
dimensão temporal através de dois parâmetros, $t_{atual}$ e $\Delta~t$, similar
ao definido por \citet{Klein2014}.  A combinação dos parâmetros definem um
filtro que funciona como um referência para selecionar um conjunto de
trajetórias  ${T}$ que contenham pontos dentro do intervalo $t_{atual} +
\Delta~t$, determinando uma janela de tempo. Valores de $\Delta~t$ pequenos ou
nulos acabam por selecionar os pontos instantâneos das trajetórias no instante
de tempo definido por $t_{atual}$, enquanto valores maiores trazem os demais
pontos ao longo do tempo. Com isso conseguimos um alcance multiescala dentro
da dimensão temporal, permitindo a seleção de momentos instantâneos e também de
um conjunto de pontos ao longo de um grande intervalo.

\subsection{Agregação Espacial com \emph{Bundling}}

Para os problemas de oclusão na visualização de uma grande quantidade de dados
do tráfego criaremos uma abstração com o uso do \emph{bundling} para agregar
trajetórias com origem-destino e direção similares. Com isso, destacamos a
estrutura geral dos fluxos para que seja possível identificar os padrões e
relações de movimentação presentes nos  dados. Essa abstração será implementada
dentro do componente \emph{VisualizationLayers} na ferramenta InterSCityPlotter
com o algoritmo \emph{Attribut-Driven Edge Bundling} (ADEB), apresentado na
Seção \ref{sec:modelo-imagem}.

 A visualização com \emph{bundling} acaba por aglomerar as trajetórias em áreas
de maior densidade, o que nos dá perspectiva dos fluxos de veículos que saem e
chegam de cada região da cidade, identificando também as áreas com uma
quantidade maior de veículos em um dado instante. A combinação dessa técnica
com o parâmetro temporal apresentado na seção anterior possibilita obter essa
agregação com \emph{bundling} em diferentes períodos do dia, como em horários
de pico, para que possamos investigar sua estrutura e seu comportamento ao
longo do tempo. Na escala da cidade, as linhas dos \emph{bundles} podem
demonstrar ainda viagens de longa distância ou desvios realizados por alguns
veículos, sugerindo melhorias no projeto da rede rodoviária.

\subsection{Agregação Espacial Multinível}

  A dimensão espacial dos dados é o nosso segundo parâmetro multiescala dentro
da visualização. Para observar os padrões de OD em diferentes níveis de detalhe
espaciais propomos fazer agrupamentos baseados na escala da visualização, que
está relacionada ao tamanho da área do mapa mostrada na imagem. Este mecanismo
é comumente visto em visualizações interativas que permitem aos usuários mudar
as configurações de zoom no mapa. A partir daí, propomos o conceito de
sub-trajetória, que são as partes de uma trajetória dentro da sub área do mapa
após um aumento do zoom. Os pontos dentro do escopo da sub área do mapa formam
um subconjunto das trajetórias originais. Os atributos de direção, origens e
destinos do subconjunto se tornam relativos em relação ao novo escopo. Em
seguida é possível reaplicar o \emph{bundling} no subconjunto. Quanto menor a
escala, mais detalhes sobre o fluxo do tráfego são obtidos, sendo possível
observar bairros, quadras ou até mesmo ruas. A Figura \ref{fig:multi-scale}
ilustra os diferentes níveis de detalhe atingidos em escalas diferentes na
visualização de um mapa.

\begin{figure}[!htb]
  \centering
  \includegraphics[width=55mm]{../figuras/multi-scale.png}
  \caption[Visualização de um mapa em diferentes escalas espaciais]{Visualização de um mapa em diferentes escalas espaciais. Fonte: \citet{Zeng2013}}
  \label{fig:multi-scale}
\end{figure}

\textbf{Parâmetros:} o algoritmo ADEB possui alguns parâmetros que
influenciam o resultado \emph{bundling}. Esses parâmetros são discutidos pelos
autores do método, e valores padrão são indicados por eles. Seguiremos
inicialmente os valores indicados, mas atentos a detalhes ou modificações que
se mostrarem necessárias. Outro parâmetro importante em nossa visualização é a
escala espacial para variação do nível de detalhes do \emph{bundling}. Esse
parâmetro age como um filtro para selecionarmos um subconjunto das trajetórias
dentro de uma sub área visualizada. Exploraremos esses parâmetros apresentando
discussões sobre seus efeitos nos resultados da visualização.

\subsection{Visualização dos Atributos}
Depois de calcular o \emph{bundling} das trajetórias, é necessário projetar uma
maneira de mostrar a estrutura dos elementos e seus atributos. Em nosso estudo
dos fluxos de OD consideramos dois atributos chave, densidade e direção dos
fluxos, os quais destacamos diretamente na visualização.

\textbf{Densidade}: a densidade é uma informação importante na análise do
fluxo. Essa informação nos ajuda a responder ``Qual a quantidade de veículos se
locomovem de uma região para outra?''. Uma forma de destacar esse atributo é
usando uma escala de cores. A Figura \ref{fig:density-mappings} mostra o
mapeamento da densidade com escalas de cores feita de duas formas, (a) ilustra
uma escala de opacidade, que funciona somando-se os valores de opacidade nos
pontos de sobreposição e (b) mostra uma escala gradiente multicores. Uma outra
forma comum de se mostrar esse atributo é simplesmente escalando a espessura
das linhas dos \emph{bundles} proporcionalmente à densidade.

\textbf{Direção}: a direção é outra informação relevante para o estudo dos
fluxos.  No algoritmo de \emph{bundling} ADEB, ela é indicada pelo ângulo da
trajetória em relação a um eixo cartesiano. Para destacar esse atributo na
visualização utilizaremos um mapa de ângulo-para-cor, como em
\citet{ZegarraFlores2016}. A Figura \ref{fig:direction} ilustra como seria o
destaque da direção e também da densidade no \emph{bundling}, note que o círculo na
legenda indica o mapa de cores. Trajetórias saindo de leste para oeste estão em
azul, norte para sul são representadas em verde, e assim por diante. No caso de
atributos numéricos, como os ângulos da direção, os valores que se sobrepõe
podem ser somados para dar um resultado final, ou simplesmente sobrescritos
pela trajetória desenhada na camada mais superior.

\begin{figure}[ht!]
  \centering
  \begin{subfigure}[t]{\textwidth}
    \centering
    \includegraphics[width=.8\textwidth]{../figuras/alpha-blending.pdf}
  \end{subfigure}
  ~
  \begin{subfigure}[t]{\textwidth}
    \centering
    \includegraphics[width=.8\textwidth]{../figuras/color-density.pdf}
  \end{subfigure}

  \caption[Densidade dos \emph{bundlings} mapeada em escalas de cores e
  opacidade.]{Formas de visualização da densidade. (a) Escala de opacidades, conhecida como combinação do canal alfa. (b) Escala de cores gradiente.  Fonte: \citet{Lhuillier2017} \label{fig:density-mappings}}
\end{figure}


\begin{figure}[ht!]
  \centering
  \includegraphics[width=.8\textwidth]{../figuras/color-wheel.pdf}
  \caption[Direção dos \emph{bundles} mapeada em escalas de cores]{Direção dos \emph{bundles} mapeada em escala de cores. Fonte: \citet{Lhuillier2017}. \label{fig:direction}}
\end{figure}

\subsection{Visualização Interativa}

Mecanismos de interação são um excelente recurso em uma visualização de dados.
Mais do que apenas informar, os usuários podem buscar seus próprios padrões e
explorar o conteúdo à sua maneira. Adicionaremos alguns meios de interação para
facilitar o estudo das trajetórias e exploração das suas propriedades.

\textbf{Parâmetros:} Todos os parâmetros mencionados anteriormente para a exploração
multinível no tempo e no espaço serão configuráveis para os usuários da visualização,
bem como os parâmetros do algoritmo de \emph{bundling}. Desta maneira ganha-se
a liberdade para seleção dos parâmetros que melhor se adequam aos objetivos
da visualização.

\textbf{Inspecionando os \emph{bundles}}: analisando-se um \emph{bundle},
podemos dar ainda mais detalhes sobre as trajetórias que o formam para
responder duas questões importantes para análise dos fluxos: ``Qual o percentual
das trajetórias veem de uma determinada região?'' ou ``Qual região é maior
responsável pelo fluxo naquele trecho?''.  Essa informação pode ser obtida
observando-se a composição do \emph{bundle} e oferece mais detalhes para que
gestores da cidade entendam melhor a estrutura do trânsito e as relações de OD.

 Para inserir esse dado na visualização recorremos a uma nova
imagem, disparada pela interação do usuário que poderá inspecionar um
\emph{bundle} para visualizar esses detalhes em uma caixa de diálogo com as
informações sobre as origens das trajetórias.

\section{Visualização de Eventos Atípicos}
\label{sec:vis-eventos}

  Outra importante tarefa na análise do trânsito é o estudo de como a
ocorrência de eventos atípicos impactam no tráfego. Para essa tarefa, propomos
a alteração na simulação do tráfego para obter um novo cenário do trânsito com
eventos controlados, e com isso observaremos como a visualização com \emph{bundling}
é impactada por esses eventos e como ela pode ajudar a identificá-los.

  Usaremos como base a simulação disponibilizada  por \citet{mabs2017} para
simular o bloqueio de vias importantes do tráfego. Para isso, iremos alterar o
arquivo \emph{maps.xml}, que descreve a estrutura da cidade, e excluiremos as
entradas de algumas vias que possuem maior quantidade de fluxos, e então
executaremos uma nova simulação com esse novo cenário. Nesse caso, as
definições dos veículos definidas no arquivo \emph{trips.xml} ainda serão as
mesmas, e esta é uma forma de forçamos os veículos a tomarem rotas alternativas
para chegarem a seus destinos.

 Para selecionar quais ruas serão excluídas, iremos contabilizar o
número de veículos que passam por cada rua e excluiremos as 5 ruas com maior
quantidade de tráfego. Partimos do pressuposto de que essas mudanças irão
causar algum efeito na simulação, pois estamos retirando totalmente 5
importantes arestas do grafo da cidade.

  Esperamos que o \emph{bundling} revele as alterações nos padrões de
deslocamento dos veículos e que esses sejam visualmente perceptíveis. Avaliar
esse tipo de cenário é importante para demonstrar como a técnica pode ajudar em
questões do dia a dia no trânsito, como a ocorrência de acidentes, alagamentos
e mudanças na rede rodoviária da cidade.

\section{Avaliação da Visualização} 
  A avaliação do \emph{bundling} ainda é um desafio em aberto
\citep{Telea2018}.  Ainda não se tem um consenso sobre métodos para avaliar a
qualidade de um \emph{bundling}, ou mesmo critérios objetivos para definir o
que é um ``bom'' \emph{bundling} \citep{Lhuillier2017}.  A maioria dos
trabalhos que identificamos fazem apenas uma comparação visual com outros
trabalhos. Nesse sentido, nós avaliaremos a visualização de dois pontos de
vista, um qualitativo, onde avaliaremos os resultados da nossa visualização
na tarefa de identificação dos fluxos de origem e destino na cidade e seus atributos
de direção e densidade, e um quantitativo, medindo o desempenho computacional
da nossa solução.

  A questão de pesquisa \textbf{Q1}, relacionada ao uso do \emph{bundling}
para a análise multiescala, é nosso primeiro objeto de avaliação qualitativa.
Para trazer artefatos que nos ajudem a responder essa pergunta iremos trabalhar
diretamente com a variação dos parâmetros da nossa visualização, tais quais a
escala, os parâmetros temporais e também a forma como apresentamos os atributos
de densidade e direção dos fluxos. Faremos então uma observação objetiva sobre
os diferentes resultados gerados com a variação dos parâmetros, suas qualidades
e limitações, para por exemplo, avaliar se conseguimos identificar
características, origens e destinos dos fluxos para responder a perguntas como:
``Quais regiões recebem maior quantidade de fluxo?'' ou ``Qual região é maior
responsável pelo fluxo em uma determinada via?''

  Já a questão \textbf{Q2} busca responder como a técnica de \emph{bundling}
ajuda na identificação de padrões ou mudanças no trânsito. Nossa abordagem para
responder a essa pergunta baseia-se nos parâmetros de controle do tempo e
também no experimento apresentado na seção anterior. O parâmetro temporal irá
permitir que vejamos o trânsito em diferentes momentos do dia de forma a
observar como muda seu comportamento, por exemplo, em horários de pico. Já no
experimento citado, propomos o bloqueio controlado de vias dentro da simulação
para observarmos, também de forma qualitativa, como esse evento afeta a
movimentação no trânsito sobre a ótica da nossa visualização.

  Para completar a avaliação da nossa proposta iremos medir quantitativamente o
desempenho da nossa solução dado o tempo de processamento para gerar a
visualização, o que está relacionado à questão \textbf{Q3}. As medições serão
feitas desprezando as etapas de pré-processamento e carregamento dos dados.
Essa análise é considerada relevante, pois contamos com uma grande quantidade
de dados obtidos com o InterSCSimulator. Os dados possuem de 31 milhões de
pontos distribuídas em $585.374$ viagens feitas por veículos no trânsito, o que
é mais de duas vezes o número de pontos nos dados de um mês do tráfego aéreo
utilizados na análise de \citet{Klein2014}.

 Uma maneira de aprofundar esta etapa de validação é através de uma comparação
mais detalhada sobre duas visualizações diferentes aplicadas sobre os mesmos
dados. Iremos estudar a viabilidade em fazer esse tipo de comparação com
algum dos trabalhos que também fazem análises dos fluxos de origem e destino no
tráfego listados no Capítulo \ref{cap:trabalhos-relacionados}.
