%!TeX root=../tese.tex
%("dica" para o editor de texto: este arquivo é parte de um documento maior)
% para saber mais: https://tex.stackexchange.com/q/78101/183146

% Apague as duas linhas abaixo (elas servem apenas para gerar um
% aviso no arquivo PDF quando não há nenhum dado a imprimir) e
% insira aqui o conteúdo dos capítulos do seu trabalho
\newcommand\rurl[1]{%
  \href{http://#1}{\nolinkurl{#1}}%
}

\lstdefinestyle{myxml}{
  numbers=none,
  xleftmargin=5pt,
  framexleftmargin=10pt,
  escapeinside={\%*}{*)},
  inputencoding=utf8
}

%% ------------------------------------------------------------------------- %%
\chapter{Introdução}
\label{cap:introducao}

O rápido  desenvolvimento de sistemas de transporte e crescimento das cidades
tem sido acompanhado de fluxos de deslocamento cada vez mais complexos e
volumosos, aumentando também os desafios para as questões de mobilidade urbana. 
\cite{Zhang2011} mostram em seu estudo uma estimativa de que cerca de 40\% da população mundial passa pelo menos uma
hora no trânsito todos os dias \citep{Zhang2011}. Os impactos negativos de
uma infraestrutura de transporte ineficiente também geram perdas econômicas,
como mostra um estudo de X, que calcula que
os gargalos em mobilidade no Brasil geram prejuízos de R\$ xxxx milhões de
reais anualmente. Tais dados mostram como ainda existe espaço para se buscar
melhorias para os sistemas de transporte, de maneira
a melhorar a qualidade de vida dos cidadãos e reduzir custos.

Para isso, a tecnologia tem sido amplamente empregada como um agente
facilitador de coleta e disponibilização de várias fontes de informação que
permitem cidadãos, pesquisadores, empresas e agentes públicos a realizarem
estudos para entender problemas de mobilidade urbana e buscarem soluções.
Várias tipos de dados podem ser coletados de diferentes fontes, como sensores
de GPS, câmeras de tráfego, e mesmo coletados manualmente. Na administração
pública, essas fontes de informação podem ser utilizadas para prover serviços
melhores para a população, melhorar a gestão da infraestrutura urbana e dar
suporte para a tomada de decisão baseada em evidências. Dentro desse
contexto, nós focamos nosso estudo em dados de mobilidade urbana da região
metropolitana de S\~ao Paulo (RMSP).

A cada dez anos, desde 1967, a Companhia Metropolitano de São Paulo (Metrô),
que gerencia o sistema de metrô da cidade, conduz um estudo censitário de
mobilidade chamado \emph{Pesquisa Origem-Destino} (OD). A pesquisa OD é feita
através de entrevistas aos cidadãos e coleta dados sobre suas vidas e
atividades de deslocamento em um dia típico de trabalho. O resultado é um amplo
panorama social e de mobilidade da população na RMSP. A última pesquisa OD
(2017) mostra que ocorrem cerca de 42 milhões de deslocamentos nas 24 horas de
um dia comum de trabalho. Além das informações de origem (O) e destino (D), a
pesquisa coleta informações socioeconômicas dos cidadãos e também outros dados
sobre os deslocamentos, como o modo de transporte utilizado, o motivo da
viagem, idade, gênero, renda familiar. Com isso, a pesquisa OD se traduz num
grande conjunto de dados com múltiplos atributos.

Embora a pesquisa OD mencionada seja bem abrangente e precisa, 
os gestores precisam de ferramentas adequadas para analisar esta grande quantidade de
dados multivariados. O uso de planilhas e gráficos com informações agregadas
ajudam a compreendê-los, mostrando, por exemplo, o número de usuários 
do sistema de transporte público ao longo dos anos, ou o número de homens ${X}$ mulheres
que saem diariamente a trabalho. Técnicas de visualização, como mapas de densidade,
podem ajudar a responder questões geoespaciais, como encontrar regiões que concentram
os maiores fluxos de mobilidade durante o dia ou quais os pares de origem e destino
mais comuns. No entando, considerando a grande quantidade de dados (e atributos) gerados
nas cidades, traduzir dados geolocalizados em imagens significativas não é uma tarefa simples.

Um estudo recente sobre métodos de visualização de fluxos de mobilidade indica um
uso comun de visualizações baseadas em linhas para estudar a estrutura desses dados,
o que geralmente inclui descoberta de padrões e agrupamentos \cite{Chen2015}. Cores e texturas
podem ser utilizados para dar informações adicionais sobre os objetos estudados, como sua velocidade, direção,
meio de transporte utilizado, etc. Porém, desenhar diretamente linhas das trajetórias em um mapa
torna-se pouco efetivo em cenários com uma grande quantidade de dados. Uma visualização das 42 milhões de trajetórias,
como os registrados pela última pesquisa OD, resulta em uma imagem completamente ofuscada
pela sobreposição e cruzamento dos fluxos, como mostramos mais adiante na Figura \cite{fig:cluttered-graph}.
Para isso, técnicas de \emph{bundling} aprimoram as visualizações baseadas em linha e podem
representar grandes volumes de dados de trajetórias.

Neste trabalho nos propomos o uso de um conjunto de técnicas de visualização
para explorar as características de uma grande quantidade de dados de
mobilidade urbana. Para isso, nós adaptamos uma técnica que revelou resultados
interessantes na visualização de trajetórias em outros cenários como o do
tráfego aéreo e rastreamento do movimento da visão para o nosso contexto.
Nosso estudo permitiu a visualização dos dados em diferentes níveis de
granularidade espaço temporal, ajudando a encontrar padrões de mobilidade e
revelar novos insights a respeito do sistema de transporte de S\~ao Paulo e da
infraestrutura da cidade. Em contraste a outros trabalhos que utilizam técnicas
similares, nossa pesquisa ainda se destaca pela grande quantidade de dados
analisada. Para isso, apresentamos também uma metodologia para reduzir o tamanho
do conjunto de dados com impacto mínimo na sua significância estatística.

.....continua
\par
%Falar sobre visualização de dados. Visualização de dados de origem e destino.
%Multi atributos. Multiplas origem e destinos. Representação geoespacial.
%Multi atributos. Propomos uma visualização que utiliza bundling, cores e filtros
%para explorar dados geoespaciais em uma grande cidade. Exploramos a visualização
%em vários aspectos desse cenário complexo (correlação, filtragem de dados, cores)
%e mostramos como ela se comporta para mostrar características dos dados de
%origem-destino, além de parâmetros e peculiaridades do modelo de bundling escolhido.
%Por exemplo, mostramos regiões contém maior fluxo filtrando por
%diferentes modais de transporte. Mostramos a correlação entre renda e modais
%para mostrar que pobre é quem anda a pé e mora longe. Além de calcular como
%é o fluxo no tempo utilizando cores, filtros e subtração de dados para observar
%a evolução ao longo do dia e ao longo dos anos. Apresentamos também os aspectos
%de escalabilidade computacional da técnica, dissertando sobre a complexidade
%do algoritmo nos cenários medindo ainda os tempos de execução.

\chapter{Referencial teórico}
\label{cap:referencial-teorico}

  Neste capítulo esclarecemos os conceitos teóricos necessários para o
entendimento do trabalho e a metodologia proposta. Inicialmente, abordamos
aspectos da visualização de dados do tráfego, que abrange uma variedade de
cenários conforme os objetivos, tipos de dado e diferentes representações
visuais da informação. Neste contexto, destacamos o \emph{bundling}, uma
técnica bastante usada na visualização de grandes quantidades de dados do
tráfego. Apresentamos os detalhes da técnica e uma breve consideração sobre os
avanços recentes nesta área. Por fim, considerando que neste trabalho faremos o
uso de dados simulados do tráfego de veículos, apresentamos o simulador
InterSCSimulator.

\section{Visualização de dados do tráfego}

  A análise de dados do tráfego tem um histórico antigo. \citet{Chen2015} fazem
um levantamento de várias pesquisas de análise e visualização de dados do
tráfego de pessoas, carros e embarcações. Eles apresentam uma taxonomia
conforme algumas características observadas, como o objetivo da visualização,
formato dos dados do tráfego e formas de apresentação de suas propriedades.
Nesta seção apresentamos essa taxonomia, que é importante para um melhor
entendimento da área e dos esforços que têm sido feitos no campo da
visualização de dados do tráfego.

\subsection{Quanto ao Objetivo da Visualização}

  Segundo \citet{Chen2015}, os trabalhos e sistemas de software sobre análise e
visualização de dados do tráfego podem ser classificadas em quatro categorias
de acordo com seus objetivos e a tarefa a que servem.

\begin{description}
  \item[Monitoramento do tráfego:] esse tipo de aplicação
foca no monitoramento em tempo real para descobertas instantâneas de eventos no
tráfego, como câmeras de gravação ao vivo e sistemas de alerta.

  \item[Descoberta de padrões e clusterização:] para a descoberta de padrões e
tendências de mobilidade, algumas aplicações utilizam métodos para o
agrupamento visual de trajetórias ou novas abstrações dos dados.

  \item[Exploração e predição:] outras aplicações focam em fornecer mecanismos
de pesquisa e exploração dos dados para que os usuários investiguem os dados
que justifiquem situações do tráfego, como congestionamentos, ou ainda técnicas
de predição para prever essas situações (e.g. predizer que haverá
congestionamento devido a ocorrência de chuva).

  \item[Planejamento de rotas e recomendação:] sistemas de transporte
inteligentes contam essencialmente com recomendações de trajetos para usuários
do transporte, várias soluções existem com esse propósito, que podem envolver monitoramento
em tempo real ou análise histórica no processo de recomendação.
\end{description}

\subsection{Quanto ao Tipo de Coleta dos Dados}

  Dados do tráfego são tipicamente coletados a partir de sensores e dispositivos
eletrônicos. A estrutura desses conjuntos de dados dependem do modo de operação
dos sensores e dispositivos que registram a movimentação dos objetos. \citet{Chen2015}
os dividem em três classes:

\begin{description}
  \item[Baseado em pontos de interesse:] a posição de um objeto é gravada assim
que ele entra na área do sensor. Como uma câmera de vídeo que capta a
movimentação e orientação de um veículo assim que ele passa pela área de
monitoramento.

  \item[Baseado em ações:] as informações sobre um objeto podem estar associados
a certas atividades. O usuário de um aparelho celular por exemplo, tem sua
atividade e localização registrados pela rede GSM quando o mesmo efetua uma
chamada.

  \item[Baseado em sinais de dispositivos:] Um dispositivo de localização
carregado por um objeto constantemente grava e envia suas informações de
localização para uma central, como aparelhos de GPS instalados em um ônibus do
transporte público de uma cidade.  \end{description}

  Uma série de posições de localização ao longo do tempo formam a trajetória de
um objeto. Uma trajetória contém informações temporais, que permitem traçar uma
linha do tempo da movimentação, e informações espaciais, que contém a posição
em cada momento no tempo. A frequência de armazenamento dessas informações
também impacta a análise. Capturar e armazenar essas informações com um
intervalo de tempo muito pequeno pode ser algo bastante custoso. Outro fator de
impacto é a precisão das informações a qual depende de questões relacionadas ao
hardware utilizado na coleta. Muitas vezes é necessária uma etapa de
pré-processamento para extrair dados inconsistentes da análise.

\subsection {Visualização das Propriedades Temporais}
\label{sec:prop-temporais}

  Dados do tráfego podem conter uma série de variáveis, das quais as mais
importantes são o tempo e espaço. Visualizações orientadas ao tempo enfatizam
tendências, periodicidades e anomalias nos dados ao longo de um eixo temporal \citep{Aigner2008}.

  Expressar a variação do tráfego ao longo do tempo e sua evolução é um típico
processo de análise linear do tempo, que é considerado a partir de um ponto de
início até um ponto final \citep{Chen2015}. Por exemplo, em um gráfico de
linhas, o tempo é representado ao longo do eixo X e outra variável é
representada no eixo Y, como na Figura \ref{fig:linear-time}. Gráficos de
linhas são fáceis de interpretar, porém limitam-se na quantidade de variáveis
que podem ser visualizadas.

\begin{figure}[!htb]
  \centering
  \includegraphics[width=\textwidth]{../figuras/linear-time.png}
  \caption[Gráfico de linha representando empo linear]{Gráfico de linha representando tempo linear. Ele mostra a média de gorjetas por viagem em diferentes regiões no período de 1 a 7 de Maio de 2011. Cada linha representa uma região. Fonte: \citep{Ferreira2013}
  \label{fig:linear-time}}
\end{figure}

  Por outro lado, temos também alguns processos recorrentes que são naturais em
nosso mundo, os quais seguem ciclos de estações, ou mesmo ciclos semanais e
diários. Um leiaute radial visto na Figura \ref{fig:periodic-time} é uma opção
para se visualizar essa periodicidade. O tempo é mostrado no eixo circular e
cada anel do círculo representa um dia da semana.  Os setores representam uma
hora, com 24 setores no total que mostram a quantidade de tráfego em um
determinado ponto da cidade, mapeada em cores. A vantagem de um leiaute radial
é que ele evidencia esses padrões recorrentes e a desvantagem é que ele possui
pouca eficiência espacial \citep{Chen2015}.

\begin{figure}[!htb]
  \centering
  \includegraphics[width=0.6\textwidth]{../figuras/periodic-time.png}
  \caption[Leiaute radial representando tempo periódico]{Leiaute radial
representando tempo periódico. O tempo é mostrado no eixo circular e cada anel
do círculo representa um dia da semana, divididos em 24 setores que indicam as
24 horas do dia. As cores mostram a quantidade de tráfego conforme o mapa à
direita, onde vermelho maior é o tráfego. Fonte:
\citet{Pu2013} \label{fig:periodic-time}}
\end{figure}

  Os aspectos temporais dos dados não se limitam a apenas essas visualizações,
existe um grande arcabouço de projeções que se desenvolveram no campo da
visualização da informação onde o tempo é uma variável de destaque,
\citet{Cairo2016} mostra uma série dessas projeções em seu livro. Uma análise
linear ou periódica são dois paradigmas que levam a decisões diferentes durante
o processo de criação da visualização.

\subsection{Visualização das propriedades espaciais}
\label{sec:prop-espaciais}

  As propriedades espaciais de localização referem-se aos locais onde ações,
incidentes e eventos ocorrem. Diferentes níveis de agregação dessa informação
levam à categorização da visualização em três classes, segundo
\citet{Chen2015}: visualização baseada em pontos (nenhuma agregação),
visualização baseada em linhas (agregação de primeira ordem), e agregação
baseada em regiões (agregação de segunda ordem).

\begin{description}
  \item[Visualização baseada em pontos:] visualizações baseadas em pontos
consideram as informações do tráfego como pontos discretos e usam sua forma
pura como representação em um leiaute espacial, como por exemplo pontos em um mapa 2D. Essa técnica
mostra intuitivamente a posição de objetos em um certo momento no tempo. O
projeto \emph{Trains of Data}, ilustrado na Figura \ref{fig:trains-of-data},
utiliza esse tipo de visualização para representar a movimentação dos trens na
França. O tamanho dos pontos indica a quantidade de passageiros e a cor indica
se o trem está atrasado, sendo verde no horário e vermelho significa atraso.
Para uma visualização integral da trajetória eles ainda usam efeitos de
animação. Visualizações baseadas em pontos tipicamente representam cada ponto
de forma individual. Em casos onde há uma grande quantidade de pontos, o uso de
mapas de calor é indicado para a visualização da densidade. A vantagem desse
método é que ele permite observar os estados de cada objeto e sua distribuição
no espaço, como também explorar regiões da cidade que estejam mais ocupadas.
Por outro lado, ele é inapropriado para representação de informações contínuas,
como quantos veículos viajam de um determinado local para outro.

\begin{figure}[!htb]
  \centering
  \includegraphics[width=.6\textwidth]{../figuras/trains-of-data.jpg}
  \caption[Visualização baseada no mapa das ferrovias da França]{Posição dos trens às 19:43 na França com uma visualização baseada no mapa
das ferrovias. Fonte: \citet{Senseable2018}.}
  \label{fig:trains-of-data}
\end{figure}

  \item[Visualização baseada em linhas:] visualizações baseadas em linhas são
desenhadas para mostrar a trajetória de objetos, mapas de vias e estradas em
uma região, ou fluxos do tráfego em uma rede de transporte. As trajetórias e
fluxos são representados por linhas ou curvas e são escaladas ou coloridas de
acordo com suas propriedades (i.e. densidade, direção, velocidade).
\citet{Klein2013} apresenta um sistema iterativo para análise do tráfego aéreo
na França. Cada trajetória é representada por uma linha que conecta os pontos
de partida e chegada de cada voo, como pode ser visto na Figura
\ref{fig:air-traffic}.

\begin{figure}[!htb]
  \centering
  \includegraphics[width=1\textwidth]{../figuras/air-traffic.png}
  \caption[Visualização do tráfego aéreo na França]{Visualização do tráfego aéreo na França. Fonte: \citet{Klein2013}.}
  \label{fig:air-traffic}
\end{figure}

  A representação espacial das linhas pode ainda sofrer transformações
geométricas e topológicas que geram novas abstrações. Por exemplo,
\citet{Tarik2009} mostram uma proposta que transforma as trajetórias de um
leiaute espacial (Figura \ref{fig:viz-espacial}) para um leiaute abstrato
(Figura \ref{fig:viz-abstrata}) para visualizar a movimentação de pessoas
durante a fuga de uma explosão em um escritório.  A abordagem abstrata é usada
para mostrar mais efetivamente os padrões de movimentação de pessoas que se
movem ao mesmo tempo para as mesmas áreas. Na Figura \ref{fig:viz-abstrata} é
possível notar a movimentação de indivíduos antes da explosão (linhas azuis), o
que sugere possíveis suspeitos ou testemunhas do evento.  As informações
temporais são difíceis de representar no leiaute espacial, assim o leiaute
abstrato complementa a visualização dos dados mostrando os instantes de tempo
no eixo X e as informações espaciais no eixo Y.

\begin{figure}[ht!]
  \centering
  \begin{subfigure}[t]{0.45\textwidth}
    \centering
    \includegraphics[width=65mm]{../figuras/proximidade-espacial.png}
    \caption{Visualização espacial das trajetórias ao longo do tempo. \label{fig:viz-espacial}}
  \end{subfigure}
  ~
  \begin{subfigure}[t]{0.45\textwidth}
    \centering
    \includegraphics[width=75mm]{../figuras/proximidade-abstrata.png}
    \caption{Visualização abstrata da movimentação ao longo do tempo. \label{fig:viz-abstrata}}
  \end{subfigure}

  \caption[Visualização espacial vs abstrata da movimentação de pessoas em um
escritório]{Simulação de uma evacuação de um escritório depois de uma explosão.
(a) Visualização da movimentação da trajetória das pessoas no espaço. (b)
Visualização abstrata, baseada na proximidade.  Fonte: \citet{Tarik2009}.
\label{fig:tarik}}
\end{figure}

%  Algumas ideias buscam ainda sintetizar as informações espaciais e temporais
%em um cubo espaço-temporal, que utiliza o eixo Z para mostrar o tempo e os
%eixos X e Y para representar o espaço. 

  No entanto, a medida que o número de objetos cresce, aumenta também os
problemas de oclusão, o que acaba afetando a estética da visualização e a
obtenção de informações sobre os dados \citep{Zhou2013}. Uma forma de reduzir
a complexidade de análise de um grande conjunto  trajetórias é utilizando
outras formas de abstração dos dados, como uma visualização baseada em regiões.

  \item[Visualização baseada em regiões:] Esse tipo de abstração agrupa fluxos
de deslocamento com origem e destino similares em um nível de macro regiões,
geralmente determinado por divisões administrativas.  \citet{Zeng2013}
apresentam um diagrama em círculos para mostrar padrões de movimentação de
pessoas entre as regiões da cidade. O círculo representa a junção ou conexão
entre as diferentes regiões e o fluxo entre elas. A densidade do fluxo é medido
pela espessura, bem como a direção é destacada dentro do círculo, como é
ilustrado na Figura \ref{fig:interchange-circo}. Apesar de facilitar a
visualização, esse tipo de agregação abre mão das informações individuais de
cada trajetória em favor de uma visão condensada dos fluxos.

\begin{figure}[!htb]
  \centering
  \includegraphics[width=1\textwidth]{../figuras/region-based.png}
  \caption[Exemplo de visualização baseada em regiões do sistema de metrô na França]{Exemplo de visualização baseada em regiões: padrões regionais de movimentação no sistema de metrô na França. Fonte: \citet{Zeng2013}.}
  \label{fig:interchange-circo}
\end{figure}
\end{description}

\section{\emph{Bundling}}
\label{sec:bundling}

  Um grande problema na visualização de grafos e dados de trajetórias é a
oclusão visual à medida que a quantidade de elementos aumenta. Uma pequena
quantidade de dados já começa a apresentar problemas de sobreposição e
cruzamento de arestas, o que dificulta a obtenção de informação sobre os dados.
Uma maneira de contornar esse problema é através de filtros que determinam a
quantidade de itens na visualização. Como consequência do filtro perde-se a
visão global de todos os itens ao mesmo tempo, o que pode ser necessário para a
identificação das correlações e padrões entre eles. Uma outra abordagem é o uso
de agregações que geram novas abstrações dos dados. Na seção anterior mostramos
como o agrupamento em regiões projetado sobre um diagrama circular simplifica a
visualização das trajetórias entre as áreas da cidade.

 Uma solução que tem sido amplamente utilizada em várias pesquisas de
visualização de dados de tráfego é o uso de uma técnica chamada
\emph{bundling}. A técnica ajuda a simplificar o desenho da visualização
através da agregação espacial das arestas em conjuntos chamados
\emph{bundles}, que funcionam de forma similar a algoritmos de clusterização. Os \emph{bundles} são
definidos como um grupo de arestas similares, compatíveis o suficiente para
serem representadas por um corpo único e compacto \citep{Lhuillier2017}. A
compatibilidade é calculada a partir de uma função de similaridade usada para
determinar quais arestas devem fazem parte do mesmo agrupamento.
Imaginando então algumas viagens no trânsito, podemos agrupá-las pela sua
região de origem, destino, distância percorrida, direção ou até mesmo o meio de
transporte utilizado.  Dessa forma, o número de \emph{bundles} é ligeiramente
inferior ao número de trajetórias a serem desenhadas, ficando mais simples
compreender e visualizar a estrutura global, padrões e tendências entre grupos
de trajetórias que ligam áreas fortemente relacionadas \citep{Zhou2013}.  A
Figura \ref{fig:bundling-hierarquico} mostra o uso de uma técnica de
\emph{bundling} para visualização da hierarquia de módulos e arquivos de um
software onde as arestas mostram suas relações de dependência. A ligação
acontece quando um módulo acessa atributos e/ou funções definidos em outras partes
do software.

\begin{figure}[!htb]
  \centering
  \includegraphics[width=.5\textwidth]{../figuras/hbundling.png}
  \caption[Bundling hierárquico na visualização de dados de software]{Bundling hierárquico na visualização de dados de software. As cores mapeiam
o sentido do acesso, verde é a origem, vermelho é o alvo. Fonte: \citep{Holten2006}.}
  \label{fig:bundling-hierarquico}
\end{figure}

\subsection{Modelos de \emph{Bundling}}

  O cálculo de um \emph{bundle} não possui uma definição precisa, e por isso há
uma grande variedade de algoritmos e abordagens para modelar problemas de
agrupamento de arestas com \emph{bundling}, que podem variar significativamente
de um para outro em questões de complexidade e aplicação dos algoritmos
\citep{Zhou2013}.  \emph{Force-directed edge bundling} (FDEB), apresentado em
\citet{Holten2009}, cria \emph{bundles} através da atração entre pontos de
controle colocados ao longo das arestas e é considerado um modelo baseado em
custo, já que tenta minimizar o valor de uma função de custo que representa a
força de atração entre as arestas.  \emph{Hierarchical Edge Bundling} (HEB)
agrupa as arestas baseadas na estrutura hierárquica do grafo para o cálculo dos
\emph{bundles} e por isso é considerado um modelo geométrico \citep{Holten2006}. Outra
classificação são os modelos baseados em imagem, como \emph{Skeleton-based edge
bundling} (SBEB) e \emph{Kernel Density Estimation-based Edge Bundling}
(KDEEB), que surgiram mais recentemente. Eles utilizam algoritmos de
clusterização para extrair a estrutura geral do grafo para então calcularem os
\emph{bundles}. O algoritmo SBED utiliza o esqueleto do grafo como guias para
criar \emph{bundles} com muitas ramificações \citep{Ersoy2011}, enquanto o
KDEEB utiliza um mapa de densidade do grafo para extrair sua estrutura no
espaço da imagem \citep{Hurter2012}. A Figura \ref{fig:subfigures} dá um
exemplo da variedade de possíveis saídas geradas por diferentes algoritmos de
\emph{bundling} (FDEB, SDEB e KDEEB) aplicados nos mesmos dados.  Nesse exemplo
são usados dados de uma semana do tráfego aéreo sobre os EUA.

\begin{figure}[!htb]
  \centering
  \begin{subfigure}{0.6\textwidth}
    \centering
    \includegraphics[width=1\textwidth]{../figuras/FDEB.png}
    \caption{FDEB}
    \label{fig:FDEB}
  \end{subfigure}

  \begin{subfigure}{0.6\textwidth}
    \centering
    \includegraphics[width=1\textwidth]{../figuras/SBEB.png}
    \caption{SBEB}
    \label{fig:SBEB}
  \end{subfigure}

  \begin{subfigure}{0.6\textwidth}
    \centering
    \includegraphics[width=1\textwidth]{../figuras/KDEEB.png}
    \caption{KDEEB}
    \label{fig:KDEEB}
  \end{subfigure}
  \caption[Diferentes algoritmos de \emph{bundling} aplicadas nos mesmos dados]{Diferentes algoritmos de \emph{bundling} aplicadas em dados do tráfego aéreo dos EUA - 235 nós, 2099 arestas. Fonte: \citep{Klein2013}.}
  \label{fig:subfigures}
\end{figure}

  A escolha do modelo geralmente parte da observação e experiências de
aplicação em problemas semelhantes, já que conhecer e testar diferentes
abordagens e suas variações não é uma tarefa simples. Por isso,
\citet{Lhuillier2017} recentemente propôs uma nova taxonomia para os modelos e
algoritmos de \emph{bundling} dividindo-os com base no tipo de dados em que se deseja
aplicar a técnica. Eles justificaram que desta forma pesquisadores e usuários
da técnica pudessem focar no seu contexto de aplicação e não em detalhes
específicos de implementação algorítmica. Seu estudo dividiu os modelos
existentes inicialmente em dois grupos, conforme o tipo de dados, grafos ou
trajetórias. Então, mais detalhes sobre a direção (direcionado ou não),
dimensão (2D ou 3D) e dependência do tempo (dinâmico ou estático) refinam a
divisão dentro da sua taxonomia.


\begin{figure}[!htb]
  \centering
  \includegraphics[width=\textwidth]{../figuras/estado-da-arte.pdf}
  \caption[Taxonomia dos métodos de \emph{bundling} baseado no tipo de dado]{Taxonomia dos métodos de \emph{bundling} baseado no tipo de dado. Fonte: \citet{Lhuillier2017}}
  \label{table:bundling-methods}
\end{figure}


Podemos então seguir a taxonomia para selecionar uma método de
\emph{bundling} que seja mais adequado para a análise de fluxos no trânsito da
cidade. \emph{Attribute-Driven Edge Bundling} (ADEB) é um algoritmo que segue um
modelo baseado em imagem e é apresentado dentro desta taxonomia como um das
opções para a análise de dados de trajetórias direcionadas. Discutimos na
próxima seção as características desse modelo e como podemos nos beneficiar
dele em nosso estudo.

\subsection{Modelo Baseado em Imagem}
\label{sec:modelo-imagem}

  Uma das etapas do \emph{bundling} é a identificação das trajetórias
similares, que requerem que cada uma delas seja comparada a todos os outras
para descobrir quais são as mais próximas, afim de agrupá-las em
\emph{bundles}.  Esse cálculo, no entanto, é bastante custoso e pode demorar
minutos em cenários com milhares de trajetórias.

\emph{Kernel Density Estimation Edge Bundling} (KDEEB) é um algoritmo proposto
por \citet{Hurter2012}. Eles observaram que ao aplicar uma operação de \emph{bundling} $B$ qualquer sobre um
grafo $G$, o resultado são áreas de maior densidade (dentro dos \emph{bundles})
e áreas de menor densidade fora deles, em comparação ao grafo original. Assim,
essa operação \emph{bundling} $B$ pode ser moldada em função de uma operação
sobre a densidade dos pontos do grafo, similar ao que faz o algoritmo de
clusterização \emph{Mean Shift} \citep{Comaniciu2012}. A partir disso, a heurística
desse algoritmo de \emph{bundling} consiste em computar repetidamente o
gradiente dos pontos em relação à uma função de densidade, e então movendo-os
para regiões mais densas apontadas pelo gradiente. O maior benefício do método
é sua implementação paralela com o uso do poder computacional de placas
gráficas (GPUs), o que levou a ganhos de desempenho de uma ordem de magnitude
em relação a métodos anteriores. KDEEB representou uma abertura para uma área
chamada \emph{image-based bundling}, ou métodos baseados em imagem, onde $B$ é
implementado via operações de processamento de imagem, diferentemente de
métodos puramente geométricos existentes até então. O algoritmo consiste nos
seguintes passos:

\begin{enumerate}
  \item Converte o grafo $G$ em um mapa de densidade usando uma função de
densidade. A função utilizada é um estimador baseado em núcleos, geralmente um
estimador Gaussiano ou Epanechnikov.

  \item Computar o gradiente da função de densidade para cada ponto/nó do
grafo. O cálculo do gradiente indica a direção onde há maior quantidade
de nós/arestas aglomeradas no grafo.

  \item Mover os nós na direção do gradiente (áreas mais densas). Essa etapa é
suavizada movendo-se os nós pela norma do gradiente, ou seja, em apenas uma unidade
a cada iteração. Isso evita que as arestas dêem grandes saltos.

  \item Corrigir distorções causadas pela movimentação dos nós com filtro
Laplaciano (opcional).  O passo anterior pode causar pequenas distorções, como
por exemplo, a sobreposição de nós. Esta é uma forma de se corrigir essas
distorções sem que se perca a estrutura geral do grafo.

  Repetir a partir do passo 1 até a convergência do algoritmo, o que leva cerca de 8..10
iterações.
\end{enumerate}

%\citet{Hurter2012} apresentam ainda uma maneira para limitar os \emph{bundles}
%com relação a obstáculos presentes no caminho das arestas. Um obstáculo
%qualquer, representado por seus limites espaciais pode ser contornado
%alterando-se a função de densidade, que sofre uma degradação dentro da área
%geométrica do obstáculo. Com isso, o gradiente age como uma força de repulsão
%das arestas que cruzam o obstáculo e faz com que as arestas sejam movidas
%na direção pra fora de suas bordas.

  {\emph{Attribute-Driven Edge Bundling} (ADEB) é uma extensão do algoritmo
KDEEB, proposta por \citet{ZegarraFlores2016}, e que permite a separação das
arestas conforme sua direção. Essa informação é obtida percorrendo cada aresta
a partir do ponto de origem até o destino e calculando para cada uma delas um
vetor unitário que aponta a sua direção, que é mapeado para o
ângulo desse mesmo vetor. Então, cria-se um mapa de densidade direcionado, onde são
consideradas apenas as arestas que vão na mesma direção (mesmo ângulo). Esse
algoritmo herda todos os benefícios de desempenho e robustez apresentados pelo
KDEEB, o que o faz um método adequado para a análise do tráfego de veículos no
trânsito da cidade.

\subsection{\emph{Bundling} em Dados Dinâmicos}

  Um grafo dinâmico é aquele no qual seus nós e arestas são dependentes do
tempo, ou seja, a cada momento \emph{t}, existe um grafo G(t) diferente a ser
explorado. Nesse contexto, dois diferentes tipos de grafos dinâmicos podem ser
considerados, grafos sequenciais e grafos de fluxo \citep{Hurter2013}. Um
grafo sequencial consiste em um conjunto de grafos $G^i = (V^i, E^i)$, onde
cada grafo $G^i$ é como um retrato no momento $i$ de um sistema que é
dependente do tempo. Grafos de fluxo, por outro lado, são definidos como um
conjunto de vértices $V$ e arestas $E$, onde as arestas são definidas pelo seu
tempo de vida e nós. Grafos de fluxo não possuem uma sequência pré-definida e
são tipicamente obtidos de fontes de dados online.

  O uso de \emph{bundling} nesse cenário é similar a outros métodos de
visualização de grafos dinâmicos. Para grafos sequenciais o \emph{bundling} é
aplicado a cada grafo $G^i$ de maneira estática. Já em grafos de fluxo, os
dados são divididos em janelas de tempo e o algoritmo é aplicado em cada
intervalo $\Delta t$, que podem ser colocados lado a lado para comparação em
uma técnica conhecida como \emph{small multiples}. Uma outra maneira de se
visualizar a dinâmica é com o uso de animações, como mostram
\citet{Hurter2014}. A vantagem desse método sobre o anterior é que ele se
adequa melhor à análise de longas séries temporais.

\section{Simulação do Tráfego de Veículos}

  O crescimento da população nas cidades ao redor do mundo trouxe também vários
desafios de gestão e controle do trânsito, especialmente em grandes cidades,
que precisam lidar com vários problemas de congestionamentos, acidentes,
poluição, controle do transporte público, entre outros. Uma maneira alternativa
para estudar e solucionar esses problemas é através do uso de ferramentas de
simulação de cidades inteligentes. Tais simuladores implementam modelos que
reproduzem o ambiente nas cidades baseados no comportamento dos cidadãos, sua
estrutura física e também outros aspectos, como veículos, construções e até
postes de luz \citep{Arthur2019}. Essas ferramentas permitem criar cenários do
tráfego de veículos e ajudam nos estudos sobre seu comportamento e na
elaboração de soluções para serem aplicadas por gestores das cidades.
Recentemente, novas ferramentas de simulação com foco em escalabilidade foram
desenvolvidas, permitindo testes e experimentos em escala real de uma grande
cidade, com milhões de veículos. O InterSCSimulator é uma destas ferramentas e
o detalhamos a seguir.

\subsection{InterSCSimulator}
\label{sec:interscsimulator}

\citet{mabs2017} apresentam o InterSCSimulator, uma ferramenta de código livre,
capaz de executar simulações com mais de 4 milhões de veículos em uma grande
metrópole.  As simulações são feita em escala mesoscópica. Isso significa que
ela segue um modelo onde cada veículo do trânsito é simulado individualmente. O
simulador calcula para cada um deles as ações de movimentação durante sua
viagem.

\subsubsection{Componentes do InterSCSimulator}

  O simulador possui quatro componentes principais: o \textbf{cenário}, que
recebe os arquivos de entrada e cria o grafo da cidade e os primeiros veículos;
\textbf{o motor de simulação}, que executa os algoritmos e modelos da simulação
para gerar os arquivos com os resultados; um componente de
\textbf{visualização do mapa}, que recebe o arquivo de saída com os resultados
e gera uma visualização da simulação baseada em pontos; e finalmente um
componente de \textbf{visualização de gráficos} que gera uma série de gráficos
com informações sobre o cenário simulado, como velocidade média dos veículos ao
longo do tempo e outros. A Figura \ref{fig:interscsimulator} ilustra a
interação entre os componentes e os arquivos de entrada e saída.

\begin{figure}[!htb]
  \centering
  \includegraphics[width=\textwidth]{../figuras/arquitetura-simulador.pdf}
  \caption[Componentes do InterSCSimulator]{Componentes do InterSCSimulator. \label{fig:interscsimulator}}
\end{figure}

  O simulador utiliza dois arquivos principais de entrada para definição do
cenário, são eles map.xml e trips.xml. O primeiro contém a descrição da rede
rodoviária da cidade e é construído com dados obtidos de serviços como Open
Street Maps (OSM)\footnote{\rurl{openstreetmap.org}}. O arquivo é composto por
nós e links, que representam cruzamentos e ruas, respectivamente. A Listagem
\ref{map.xml} mostra um exemplo de um arquivo map.xml com 3 nós e 3 links. Um
usuário da simulação pode ainda fazer alterações nesse arquivo para alterar o
mapa da cidade para atender a um novo cenário, como por exemplo, remover os nós
e links de uma rua conhecida para simular o bloqueio causado por um acidente.
Utilizaremos desse mecanismo para estudar diferentes cenários do trânsito.

\begin{lstlisting}[style=myxml, caption={Exemplo de arquivo map.xml que define a rede rodoviária da cidade. Fonte: \citet{mabs2017}}, label=map.xml]
<network>
  <nodes>
   <node id ="1" x="-46.65805" y="-23.58162"/>
   <node id ="2" x="-46.65828" y="-23.58342"/>
   <node id ="3" x="-46.65228" y="-23.59341"/>
  </nodes>
  <links>
    <link id="35985" from="1" to="2" length="100" free speed="40"/>
    <link id="35985" from="2" to="3" length="200" free speed="40"/>
    <link id="35985" from="3" to="1" length="80"  free speed="50"/>
  </links>
</network>
\end{lstlisting}

  O segundo arquivo de entrada, \emph{trips.xml}, descreve todas as viagens que
devem ser simuladas. Cada viagem deve informar os nós de origem e destino e o
horário da simulação em que ela será iniciada, como pode ser visto na Listagem
\ref{trips.xml}. O caminho entre a origem e o destino pode ainda ser fixada
previamente nesse arquivo ou pode ser calculado pelo simulador, o qual fica
responsável por determinar a rota a ser seguida. Um aspecto interessante é que
se quisermos duplicar o tamanho da simulação, podemos fazer isso simplesmente
duplicando as linhas desse arquivo.

\begin{lstlisting}[style=myxml, caption={Exemplo de arquivo trips.xml que define a rede rodoviária da cidade. Fonte: \citet{mabs2017}}, label=trips.xml]

<scsimulator_matrix>
  <trip origin="247951669" destination="60641382"
    type="car" start_time="28801"/>
  <trip origin="60641382" destination="247951669"
    type="car" start_time="63001"/>
  <trip origin="4511105625" destination="2109902387"
    type="car" start_time="16201"/>
  <trip origin="247951669" destination="60641382"
    type="car" start_time="54001"/>
  <trip origin="246650787" destination="247951670"
    type="car" start_time="54001"/>
  <trip origin="247951670" destination="246650787"
    type="car" start_time="66601"/>
  <trip origin="246650787" destination="60641382"
    type="bus" start_time="54001"/>
</scsimulator_matrix>
\end{lstlisting}

  Cada veículo (carro ou ônibus) definido no arquivo \emph{trips.xml} é um
elemento independente dentro da simulação. Eles percorrem seu trajeto do nó de
origem até o nó de destino seguindo a rede rodoviária da cidade definida no
arquivo \emph{map.xml}.  Cada veículo possui quatro estados dentro da
simulação: \emph{Start}, quando o tempo da simulação atinge o tempo de início
do veículo; \emph{Move}, quando a simulação atinge o tempo do próximo movimento
do veículo; \emph{Wait}, quando o veículo tem que esperar até o próximo
movimento; e \emph{Finish}, quando o veículo chega ao seu destino. Os estados
representam eventos ocorridos dentro da simulação, que são registrados no
arquivo de saída \emph{events.xml}. A Listagem \ref{events.xml} mostra parte de
um arquivo de saída gerado durante uma simulação. É possível ver os eventos de
\emph{start\_trip}, \emph{move} e \emph{finish\_trip} para dois veículos com os
identificadores $2121$ e $2223$.  É importante notar também o atributo
\emph{time}, que indica o tempo de ocorrência do evento dentro da simulação.
Esse atributo  é contabilizado em segundos. Na simulação de um dia do trânsito,
por exemplo, temos então 86400 segundos, correspondentes às 24 horas do dia.

\begin{lstlisting}[style=myxml, caption={Exemplo de arquivo de saída events.xml com os eventos da simulação. Fonte: \citet{mabs2017}}, label=events.xml]

<events version="1.0">
  <event time="4" type="start_trip" vehicle="2121" link="5243" />
  <event time="4" type="start_trip" vehicle="2223" link="1002" />
  <event time="11" type="move" vehicle="2223" link="4005" />
  <event time="31" type="move" vehicle="2121" link="4005" />
  <event time="38" type="move" vehicle="2223" link="2007" />
  <event time="52" type="finish_trip" vehicle="2121" link="4005" />
  <event time="52" type="finish_trip" vehicle="2223" link="5243" />
</events>
\end{lstlisting}

  A partir desse arquivo de saída e do mapa da cidade é possível montar o o
rastro de mobilidade dos veículos ao longo do tempo, recuperando as posições de
latitude e longitude dos \emph{links} sobre o qual eles se movimentam, como
mostra a Tabela \ref{table:rastro}. O rastro de um veículo da simulação pode
ser visto na Figura \ref{fig:rastro}. O ponto verde indica a origem na cor
preta está marcado o destino. Já os pontos vermelhos marcam cada evento de
movimentação que ele fez dentro da simulação.

\begin{table}[!htb]
\centering
\begin{tabular}{|c|c|c|c|c|}
\hline
\textbf{Horário} & \textbf{Ação} & \textbf{ID} & \textbf{Latitude} & \textbf{Longitude} \\
\hline
25 & start & 4858\_52 & -23.624235 & -46.648388 \\
27 & start & 4858\_73 & -23.624235 & -46.648388 \\
27 & start & 94\_17 & -23.535976 & -46.63561    \\
31 & move & 4858\_52 & -23.624077 & -46.648453 \\
33 & move & 4858\_73 & -23.624077 & -46.648453 \\
35 & start & 8394\_43 & -23.561275 & -46.695423 \\
\hline
\end{tabular}
\caption{Tabela com o rastro do tráfego montado a partir do arquivo de saída e o mapa da cidade. \label{table:rastro}}
\end{table}

\begin{figure}[!htb]
  \centering
  \includegraphics[width=\textwidth]{../figuras/pontos.png}
  \caption{Visualização dos pontos da trajetória do veículo com ID 4858\_52. \label{fig:rastro}}
\end{figure}

Já a Figura \ref{fig:simulated-traffic} mostra uma simples visualização em linhas
de um conjunto de trajetórias do tráfego simulado. Os dados são resultado
da pesquisa de \citet{santana2018courb}, que criaram uma simulação com milhares
de veículos e disponibilizaram os arquivos com os resultados. Utilizamos dados
no intervalo de 1 hora do trânsito filtrando pontos que estivessem entre os
intervalos de tempo $21600$ e $25200$, o que corresponde ao horário entre 6 e 7
da manhã. Note que há bastante oclusão, sendo também impossível distinguir
quais origens vão para quais destinos ou a quantidade de veículos nas vias.

\begin{figure}[!htb]
  \centering
  \includegraphics[width=1\textwidth]{../figuras/trafego-ocluso.png}
  \caption{Simulação do tráfego entre 6 e 7 da manhã.}
  \label{fig:simulated-traffic}
\end{figure}

  Os conceitos apresentados servem de base para a nossa proposta de
visualização dos dados do tráfego de veículos no trânsito. O processo processo
de formulação da proposta inclui a definição do tipo de visualização que iremos
utilizar, obter domínio sobre os dados do trânsito e suas propriedades, para
posteriormente chegarmos a uma análise dos fluxos de origem e destino em
diferentes níveis de detalhe a partir de um modelo de \emph{bundling}, além de
investigar formas de destacar essas propriedades dentro da visualização. No
próximo capítulo apresentamos alguns trabalhos que utilizam \emph{bundling} e
outras técnicas para análise de dados de tráfego.

\par
%Falar de visualizações no geral, multi atributos, multi-escalas, agrupamento,
%filtros, cores, bundling, modelos, KDEEB, CUBU.

\chapter{Trabalhos Relacionados}
\label{cap:trabalhos-relacionados}

  A visualização de dados de tráfego tem um amplo histórico. Nossa revisão traz
alguns dos trabalhos e avanços recentes nessa área. As pesquisas que julgamos
relevantes tocam as áreas de visualização da informação - \emph{InfoViz}, cujo
foco é desvendar novos métodos para visualizar uma informação, como
\emph{bundling}, e também a área de análise visual aplicada - \emph{Visual
Analytics}, que se relaciona mais a sistemas e soluções utilizadas em ambientes
do mundo real. Ambos trabalhos trazem contribuições em diferentes níveis para a
nossa pesquisa. Grande parte do conhecimento em visualização do tráfego e
também sobre \emph{bundling} pode ser encontrado em quatro trabalhos recentes
que fizeram um levantamento geral sobre os esforços na área.

  \citet{Telea2018} apresenta uma revisão geral das áreas de visualização
científica e de processamento de imagem, apontando métodos utilizados nessas
áreas para que beneficiaram a visualização de grandes grafos multivariados e
complexos. Os novos métodos para simplificação de dados, então chamados de
baseados em imagem, geraram novas técnicas de \emph{bundling} escaláveis, que
permitiram a visualização de grandes grafos, com diversos atributos e
milhares de nós. O algoritmo \emph{Attributed-Driven Edge Bundling} (ADEB), que utilizaremos,
é um dos algoritmos que vêm dessa linhagem e se beneficia desses métodos. 

  \citet{Lhuillier2017} traz um estudo ainda mais aprofundado sobre o
\emph{bundling}. Os autores sugerem uma definição formal sobre a operação de
\emph{bundling} em um conjunto de dados, e a partir daí listam as características,
objetivos e limitações de uso das várias técnicas e algoritmos
existentes na literatura. O resultado é uma visão geral sobre o estado da arte na área
e seu desenvolvimento em vários segmentos como grafos, mapas de fluxo,
coordenadas paralelas, campos de tensores e outras aplicações. Dentre suas
contribuições está uma nova taxonomia para a ajudar pesquisadores e usuários a
selecionarem os algoritmos de \emph{bundling} com base no tipo de dado.
Apresentamos essa taxonomia na Figura \ref{table:bundling-methods} na Seção
\ref{sec:modelos-de-bundling}. A partir dela, selecionamos o algoritmo ADEB
para a nossa pesquisa.

  Já \citet{Andrienko2017Visual} e \citet{Chen2015} avaliam uma série de outros
trabalhos que fazem análises de dados do tráfego e de mobilidade em geral e
discutem questões sobre os tipos de dados, técnicas de visualização utilizadas
e principalmente o objetivo das propostas. Os trabalhos listados abrangem
vários contextos, como análise de incidentes no tráfego, monitoramento de
veículos em tempo real, detecção de congestionamentos, sugestão de rotas e
outras atividades executadas por usuários e especialistas de transporte. (O que
eles concluem?)

  Além dos estudos citados anteriormente, reunimos um conjunto de propostas
cujo o foco está na análise de fluxos de origem-destino em dados do tráfego
para visualização de tendências, fluxos dominantes e seus atributos, como
direção, velocidade, distâncias percorridas e outros.  \citet{Zeng2013} e
\citet{Andrienko2017}  apresentam sistemas interativos para a visualização de
fluxos de origem e destino em dados geoespaciais gerais, logo se encaixam nesse
grupo. Eles propõem novos tipos de leiaute radial para visualizar o tráfego, e
codificam a direção, intensidade e distância percorrida pelos objetos que se
locomovem. \citet{Zeng2013} menciona o uso de \emph{bundling} para agrupar as
linhas dentro do anel de seu leiaute radial (Figura
\ref{fig:interchange-circo}), mas não se aprofundam no assunto. 

 \citet{Landersberg2016} usa uma representação em linhas do tráfego sobre um
mapa e apresenta algoritmos de clusterização para agrupar o tráfego por regiões
e também no tempo, reduzindo a oclusão do desenho. Sua análise é focada em destacar diferentes momentos onde há
mudanças significativas nos fluxos. O desafio de sua técnica é estabelecer métricas
que capturam esses momentos.  Eles apresentam um sistema interativo e
verificam seu método com dados de \emph{tweets} geolocalizados na cidade de Londres e
também de redes de celulares. \citet{Klein2014} faz uma análise do tráfego aéreo
da França para detectar as conexões entre os diferentes aeroportos, pontos de
congestionamentos e permitir uma exploração com base em vários atributos, como
direção e altitude dos voos, além de uma janela de tempo que permite navegar
entre diferentes instantes do tráfego. Eles utilizam o algoritmo de
\emph{bundling} KDEEB para reduzir a oclusão na visualização das trajetórias
e apresentam uma interpolação das trajetórias sem \emph{bundling} para visualizar
sua direção, já que o KDEEB não inclui atributos como direção no processo de \emph{bundling}.

\citet{Ferreira2013} fazem uma análise de origem e destino de 500 mil viagens
de taxis feitas em um dia na cidade de Nova Iorque. Sua visualização é baseada
em pontos que destacam as origens e destinos das viagens com cores diferentes.
O resultado é similar a um mapa de densidades que mostra a distribuição das
viagens sobre a cidade. Eles também implementam uma estratégia robusta de
organização dos dados na memória para permitir que os usuários façam buscas e
apliquem filtros em uma grande quantidade de dados de maneira interativa. 

\citet{Anita2017} apresenta uma nova técnica de \emph{bundling}, resultado de
uma otimização do algoritmo FDEB, criado por \citet{Selassie2011}. Para isso,
utilizam uma etapa de clusterização aplicada previamente nos dados, e
posteriormente aplicam o \emph{bundling} em cada cluster. A vantagem da sua
abordagem é que ela pode ser utilizada em conjunto com outros métodos de
\emph{bundling}. \citet{Kim2018} utilizam campos de tensores para construir
um mapa de fluxo de dados geoespaciais analisando apenas as informações
estatísticas da distribuições dos pontos ao longo do tempo. Sua abordagem, no entanto
implica uma série de restrições estatísticas nos dados, como número mínimo 
de amostras e distribuição uniforme dos dados ao longo do tempo.

  Em nossa revisão, buscamos obter um panorama geral das lacunas que podem ser
preenchidas em relação ao uso de \emph{bundling} na visualização multi escala
de veículos no trânsito. Observamos vários aspectos de cada trabalho, como
o uso de recursos para visualização de padrões globais e locais do tráfego, o tipo de
objeto do tráfego analisado e tamanho do conjunto de dados, atentando-nos com interesse
especial em dados do trânsito e também o uso de \emph{bundling}. Concluímos que os dados de
trânsito ainda são pouco explorados, principalmente em trabalhos com
\emph{bundling}. A Tabela \ref{table:trabalhos} mostra os trabalhos que avaliamos
quanto ao uso de dados do trânsito, \emph{bundling} e o leiaute da visualização.

%O maior problema que observamos é a variedade de trabalhos que utilizam
%técnicas e conjuntos de dados diferentes sem comparar de fato os benefícios que
%eles trazem aos usuários finais, confirmando o que diz
%\cite{Andrienko2017Visual}, que os trabalhos de visualização são muito
%distantes do grupo de transportes e que as soluções nem sempre endereçam
%problemas reais. Seria interessante fazer experimentos comparativos de cada uma
%dessas visualizações que possuem objetivos em comum na análise dos fluxos de
%origem-destino na cidade. Desenhar tal experimento, no entanto é um grande
%desafio dadas as muitas variáveis subjetivas. Muitas delas, no entanto, abordam
%vantagens do ponto de vista computacional, que poderiam ser melhor verificados
%no que tange a aspectos de escalabilidade e flexibilidade quanto ao tipo de
%dados de tráfego suportado.

\begin{table}[htb!]
\begin{tabular}{|M{0.58}|M{0.11}|M{0.115}|M{0.09}|}
\hline
\textbf{Trabalho}       & \textbf{Dados do Trânsito} & \textbf{\emph{Bundling}} & \textbf{Leiaute}  \\ \hline
Nossa proposta          & \checkmark                 & \checkmark               &          linhas   \\ \hline
\citet{Kim2018}         & x                          &  x                       &          linhas   \\ \hline
\citet{Andrienko2017}   & \checkmark                 &  x                       &          radial   \\ \hline
\citet{Anita2017}       & x                          & \checkmark               &          linhas   \\ \hline
\citet{Landersberg2016} & x                          &  x                       &          linhas   \\ \hline
\citet{Klein2014}       & x                          & \checkmark               &          linhas   \\ \hline
\citet{Ferreira2013}    & \checkmark                 &  x                       &          pontos   \\ \hline
\citet{Zeng2013}        & x                          & \checkmark               &          radial   \\ \hline

\end{tabular}
\caption{Análise dos trabalhos relacionados quanto ao uso de dados do trânsito, uso de \emph{bundling} e leiaute da visualização. \label{table:trabalhos}}
\end{table}


\par

\chapter{Mobilidade em São Paulo}
\label{cap:mobilidade-rmsp}

Para fornecer uma melhor compreensão do cenário de nossa análise, esta seção
descreve as características da Região Metropolitana de S\~ao Paulo e os dados da
última pesquisa OD realizada na região. Para evitar antagonismos entre a
Região Metropolitana de S\~ao Paulo Área (RMSP) e a cidade de S\~ao Paulo, vamos nos
referir à cidade como capital ou simplesmente S\~ao Paulo e vamos nos referir à
área metropolitana como RMSP ou simplesmente área metropolitana. 

\section{Região Metropolitana de São Paulo}

O estado de S\~ao Paulo está localizado no litoral sudeste do Brasil. A Regi\~ao
Metropolitana de S\~ao Paulo é a região mais populosa da América do Sul. De acordo
com o Instituto Brasileiro de Geografia e Estatística \citep{ibge2020}, a RMSP contabilizou $21,9$
milhões de cidadãos em 2020, o que representa cerca de 10\% da população
brasileira. A RMSP é composta por 39 municípios em uma área de \num{7946.84} $km^{2}$. A
cidade de São Paulo é a capital do estado de São Paulo e está localizada no
centro da região metropolitana. A Figura \ref{fig:map-spma} mostra os 39 municípios que compõem a
RMSP. A capital é a cidade mais populosa do Brasil, com 12.3 milhões de
habitantes. Na RMSP, as outras cidades com mais habitantes são Guarulhos (1.4
milhão), São Bernardo do Campo (844 mil), Santo André (721 mil) e Osasco (699
mil).

\begin{figure}[!htb]
  \centering
  \captionsetup{justification=centering}
  \includegraphics[width=0.95\textwidth]{../figuras/map-spma.png}
    \caption{Municípios da RMSP.\label{fig:map-spma}}
\end{figure}

A capital e as cidades mais próximas concentram a maior parte das oportunidades
de emprego, instalações e serviços públicos, universidades, museus e opções
de entretenimento. Portanto, há um grande deslocamento diário para o centro da
capital e seus arredores. Em São Paulo, os bairros próximos ao centro da
cidade são mais valorizados, portanto, morar nessas partes da cidade tem um maior custo.
Pessoas com menos condições financeiras costumam residir na periferia da capital
ou em outras cidades da RMSP, por isso, algumas cidades da RMSP tornaram-se cidades
dormitório para aqueles que não podem pagar comprar ou alugar uma casa na
capital.

Em relação à infraestrutura de transporte, a capital possui uma rede metroviária
que atende as regiões norte, sul, leste, oeste, sudoeste e sudeste da cidade.
Existem algumas linhas de metrô, a maioria delas cruzando o centro da capital, o
que limita o acesso ao sistema de metrô a algumas regiões da cidade. As demais
cidades não possuem sistemas de metrô, mas existe um outro sistema ferroviário
metropolitano que atende várias cidades do entorno e que é integrado ao sistema
metroviário da capital.

Cada cidade tem seu próprio sistema de ônibus e há um sistema de ônibus
intermunicipal para ligar o cidades vizinhas. Durante o século 20 até o final da
década de 1990, a maioria dos investimentos no transporte foram guiados por uma
abordagem focada no transporte individual de carros em detrimento do transporte
público \cite{rolnik2011}. Os objetivos eram ampliar estradas e ruas para
atender à crescente demanda por carros particulares. Nas últimas duas décadas,
os governos locais têm investido mais em políticas de incentivo ao transporte
público, como implantação de corredores de ônibus e substituição da frota de
ônibus, construção de novas linhas de metrô e modernização do sistema
ferroviário. Além disso, há uma crescente infraestrutura de ciclismo na capital
que vem sendo expandida na última década. Embora esses investimentos tenham
aumentado nos últimos anos, a RMSP ainda sofre com o congestionamento do
tráfego, principalmente nos horários de pico \cite{rolnik2011,ricardo:18}.
Assim, é necessário um melhor entendimento do o comportamento do trânsito na
RMSP para propor novas políticas de melhoria da mobilidade para seus cidadãos.

\section{Pesquisa Origem-Destino}

\par
%Trabalhos que usam bundling na visualização de dados em geral (outros modelos/frameworks)
%Trabalhos que usam visualização de dados de OD, com bundling ou não
%Trabalhos que usam o CUBU 
%Trabalhos que usam os dados da OD

\chapter{Metodologia}
\label{cap:metodologia}
 Este capítulo descreve a nossa proposta de visualização para exploração dos
fluxos no tráfego de veículos. O objetivo principal da visualização é centrado
nas propriedades espaciais e temporais dos dados para entender como são os
fluxos de origem e destino pela cidade ao longo do tempo. Buscamos um maior
nível de detalhes através de uma abordagem multinível para explorar padrões
globais e locais do tráfego, com diferentes níveis de agregação temporais e
espaciais.

 Apresentamos inicialmente a ferramenta que iremos utilizar para responder às
questões de pesquisa levantas no Capítulo \ref{cap:introducao}, e que
relembramos a seguir. Damos uma visão geral dos seus componentes e dos
artefatos de entrada e saída que irão resultar na visualização e possibilitar a
análise dos dados. A ferramenta faz parte do ecossistema de soluções para
cidades inteligentes que são desenvolvidas no contexto do projeto
InterSCity\footnote{\rurl{interscity.org}}. Em seguida detalhamos os conjuntos
de dados que iremos utilizar e suas características, e por fim, explicamos as
técnicas que pretendemos aplicar para destacar as propriedades dos dados dentro
de uma visualização para responder às nossas questões:

Como podemos oferecer uma visualização de grandes
quantidades de dados de mobilidade de uma região
metropolitana:

\begin{itemize}
  \item \textbf{Q1:} Como podemos oferecer uma visualização de grandes quantidades
de dados de mobilidade de uma região metropolitana?

\end{itemize}


\section{Conjuntos de Dados}

Falar das ODS, falar quais ODS vamos utilizar


\subsection{Processamento dos dados}

 - Expansão das viagens
 - Limpeza
 - Filtragem das features de interesse
 - Features derivadas
 - Transformação dos dados para o formato do CUBU

\section{Visualização}

  Explorar todas as possibilidades seria um trabalho extensivo. Focamos
então em algumas classes de problemas dentro da visualização de dados
com bundling para explorar o seu potencial na análise de fluxos de origem e
destino na cidade.

\subsection{Agregação Espacial dos Dados}

1) Escalar o agrupamento conforme o zoom e resolução diferentes e dados diferentes (tunning não muito óbvio)

	- Número de pontos

	- Resolução da imagem

	- tamanho do kernel

	- Glanuralidade dos dados

  \begin{itemize}
    \item x) Aplicar bundling com alta resolução e alto numero de pontos, kernel grande e kernel pequeno
    \item x) Aplicar bundling com alta resolução e baixo numero de pontos, kernel grande e kernel pequeno
    \item x) Aplicar bundling com baixa resolução e alto numero de pontos, kernel grande e kernel pequeno
    \item x) Agrupar os dados na mão em regiões e aplicar o bundling em macro regiões. Qual a alguma diferença?
    \item x) Mostrar o bundling dos dados dos centróides e dos dados específicos em alguns modais (metrô, a pé, bike, taxi)
  \end{itemize}

	- Explicar que não há uma métrica, mas dissertar sobre qual fica melhor visualmente

	- Vantagens e desvantagens em relação às visualizações da OD em diferentes escalas

\subsection{Explorando outras propriedades}

- Aplicar o bundling para ver regiões densas, conforme mostrados na OD

  \begin{itemize}
    \item x) Mostrar as features de cores padrão (distância, direção, densidade)
    \item x) Mostrar as features de cores implementadas (tipo de transporte, modo de transporte, renda, gênero)
    \item x) Mostrar que tem pessoas que andam de metrô em locais que não tem metrõ (adicionar mapa)
    \item x) Mostrar como os ônibus metropolitanos se interceptam com os de são paulo
    \item x) Mostrar que os filtros de dados por horário pra mostrar que há mais fluxo
       no horário de pico, que os onibus escolares saem todos pela manhã, mas que também temos
      pessoas que saem de madrugada para trabalhar
    \item x) Mostrar a renda e correlacionar distância do centro com renda
    \item x) Mostrar correlação de renda com carro (pobre anda de carro?)
    \item x) Derivar um atributo do quão o horário de pico afeta e usar como VALUE

    \item x) Quem é baixa renda e não vão ao destino do centro, ele fica no local dele ou vai para lugares absurdos? (Excluir o Centro)
    \item x) Tentar extrair alguma ideia do mestrado da Haydee sobre genero
    % Tentar mostrar umas duas ou 3 coisas para mostrar pra o pessoal da CET
  \end{itemize}

	- Quais vantagens e desvantagens em relação ao que existe na OD (substitui alguma coisa do relatório da OD?)

	- Qual o nível de detalhes conseguimos extrair alguma informação?

  
\subsection{Estratégias de bundling}

2) Mostrar a mesma grandeza de diferentes maneiras

- Grandezas: Tempo/Tipo de transporte/Direção
- Estratégias: Fazer bundle de tudo e aplicando filtro E cores vs separado,
diferenciar por cores no bundling de tudo, e filtrar as arestas no tempo (manhã
e tarde)

  \begin{itemize}
    \item x) Fazer bundling de todos os dados de transporte e filtrar por modo
    \item x) Fazer bundling de todos os dados de transporte e colorir por modo
    \item x) Fazer bundling dos modos de transporte separados
    \item x) Fazer bundling dos dados em direção opostas
  \end{itemize}

	- Vantagens e desvantagens em relação aos gráficos da OD?

	- Qual os níveis de detalhes conseguimos obter?


\subsection{Evolução Temporal ao Longo dos anos}

4) Evolução temporal. Desafios: Correspondência de regiões

  \begin{itemize}
    \item x) Aplicar bundling de tudo: filtrar por ano
    \item x) Aplicar bundling separado em cada ano
    \item x) Aplicar bundling na diferença de um ano pro outro
    \item x) Observar viagens de Taxi não convencional de um ano pro outro
    \item x) Observar viagens de linhas novas de metro de um ano pro outro
  \end{itemize}

	Que vantagens/diferenças consigo ver em relação aos relatórios da OD?
	- Falar de bundling estático e dinâmico, mas que é future work...


\section{Avaliação da Visualização} 

\par
%Aqui a gente fala dos dados da OD que vamos utilizar. O que representam, como
%obter, quais vamos usar e porque.
%
%Aqui falamos do framework CUBU, suas funcionalidades, como obtivemos e como
%vamos implementar coisas novas nele


%O que vou mostrar/experimentar: exemplos dentro de algumas categorias
%
%
%1) Explorando as propriedades espaciais (coisas óbvias)
%- Aplicar o bundling para ver regiões densas, conforme mostrados na OD
%	x) Mostrar as features de cores padrão (distância, direção, densidade)
%	x) Mostrar as features de cores implementadas (tipo de transporte, modo de transporte, renda, gênero)
%	x) Mostrar que tem pessoas que andam de metrô em locais que não tem metrõ (adicionar mapa)
%	x) Mostrar como os ônibus metropolitanos se interceptam com os de são paulo
%	x) Mostrar que os filtros de dados por horário pra mostrar que há mais fluxo
%		 no horário de pico, que os onibus escolares saem todos pela manhã, mas que também temos
%		pessoas que saem de madrugada para trabalhar
%	x) Mostrar a renda e correlacionar distância do centro com renda
%	x) Mostrar correlação de renda com carro (pobre anda de carro?)
%
%	- Quais vantagens e desvantagens em relação ao que existe na OD (substitui alguma coisa do relatório da OD?)
%	- Qual o nível de detalhes conseguimos extrair alguma informação?
%
%
%2) Mostrar a mesma grandeza de diferentes maneiras
%- Grandezas: Tempo/Tipo de transporte/Direção
%- Estratégias: Fazer bundle de tudo e aplicando filtro&cores vs separado, diferenciar por cores no bundling de tudo, e filtrar as arestas no tempo (manhã e tarde)
%	x) Fazer bundling de todos os dados de transporte e filtrar por modo
%	x) Fazer bundling de todos os dados de transporte e colorir por modo
%	x) Fazer bundling dos modos de transporte separados
%	x) Fazer bundling dos dados em direção opostas
%	x) Filtrar por horário de pico
%	x) Derivar um atributo do quão o horário de pico afeta e usar como VALUE
%	x) Usar o fator de expansão para mostrar a densidade como VALUE
%
%	Vantagens e desvantagens em relação aos gráficos da OD?
%	Qual os níveis de detalhes conseguimos obter?
%
%
%3) Escalar o agrupamento conforme o zoom e resolução diferentes e dados diferentes (tunning não muito óbvio)
%	- Número de pontos
%	- Resolução da imagem
%	- tamanho do kernel
%	- Glanuralidade dos dados
%	x) Aplicar bundling com alta resolução e alto numero de pontos, kernel grande e kernel pequeno
%	x) Aplicar bundling com alta resolução e baixo numero de pontos, kernel grande e kernel pequeno
%	x) Aplicar bundling com baixa resolução e alto numero de pontos, kernel grande e kernel pequeno
%	x) Agrupar os dados na mão em regiões e aplicar o bundling em macro regiões. Qual a alguma diferença?
%	x) Mostrar o bundling dos dados dos centróides e dos dados específicos em alguns modais (metrô, a pé, bike, taxi)
%
%	Explicar que não há uma métrica, mas dissertar sobre qual fica melhor visualmente
%	- Vantagens e desvantagens em relação às visualizações da OD em diferentes escalas
%
%4) Evolução temporal
%	- Desafios: Correspondência de regiões
%	x) Aplicar bundling de tudo: filtrar por ano
%	x) Aplicar bundling separado em cada ano
%	x) Aplicar bundling na diferença de um ano pro outro
%
%	Que vantagens/diferenças consigo ver em relação aos relatórios da OD?
%	- Falar de bundling estático e dinâmico, mas que é future work...


% \chapter{Resultados}
\label{sec:results}

Essa seção apresenta os resultados das análises que desenvolvemos sobre os dados
da pesquisa OD 17. Começamos com a exploração dos recursos de visualização
suportados pelo \emph{CUBu}, com ênfase na codificação visual de atributos
densidade, distância e direção das viagens. Esses aspectos são apresentados a
seguir nas Seções~\ref{sec:density},~\ref{sec:trail-overlap},~\ref{sec:length-direction},~and~\ref{sec:coloring}.
Posteriormente, utilizando tais recursos visuais analisamos outros padrões de
mobilidade específicos a subconjuntos dos dados, os quais são detalhados nas
Seções~\ref{sec:strata},~\ref{sec:students},~\ref{sec:peak-hours},~\ref{sec:dist_reasons},~and~\ref{sec:mode}.

\section{Visualizando a densidade dos \emph{bundles}}
\label{sec:density}

Na Seção~\ref{sec:bundling} explicamos como a operação de \emph{bundling}
faz o agrupamento de trajetórias simplificando a visualização e reduzindo a oclusão
da imagem. No entanto, tal operação não nos diz quantas trajetórias foram agrupadas em
um \emph{bundle}. A solução para isso, primeiramente apresentada por \cite{holten06},
é desenhar linhas semi-transparentes, cada uma com uma transparência fixa $\alpha < 1$.
Assim, a combinação mostrará trajetórias de alta densidade como mais opacas e as de baixa
densidade como mais transparentes. Apesar da transparência ajudar na diferenciação
de áreas densas, ela por si só não é uma variável visual quantitativa forte \cite{slocum09}.
Então, codificamos também a densidade das trajetórias em cores, utilizando
os valores estimados pelo KDE durante o processo de \emph{bundling} (ver Seção~\ref{sec:bundling}).
A Figura~\ref{fig:bundled-graph-density}a mostra uma visualização obtida usando codificação
da densidade em cores aplicada em todo o conjunto de dados da OD17 contendo
as \num{685115} viagens. Podemos ver alguns caminhos com maior densidade, mas a imagem
ainda apresenta uma demasiada carga de informação e muitas áreas opacas. Isso ocorre pelo fato de que,
usualmente em GPUS de consumo comum, a transparência $\alpha$ é modelada por um valor
inteiro de 8 bits. Portanto, apenas 255 níveis de transparência diferentes são possíveis,
ou seja, apenas 255 níveis de densidade das trajetórias podem ser exibidos. Valores
de $\alpha$ muito altos saturam o canal de transparência
onde ocorrem as densidades mais altas - todas as densidades acima de 255 são fixadas
em 255. Valores abaixo de 1/255 resultam em nenhuma imagem, uma vez que
isso corresponde a opacidade zero na representação de 8 bits.

Para resolver este problema, mapeamos a densidade $\rho$ de duas maneiras,
transparência e cor. Já que $\rho$ é calculado precisamente como um número
de ponto flutuante durante a estimação do KDE, nenhum valor é truncado ou arredondado.
Essa estimativa da densidade permite modular a transparência para destacar ainda
mais as áreas de alta densidade e reduzir a oclusão da imagem -- uma outra alternativa
seria utilizar valores maiores de kernel $k$, o que agruparia ainda mais as trajetórias, gerando
\emph{bundles} mais fortes, porém também iria causar uma maior distorção das linhas.
A Figura~\ref{fig:bundled-graph-density}b mostra o \emph{bundling} aplicado nos mesmos
dados da Figura~\ref{fig:bundled-graph-density}a. Podemos observar que os \emph{bundles}
aparecem mais salientes após aplicar a modulação da transparência. A imagem sugere que a rede do tráfego metropolitano
pode ser dividida em algumas ramificações principais que são fortemente conectadas à área central,
onde a cidade de São Paulo está localizada. Isso faz sentido considerando que esta é a parte
mais populosa da área metropolitana. Além disso, a maioria dos sistemas de transporte
cruzam o centro da capital, incluindo linhas de metrô, trem, e as principais vias expressas.

\begin{figure}[!htb]
  \centering
  \captionsetup{justification=centering}
  \includegraphics[width=0.98\textwidth]{../figuras/figure1}
  \caption{\emph{Bundling} das trajetórias coloridas pela densidade (a) valores fixos e \\(b) transparência modulada.}
  \label{fig:bundled-graph-density}  
\end{figure}

% % How we calculated 44%:
% %
% % 1. Take all trips that indicate bus, metro, and train as
% %    the main transportation mode -> this is the amount of
% %    trips by public transportation
% %
% % 2. Take all trips that indicate use of metro or train for
% %    some portion of the trip and divide that by the amount
% %    of trips by public transportation
% %
% % Using the tables from pages 46 and 49 of
% % http://www.metro.sp.gov.br/pesquisa-od/arquivos/Ebook%20Pesquisa%20OD%202017_final_240719_versao_4.pdf
% %
% % We consider that:
% % trips using bus as the main transportation mode: 8034
% % trips using train as the main transportation mode: 1245
% % trips using metro as the main transportation mode: 3400
% % total trips by public transportation: 8034 + 1245 + 3400 = 12679
% %
% % trips using metro for some portion of the trip: 3400
% % trips using train for some portion of the trip: 2272
% % total trips involving train or metro: 3400 + 2272 = 5672
% %
% % Percentage = 5672*100/12679 = 44.73%
% %
% % Not relevant for the text, just a curiosity:
% % the percentage in relation to all motorized trips is 5672*100/28280=20.05%

\section{Infraestrutura de metrô e trem \emph{vs} \emph{bundling}}
\label{sec:trail-overlap}

O sistema de transporte público é o mais utilizado pelos moradores da RMSP. O
impacto da malha ferroviária sobre o deslocamento de pessoas fica claro quando
desenhamos as linhas ferroviárias ao longo das trajetórias agrupadas com o
\emph{bundling}. A Figura~\ref{fig:rails} mostra a alta correspondência dos
\emph{bundles} com os caminhos das linhas ferroviárias (desenhadas em preto).
Tendo em vista que, de acordo com a pesquisa OD17, cerca de 44\% das viagens
diárias de transporte público envolvem metrô ou trens, este é um resultado
esperado. Curiosamente pode-se questionar se o sistema ferroviário foi planejado
com precisão para atender a demanda, como sugere a visualização agrupada, ou se
a disponibilidade dessa opção de transporte influenciou a existência de fluxos
tão densos. Embora não temos os insumos para responder a essa pergunta, os
gestores de tráfego podem usar esse tipo de visualização para elaborar políticas
para o transporte público. Apesar de não expressar nenhuma grande surpresa sobre os
dados analisados, este é um resultado bastante importante, pois consideramos que a alta
correlação entre o \emph{bundling} das trajetórias e as linhas das ferrovias
também indica boas configurações de parâmetros para esse tipo de visualização na
escala da região metropolitana.

Ressaltamos que que este tipo de correlação (de \emph{bundles} com estradas) não
é o mesmo que o utilizado no método RAEB, \cite{zeng:19}. No método RAEB, o
agrupamento foi feito explicitamente para seguir estradas. Em nosso caso, as
linhas são sobrepostas sobre \emph{bundles}, que foram gerados unicamente a partir dos
dados da OD. Pode-se argumentar que RAEB, neste sentido, produz \emph{bundles}
mais ``corretos'', uma vez que estes são forçados para seguir as estradas. No
entanto, olhando mais de perto, podemos ver que RAEB não pode ter todos os
\emph{bundles} seguindo precisamente todos os caminhos das estradas - pois isso
basicamente bloquearia qualquer agrupamento do \emph{bundling} e resultaria no próprio
mapa das estradas. Além disso, RAEB requer que o registro dos pontos das trajetórias
seja feito dentro de uma rede rodoviária precisa. Isso torna-o
significativamente mais complexo para implementar e mais caro para processar do
que nossa solução baseada em \emph{CUBu}.

\begin{figure}[!htb]
  \centering
  \captionsetup{justification=centering}
  \includegraphics[width=0.98\textwidth]{../figuras/rail-lines.png}
    \caption{\emph{Bundling} das trajetórias coloridas pela densidade e a malha ferroviária da RMSP}
  \label{fig:rails}  
\end{figure}

\section{Mapeando distância e direção no \emph{bundling}}
\label{sec:length-direction}

Para explorar a mobilidade urbana de diferentes perspectivas, precisamos de
meios para visualizar seus múltiplos atributos de dados. No entanto, esta
variedade de atributos requer diferentes estratégias de visualização. Dois
importantes atributos para o estudo de padrões de mobilidade são distância
percorrida na viagem e a sua direção. As Figuras~\ref{fig:attributes-length} e
\ref{fig:attributes-direction} mostram a visualização de todo o conjunto de
dados OD17 mapeando a distância e direção, respectivamente.

A Figura~\ref{fig:attributes-length} exibe os comprimentos das viagens
codificados por cores. Nela utilizamos o mesmo mapa de cores arco-íris da
Figura~\ref{fig:rails} e também a modulação da transparência por densidade,
conforme explicado na Seção~\ref{sec:density}. Nesta imagem, podemos observar
uma única curva vermelha aparentemente na horizontal. Sua alta opacidade implica
que há muitas viagens longas, todas mapeadas perfeitamente para essa trajetória
entre a mesma origem e destino (se não o fizessem, veríamos um \emph{bundle} se
ramificando no formato de um leque em vez de uma curva precisa). Esta é uma
descoberta interessante que, argumentamos, não poderia ser facilmente encontrada
usando métodos não visuais. Apesar desse ponto incomum, as outras trajetórias,
em geral, percorrem distâncias regulares. \emph{Bundles} de longa distância como
este podem indicar falta de serviços ou recursos que não satisfazem as regiões
locais, obrigando as pessoas a percorrerem longas distâncias para acessá-los. A
pesquisa OD17 contém mais informações que podem ajudar a investigar o motivo
dessas longas viagens.

A Figura~\ref{fig:attributes-direction} mostra os mesmos dados da
Figura~\ref{fig:attributes-length}, mas ao invés da distância são as direções
das viagens que estão codificadas em cores. Para este atributo em específico,
usamos ainda um recurso do \emph{CUBu}, que separa trilhas em direções opostas
em dois \emph{bundles} quase paralelos. Podemos ver claramente a existência de
trajetórias paralelas ao longo dos \emph{bundles}, o que não é surpreendente
porque a pesquisa OD registra o trajeto típico das pessoas que inclui os
deslocamentos de ida e vinda de volta para suas origens. No entanto, essa
simetria das trajetórias possivelmente não seria observada se analisássemos um
curto período do dia.

\begin{figure}[!htb] \centering \captionsetup{justification=centering}
  \includegraphics[width=0.98\textwidth]{../figuras/distances.png}
  \caption{Mapeamento da distância das viagens em cores. \label{fig:attributes-length}}
  \end{figure}

\begin{figure}[!htb]
  \centering
  \captionsetup{justification=centering}
  \includegraphics[width=0.98\textwidth]{../figuras/directions.png}
  \caption{Mapeamento da direção das viagens em cores. \label{fig:attributes-direction}}
\end{figure}

\section{Coloring transportation modes: local \emph{vs} intercity buses}
\label{sec:coloring}
% %
% The OD17 dataset contains 17 transportation modes. While it would be ideal to be able to see the 17 categories all at once in our bundled visualization, that would not be easy to do, since it would require the simultaneous encoding of 17 different categorical attributes. Instead, we use transparency to hide trails according to a user-set selector that filters them by transportation mode. Figure~\ref{fig:bus-integration} shows how we can use these filters to visualize the integration between buses from the city of S\~ao Paulo (local buses) and intercity buses. %Unbundled and bundled trajectories are presented on the left and right sides respectively.
% Each transportation mode has a distinct color -- olive for local buses and blue for intercity buses.

% Figure~\ref{fig:bus-integration} also highlights that these different transportation systems appear to complement each other. The city of S\~ao Paulo has a very active commerce and industry, and many people from neighboring cities work there. Thus, the availability of public transportation and its integration is very important for these people. This kind of filtering along with bundling helps to better understand correlations between data attributes -- in this case, \hbox{transportation}~modes.


% \begin{figure}[!htb]
%   \centering
%   \captionsetup{justification=centering}
%   \includegraphics[width=0.98\textwidth]{figures/local-intercity-buses}
%   \caption{Edges filtered by transportation modes: bundled trails of local and intercity buses. \label{fig:bus-integration}}
% \end{figure}

\section{Density per social strata}
\label{sec:strata}
% % 6 imagens, uma de cada classe social, density1
% We used our bundle visualization to study how citizens with different economical conditions commute in the SPMA. The Brazilian Economical Classification Criterion (BECC)\,\cite{cceb2008} is the official social-economic index used in the Brazilian Demographic Census, which is performed by the Brazilian Institute of Geography and Statistics. It measures the purchasing power of the Brazilian society. The BECC is divided into six levels or strata (Table~\ref{tab:becc}). This index is used in the OD17 survey to complement the mobility data. Table~\ref{tab:becc} also shows the average monthly income in the local currency (Brazilian reals) and in US dollars, and the number of trips in the SPMA for each BECC level considering the whole population and only citizens with age between 6 and 18 years that commute for study (see Section~\ref{sec:students}).

% To compare the mobility patterns of different BECC social strata, we bundled the trails in each stratum separately, as shown in Figures~\ref{fig:becc-axd-e}~to~\ref{fig:becc-d-e}. We see significant differences in the mobility patterns between the highest and lowest income levels as shown in Figure~\ref{fig:becc-axd-e}. The $A$ level (on the left-hand side) has a high density in the center of SPMA, which includes the capital downtown surroundings. The highest density is located in the west, southwest, and northeast neighborhoods near downtown. There are density flows between the capital and the cities of Barueri and Cotia, which have high-income residential areas. There are other high dense flows linking the capital to the cities of S\~ao Bernardo do Campo and Santo Andr\'e. Comparing $A$ to the $D$-$E$ level (on the right-hand side of Figure~\ref{fig:becc-axd-e}), we see that $D$-$E$ has the highest dense flows in the capital eastern region. In the $D$-$E$ level map, we can see the absence of high-dense flows in regions that are nearest to the capital downtown; in contrast, these are present in the $A$ level map. We can see more details of $A$ and $D$-$E$ strata in Figures~\ref{fig:becc-a}~and~\ref{fig:becc-d-e}

% \begin{figure}[!htb]
%   \centering
%   \captionsetup{justification=centering}
%   \includegraphics[width=0.98\textwidth]{figures/comparison-axd-e-strata-leg.png}
%   \caption{Comparing density of trails between social strata $A$ (left-hand side) and $D$-$E$ (right-hand side). \label{fig:becc-axd-e}}
% \end{figure}


% \begin{figure}[!htb]
%   \centering
%   \captionsetup{justification=centering}
%   \includegraphics[width=0.98\textwidth]{figures/1-class-a.png}
%   \caption{Density of trails of social stratum $A$. \label{fig:becc-a}}
% \end{figure}

% \begin{figure}[!htb]
%   \centering
%   \captionsetup{justification=centering}
%   \includegraphics[width=0.98\textwidth]{figures/2-class-b1.png}
%   \caption{Density of trails of social stratum $B1$. \label{fig:becc-b1}}
% \end{figure}

% \begin{figure}[!htb]
%   \centering
%   \captionsetup{justification=centering}
%   \includegraphics[width=0.98\textwidth]{figures/3-class-b2.png}
%   \caption{Density of trails of social stratum $B2$. \label{fig:becc-b2}}
% \end{figure}

% \begin{figure}[!htb]
%   \centering
%   \captionsetup{justification=centering}
%   \includegraphics[width=0.98\textwidth]{figures/4-class-c1.png}
%   \caption{Density of trails of social stratum $C1$. \label{fig:becc-c1}}
% \end{figure}

% \begin{figure}[!htb]
%   \centering
%   \captionsetup{justification=centering}
%   \includegraphics[width=0.98\textwidth]{figures/5-class-c2.png}
%   \caption{Density of trails of social stratum $C2$. \label{fig:becc-c2}}
% \end{figure}

% \begin{figure}[!htb]
%   \centering
%   \captionsetup{justification=centering}
%   \includegraphics[width=0.98\textwidth]{figures/6-class-d-e.png}
%   \caption{Density of trails of social strata $D$-$E$. \label{fig:becc-d-e}}
% \end{figure}

% When we compare all maps from the $A$ to the $D$-$E$ level (Figures~\ref{fig:becc-a}~to~\ref{fig:becc-d-e}), we see that the densest flows (red) tend to displace from the capital downtown to the eastern region of the city. The concentration of high-density flows is increasingly spreading from the center to the peripheral regions of the SPMA. Even the less dense flows are increasing and spreading over the SPMA. However, the $D$-$E$ map shows that those flows diminish considerably for these social strata. This may indicate that low-income citizens have less access to the urban mobility system. As a consequence, these people would have less access to the social, educational, health, and cultural services of the SPMA, as those facilities are concentrated in the center regions of the cities. It is worthy to note that those central regions also have more job opportunities. Looking at the $D$-$E$ map, we can see a ``hole" in the capital west downtown. This region (Pinheiros district) concentrates a large number of jobs related to information technology and financial services, which requires workers with high and medium education levels. Thus, the map shows that low-income citizens are not going to that region, which reflects the inequality of opportunities that these citizens face.

% \begin{table}[!htb]
%   \small
%   \newcommand{\hdr}[1]{\bfseries#1}
%   \centering
%   \caption{Trips grouped per BECC income level, social stratum, and traveler age.\label{tab:becc}}
%   \begin{tabular}{>{\footnotesize}c>{\footnotesize}r>{\footnotesize}r>{\footnotesize}r>{\footnotesize}r}
%     \toprule
%     \multirow{2}[2]{*}{\hdr{BECC level}} & \hdr{Monthly income} & \hdr{Monthly income}& \multirow{2}[2]{*}{\hdr{Trips}} & \hdr{Trips of 6 to 18}\\
%     & \hdr{(Brazilian reals (R\$))} & \hdr{(US dollars ($\sim$US\$))} & & \hdr{years old students}\\
%     \midrule
%     A   & 23,345    & 4,245 & 3,062,892  &   184,772\\
%     B1  & 10,386    & 1,888 & 3,854,040  &   260,652\\
%     B2  & 5,363     & 975   & 12,856,182 &   963,242\\
%     C1  & 2,965     & 539   & 11,277,159 &   976,745\\
%     C2  & 1,691     & 307   & 7,852,806  &   721,218\\
%     D-E & 708       & 128   & 2,233,801  &   219,612\\
%     \bottomrule
%   \end{tabular}
% \end{table}

\section{Mobility of young students from different social strata}
\label{sec:students}
% % od17-escola-alta-renda-6-18, od17-escola-baixa-renda-6-18 density1
% To explore even more the mobility patterns showed bundling visualizations, we compared the trips of students from different social strata. We filtered citizens with age between 6 and 18 years whose commuting reason is study. We split them into two groups, the high- to moderate-income, which includes the BECC levels $A$, $B1$, $B2$, and $C1$; and the low-income, which includes levels $C2$ and $D$-$E$. Figures~\ref{fig:students-high}~and~\ref{fig:students-low} show the density maps for both groups.

% \begin{figure}[!htb]
%   \centering
%   \captionsetup{justification=centering}
%   \includegraphics[width=0.98\textwidth]{figures/high-income-density-leg.png}
%   \caption{Density of trails of young students from high-income households. \label{fig:students-high}}
% %\end{figure}
% \vspace*{\floatsep}
% %\begin{figure}[!htb]
%   \centering
%   \captionsetup{justification=centering}
%   \includegraphics[width=0.98\textwidth]{figures/low-income-density-leg.png}
%   \caption{Density of trails of young students from low-income households. \label{fig:students-low}}
% \end{figure}

% The density map of the high- to moderate-income students (Figure~\ref{fig:students-high}) shows a large number of dense flows spread across the central region of SPMA. This part of SPMA concentrates most private schools, universities, and complementary colleges. In addition, high density flows are not as long as flows from other maps with all the data (e.g., Figure~\ref{fig:bundled-graph-density}). This indicates that trips to study are shorter than trips to work. 

% The density map of low-income students (Figure~\ref{fig:students-low}) shows that their mobility is very limited compared to the higher-income students. There are a few dense flows, most of them out of the capital downtown. The high density flows of low-income students are more present in the peripherals of the capital and also in the neighboring cities. There is a concentration of both groups in the southwest region, where are the neighborhoods of the Campo Limpo district. 

% It is worthy to note that the public schools in the SPMA are spread across the central and peripheral parts of the cities. The students are enrolled in these schools according to the proximity of their residences. Thus, they do not have to travel long distances to reach their schools. Also, public schools have lower educational performance than private schools in S\~ao Paulo. Thus, citizens with better financial conditions use to put their children in private schools.

% The scarcity of flows from low-income students may indicate that they do not have equal opportunities to study, being enrolled in schools in their neighborhood. They also do not use to go to the central region of the city and, thus, have less access to universities and complementary colleges. This inequality of opportunities will probably impact these students' jobs and economical conditions.

% We also see that there are many more trails for the high- to moderate-income students (Figure~\ref{fig:students-high}) than for low-income students (Figure~\ref{fig:students-low}). The high-to moderate-income students also travel large distances to study, which indicates that they can choose more flexibly where to study. This fact is corroborated by urban mobility studies that indicate that people with better financial conditions have more mobility than those with poorest conditions\,\cite{carruthers2005,lucas2016}.

\section{Directions at peak hours}
\label{sec:peak-hours}
% %
% As discussed earlier in Sec.~\ref{sec:data}, Figure~\ref{fig:trips_by_hour} shows the distribution of trips by hour of the day, with two main rush-hour peaks (6-9 AM and 5-8 PM). 
% However, this aggregated table does not give us insights in how the rush-hour patterns may differ. To see this, we selected the two rush-hour time intervals and visualized them separately, using directional bundling and color-coding.

% \begin{figure}[!htb]
%   \centering
%   \captionsetup{justification=centering}
%   \includegraphics[width=0.98\textwidth]{figures/peak-hours-6h-to-9h-direction-leg.png}
%   \caption{Directions of trips between 6 to 9 AM \label{fig:peak-hours-6h-9h}}
% %\end{figure}
% \vspace*{\floatsep}
% %\begin{figure}[!htb]
%   \centering
%   \captionsetup{justification=centering}
%   \includegraphics[width=0.98\textwidth]{figures/peak-hours-17h-to-20h-direction-leg.png}
%   \caption{Directions of trips between 5 to 8 PM \label{fig:peak-hours-17h-20h}}
% \end{figure}

% Comparing the peak hours, we can see that morning flows going to the SPMA center (Figure~\ref{fig:peak-hours-6h-9h}, cyan bundle)
% are overall denser and longer than the flows coming from the SPMA center during the afternoon/evening peak (Figure~\ref{fig:peak-hours-17h-20h}, red). This suggests that in the morning people are in a hurry to reach their work, while they are less in a hurry to go back home (or to other destinations like schools or the gym) in the afternoon/evening.

% In Figure~\ref{fig:peak-hours-6h-9h}, we can see that flows in the morning peak going to the capital downtown (cyan bundle coming from the east) are denser than opposite flows (red bundle going to the east). Although flows leaving the capital downtown in the morning are thinner than their opposite ones, they also concentrate a large number of trips, especially to the east and southwest. In Figure~\ref{fig:peak-hours-17h-20h}, the opposite flows seem more equally distributed.

\section{Density by transportation mode}
\label{sec:mode}
% %% on foot, bike, car, metro (sliced, density)
% We next split the OD17 data by transportation mode to compare the flow patterns for four different transportation modes: pedestrians, bicycles, cars, and subway. Figures~\ref{fig:mode-pedestrian}~to~\ref{fig:mode-subway} show the respective visualizations.

% \begin{figure}[!htb]
%   \centering
%   \captionsetup{justification=centering}
%   \includegraphics[width=0.98\textwidth]{figures/mode-pedestrian-density-leg.png}
%   \caption{Density of pedestrian trips \label{fig:mode-pedestrian}}
% \end{figure}

% \begin{figure}[!htb]
%   \centering
%   \captionsetup{justification=centering}
%   \includegraphics[width=0.98\textwidth]{figures/mode-bike-density-leg.png}
%   \caption{Density of bicycle trips \label{fig:mode-bike}}
% \end{figure}

% \begin{figure}[!htb]
%   \centering
%   \captionsetup{justification=centering}
%   \includegraphics[width=0.98\textwidth]{figures/mode-car-density-leg.png}
%   \caption{Density of car trips \label{fig:mode-car}}
% \end{figure}

% \begin{figure}[!htb]
%   \centering
%   \captionsetup{justification=centering}
%   \includegraphics[width=0.98\textwidth]{figures/mode-subway-density-leg.png}
%   \caption{Density of subway trips \label{fig:mode-subway}}
% \end{figure}

% Pedestrian trails (Figure~\ref{fig:mode-pedestrian}) form several low-density `islands' spread across the SPMA, with the densest one (red in figure) being in the capital downtown. Most trails are quite short, which is expected (pedestrians). However, we see a few longer bundles between the capital downtown and the south and north regions of the city. Dense flows are also present in the neighboring cities of Diadema, Tabo\~ao da Serra, Osasco, Guarulhos, Po\'a, and Mogi das Cruzes. Upon examination, we found these dense flows to match the cities' downtown and commercial areas. This information could be useful to find places that could deserve the attention of local governments to provide improvements for pedestrians.

% As most of the pedestrian trips are short, the bundling technique forms a few flows over the SPMA.
% Using bundling for those short trips result in low-density trails, which is less useful compared to long trips. Thus, in these cases it may not be necessary to use bundling. In the upper left area of Figure~\ref{fig:mode-pedestrian}, we can see the OD trails without using bundling, which are near identical to the main bundled area.
% %For pedestrian trails, it would be interesting to apply bundling over small areas. \textbf{AT: Actually, you can do this by running CUBu simply using a much smaller kernel size. But, if we do not do that, I think we should omit the above sentence.}

% Bicycle trips (Figure~\ref{fig:mode-bike}) exhibit similar patterns to pedestrian ones. They are shorter than three kilometers on average. In this figure, we see some thin flows in the capital downtown area. There are also some more salient flows in the capital northeast and in the neighboring cities of Suzano and Guarulhos.
% However, comparing Figure~\ref{fig:mode-bike} with all other transportation means, we immediately see that bicycle trips are by far the least numerous, and exhibit a far sparser pattern, with few star-shaped `hubs' where many trails meet. This suggests that the cycling infrastructure is quite limited, and fragmented. Figure~\ref{fig:mode-bike} also shows trails without using bundling in the upper left corner.

% The car trips (Figure~\ref{fig:mode-car}) show a pattern similar to the one displaying the entire dataset, \emph{i.e.}, all transportation modes (see \emph{e.g.} Figure~\ref{fig:bundled-graph-density}). For a start, this tells that cars are \emph{the} dominant form of transportation in the SPMA, accounting for the main traffic patterns. The highest-density flows occur in the capital downtown. There are several high-density flows linking the downtown area to the other regions of the capital, and also coming and going from the cities of Guarulhos, Barueri, Cotia, S\~ao Bernardo do Campo, Santo Andr\'e, Mau\'a, and Mogi das Cruzes. Compared to all other transportation modes, cars show a far more `spread out' pattern that covers very large areas, indicating that cars are the prevalent transportation mode in most parts of the SPMA.

% Finally, subway trips (Figure~\ref{fig:mode-subway}) show a strong star-shaped pattern, with very high density bundles that connect the capital with the neighboring cities, due to the integration of the subway system with the train system. Compared to all other transportation modes, subways show a clearer, simpler, trip pattern structure.

\section{Different trip reasons}
\label{sec:dist_reasons}
% We next aim to study whether trips done for different reasons exhibit distinct trip patterns. For this, we create bundled visualizations from the OD17 data with trips grouped by work, health, education, and shopping. Figures~\ref{fig:reason-work}~to~\ref{fig:reason-shopping} show the results.

% Work-related trips (Figure~\ref{fig:reason-work}) are overall longer than the other trip reasons, and also cover a larger area (see the central agglomeration in the figure). Interestingly, the longest trips, between the east side and the city center (red bundle), are similar in pattern to the longest trips for health and education. 
% Trips for health reasons are sparser than work-related ones, and also show a more star-like pattern, with long bundles connecting to the central area. This may indicate that peripheral regions are not well served by health services. Trips for studying reasons (Figure~\ref{fig:reason-education}) have the largest distances between the northeast and the western regions of the SPMA. Their pattern is somewhere in-between the work and health trips. Interestingly, education trips show several `loops' in the center of the SPMA. Finally, shopping trips (Figure~\ref{fig:reason-shopping}) show the least dense, and overall also shortest, patterns, apart from a few outliers like the red (important) bundle connecting the center to the northeast. This tells that, unlike health, education, and work, shopping facilities (which are actually provided by private companies) are better distributed over the SPMA. This outlines that bundled visualizations are useful not only when they show the \emph{presence} of certain data, \emph{e.g.} trails linking far-apart regions; the \emph{absence} of patterns is also insightful, as in the case of the lack of long shopping trips.

% \begin{figure}[!htb]
% \centering
% \captionsetup{justification=centering}
% \includegraphics[width=0.98\textwidth]{figures/reason-work-leg.png}
% \caption{Distance of trips for work reasons.\label{fig:reason-work}}
% \end{figure}
  
% \begin{figure}[!htb]
% \centering
% \captionsetup{justification=centering}
% \includegraphics[width=0.98\textwidth]{figures/reason-health-leg.png}
% \caption{Distance of trips for health-related reasons.\label{fig:reason-health}}
% \end{figure}

% \begin{figure}[!htb]
% \centering
% \captionsetup{justification=centering}
% \includegraphics[width=0.98\textwidth]{figures/reason-school-leg.png}
% \caption{Distance of trips for education reasons.\label{fig:reason-education}}
% \end{figure}

% \begin{figure}[!htb]
% \centering
% \captionsetup{justification=centering}
% \includegraphics[width=0.98\textwidth]{figures/reason-shopping-leg.png}
% \caption{Distance of trips for shopping reasons.\label{fig:reason-shopping}}
% \end{figure}


% \par

\chapter{Resultados}
\label{sec:results}

Essa seção apresenta os resultados das análises que desenvolvemos sobre os dados
da pesquisa OD 17. Começamos com a exploração dos recursos de visualização
suportados pelo \emph{CUBu}, com ênfase na codificação visual de atributos
densidade, distância e direção das viagens. Esses aspectos são apresentados a
seguir nas Seções~\ref{sec:density},~\ref{sec:trail-overlap},~\ref{sec:length-direction},~and~\ref{sec:coloring}.
Posteriormente, utilizando tais recursos visuais analisamos outros padrões de
mobilidade específicos a subconjuntos dos dados, os quais são detalhados nas
Seções~\ref{sec:strata},~\ref{sec:students},~\ref{sec:peak-hours},~\ref{sec:dist_reasons},~and~\ref{sec:mode}.

\section{Visualizando a densidade dos \emph{bundles}}
\label{sec:density}

Na Seção~\ref{sec:bundling} explicamos como a operação de \emph{bundling}
faz o agrupamento de trajetórias simplificando a visualização e reduzindo a oclusão
da imagem. No entanto, tal operação não nos diz quantas trajetórias foram agrupadas em
um \emph{bundle}. A solução para isso, primeiramente apresentada por \cite{holten06},
é desenhar linhas semi-transparentes, cada uma com uma transparência fixa $\alpha < 1$.
Assim, a combinação mostrará trajetórias de alta densidade como mais opacas e as de baixa
densidade como mais transparentes. Apesar da transparência ajudar na diferenciação
de áreas densas, ela por si só não é uma variável visual quantitativa forte \cite{slocum09}.
Então, codificamos também a densidade das trajetórias em cores, utilizando
os valores estimados pelo KDE durante o processo de \emph{bundling} (ver Seção~\ref{sec:bundling}).
A Figura~\ref{fig:bundled-graph-density}a mostra uma visualização obtida usando codificação
da densidade em cores aplicada em todo o conjunto de dados da OD17 contendo
as \num{685115} viagens. Podemos ver alguns caminhos com maior densidade, mas a imagem
ainda apresenta uma demasiada carga de informação e muitas áreas opacas. Isso ocorre pelo fato de que,
usualmente em GPUS de consumo comum, a transparência $\alpha$ é modelada por um valor
inteiro de 8 bits. Portanto, apenas 255 níveis de transparência diferentes são possíveis,
ou seja, apenas 255 níveis de densidade das trajetórias podem ser exibidos. Valores
de $\alpha$ muito altos saturam o canal de transparência
onde ocorrem as densidades mais altas - todas as densidades acima de 255 são fixadas
em 255. Valores abaixo de 1/255 resultam em nenhuma imagem, uma vez que
isso corresponde a opacidade zero na representação de 8 bits.

Para resolver este problema, mapeamos a densidade $\rho$ de duas maneiras,
transparência e cor. Já que $\rho$ é calculado precisamente como um número
de ponto flutuante durante a estimação do KDE, nenhum valor é truncado ou arredondado.
Essa estimativa da densidade permite modular a transparência para destacar ainda
mais as áreas de alta densidade e reduzir a oclusão da imagem -- uma outra alternativa
seria utilizar valores maiores de kernel $k$, o que agruparia ainda mais as trajetórias, gerando
\emph{bundles} mais fortes, porém também iria causar uma maior distorção das linhas.
A Figura~\ref{fig:bundled-graph-density}b mostra o \emph{bundling} aplicado nos mesmos
dados da Figura~\ref{fig:bundled-graph-density}a. Podemos observar que os \emph{bundles}
aparecem mais salientes após aplicar a modulação da transparência. A imagem sugere que a rede do tráfego metropolitano
pode ser dividida em algumas ramificações principais que são fortemente conectadas à área central,
onde a cidade de São Paulo está localizada. Isso faz sentido considerando que esta é a parte
mais populosa da área metropolitana. Além disso, a maioria dos sistemas de transporte
cruzam o centro da capital, incluindo linhas de metrô, trem, e as principais vias expressas.

\begin{figure}[!htb]
  \centering
  \captionsetup{justification=centering}
  \includegraphics[width=0.98\textwidth]{../figuras/figure1}
  \caption{\emph{Bundling} das trajetórias coloridas pela densidade (a) valores fixos e \\(b) transparência modulada.}
  \label{fig:bundled-graph-density}  
\end{figure}

% % How we calculated 44%:
% %
% % 1. Take all trips that indicate bus, metro, and train as
% %    the main transportation mode -> this is the amount of
% %    trips by public transportation
% %
% % 2. Take all trips that indicate use of metro or train for
% %    some portion of the trip and divide that by the amount
% %    of trips by public transportation
% %
% % Using the tables from pages 46 and 49 of
% % http://www.metro.sp.gov.br/pesquisa-od/arquivos/Ebook%20Pesquisa%20OD%202017_final_240719_versao_4.pdf
% %
% % We consider that:
% % trips using bus as the main transportation mode: 8034
% % trips using train as the main transportation mode: 1245
% % trips using metro as the main transportation mode: 3400
% % total trips by public transportation: 8034 + 1245 + 3400 = 12679
% %
% % trips using metro for some portion of the trip: 3400
% % trips using train for some portion of the trip: 2272
% % total trips involving train or metro: 3400 + 2272 = 5672
% %
% % Percentage = 5672*100/12679 = 44.73%
% %
% % Not relevant for the text, just a curiosity:
% % the percentage in relation to all motorized trips is 5672*100/28280=20.05%

\section{Infraestrutura de metrô e trem \emph{vs} \emph{bundling}}
\label{sec:trail-overlap}

O sistema de transporte público é o mais utilizado pelos moradores da RMSP. O
impacto da malha ferroviária sobre o deslocamento de pessoas fica claro quando
desenhamos as linhas ferroviárias ao longo das trajetórias agrupadas com o
\emph{bundling}. A Figura~\ref{fig:rails} mostra a alta correspondência dos
\emph{bundles} com os caminhos das linhas ferroviárias (desenhadas em preto).
Tendo em vista que, de acordo com a pesquisa OD17, cerca de 44\% das viagens
diárias de transporte público envolvem metrô ou trens, este é um resultado
esperado. Curiosamente pode-se questionar se o sistema ferroviário foi planejado
com precisão para atender a demanda, como sugere a visualização agrupada, ou se
a disponibilidade dessa opção de transporte influenciou a existência de fluxos
tão densos. Embora não temos os insumos para responder a essa pergunta, os
gestores de tráfego podem usar esse tipo de visualização para elaborar políticas
para o transporte público. Apesar de não expressar nenhuma grande surpresa sobre os
dados analisados, este é um resultado bastante importante, pois consideramos que a alta
correlação entre o \emph{bundling} das trajetórias e as linhas das ferrovias
também indica boas configurações de parâmetros para esse tipo de visualização na
escala da região metropolitana.

Ressaltamos que que este tipo de correlação (de \emph{bundles} com estradas) não
é o mesmo que o utilizado no método RAEB, \cite{zeng:19}. No método RAEB, o
agrupamento foi feito explicitamente para seguir estradas. Em nosso caso, as
linhas são sobrepostas sobre \emph{bundles}, que foram gerados unicamente a partir dos
dados da OD. Pode-se argumentar que RAEB, neste sentido, produz \emph{bundles}
mais ``corretos'', uma vez que estes são forçados para seguir as estradas. No
entanto, olhando mais de perto, podemos ver que RAEB não pode ter todos os
\emph{bundles} seguindo precisamente todos os caminhos das estradas - pois isso
basicamente bloquearia qualquer agrupamento do \emph{bundling} e resultaria no próprio
mapa das estradas. Além disso, RAEB requer que o registro dos pontos das trajetórias
seja feito dentro de uma rede rodoviária precisa. Isso torna-o
significativamente mais complexo para implementar e mais caro para processar do
que nossa solução baseada em \emph{CUBu}.

\begin{figure}[!htb]
  \centering
  \captionsetup{justification=centering}
  \includegraphics[width=0.98\textwidth]{../figuras/rail-lines.png}
    \caption{\emph{Bundling} das trajetórias coloridas pela densidade e a malha ferroviária da RMSP}
  \label{fig:rails}  
\end{figure}

\section{Mapeando distância e direção no \emph{bundling}}
\label{sec:length-direction}

Para explorar a mobilidade urbana de diferentes perspectivas, precisamos de
meios para visualizar seus múltiplos atributos de dados. No entanto, esta
variedade de atributos requer diferentes estratégias de visualização. Dois
importantes atributos para o estudo de padrões de mobilidade são distância
percorrida na viagem e a sua direção. As Figuras~\ref{fig:attributes-length} e
\ref{fig:attributes-direction} mostram a visualização de todo o conjunto de
dados OD17 mapeando a distância e direção, respectivamente.

A Figura~\ref{fig:attributes-length} exibe os comprimentos das viagens
codificados por cores. Nela utilizamos o mesmo mapa de cores arco-íris da
Figura~\ref{fig:rails} e também a modulação da transparência por densidade,
conforme explicado na Seção~\ref{sec:density}. Nesta imagem, podemos observar
uma única curva vermelha aparentemente na horizontal. Sua alta opacidade implica
que há muitas viagens longas, todas mapeadas perfeitamente para essa trajetória
entre a mesma origem e destino (se não o fizessem, veríamos um \emph{bundle} se
ramificando no formato de um leque em vez de uma curva precisa). Esta é uma
descoberta interessante que, argumentamos, não poderia ser facilmente encontrada
usando métodos não visuais. Apesar desse ponto incomum, as outras trajetórias,
em geral, percorrem distâncias regulares. \emph{Bundles} de longa distância como
este podem indicar falta de serviços ou recursos que não satisfazem as regiões
locais, obrigando as pessoas a percorrerem longas distâncias para acessá-los. A
pesquisa OD17 contém mais informações que podem ajudar a investigar o motivo
dessas longas viagens.

A Figura~\ref{fig:attributes-direction} mostra os mesmos dados da
Figura~\ref{fig:attributes-length}, mas ao invés da distância são as direções
das viagens que estão codificadas em cores. Para este atributo em específico,
usamos ainda um recurso do \emph{CUBu}, que separa trilhas em direções opostas
em dois \emph{bundles} quase paralelos. Podemos ver claramente a existência de
trajetórias paralelas ao longo dos \emph{bundles}, o que não é surpreendente
porque a pesquisa OD registra o trajeto típico das pessoas que inclui os
deslocamentos de ida e vinda de volta para suas origens. No entanto, essa
simetria das trajetórias possivelmente não seria observada se analisássemos um
curto período do dia.

\begin{figure}[!htb] \centering \captionsetup{justification=centering}
  \includegraphics[width=0.98\textwidth]{../figuras/distances.png}
  \caption{Mapeamento da distância das viagens em cores. \label{fig:attributes-length}}
  \end{figure}

\begin{figure}[!htb]
  \centering
  \captionsetup{justification=centering}
  \includegraphics[width=0.98\textwidth]{../figuras/directions.png}
  \caption{Mapeamento da direção das viagens em cores. \label{fig:attributes-direction}}
\end{figure}

\section{Coloring transportation modes: local \emph{vs} intercity buses}
\label{sec:coloring}
% %
% The OD17 dataset contains 17 transportation modes. While it would be ideal to be able to see the 17 categories all at once in our bundled visualization, that would not be easy to do, since it would require the simultaneous encoding of 17 different categorical attributes. Instead, we use transparency to hide trails according to a user-set selector that filters them by transportation mode. Figure~\ref{fig:bus-integration} shows how we can use these filters to visualize the integration between buses from the city of S\~ao Paulo (local buses) and intercity buses. %Unbundled and bundled trajectories are presented on the left and right sides respectively.
% Each transportation mode has a distinct color -- olive for local buses and blue for intercity buses.

% Figure~\ref{fig:bus-integration} also highlights that these different transportation systems appear to complement each other. The city of S\~ao Paulo has a very active commerce and industry, and many people from neighboring cities work there. Thus, the availability of public transportation and its integration is very important for these people. This kind of filtering along with bundling helps to better understand correlations between data attributes -- in this case, \hbox{transportation}~modes.


% \begin{figure}[!htb]
%   \centering
%   \captionsetup{justification=centering}
%   \includegraphics[width=0.98\textwidth]{figures/local-intercity-buses}
%   \caption{Edges filtered by transportation modes: bundled trails of local and intercity buses. \label{fig:bus-integration}}
% \end{figure}

\section{Density per social strata}
\label{sec:strata}
% % 6 imagens, uma de cada classe social, density1
% We used our bundle visualization to study how citizens with different economical conditions commute in the SPMA. The Brazilian Economical Classification Criterion (BECC)\,\cite{cceb2008} is the official social-economic index used in the Brazilian Demographic Census, which is performed by the Brazilian Institute of Geography and Statistics. It measures the purchasing power of the Brazilian society. The BECC is divided into six levels or strata (Table~\ref{tab:becc}). This index is used in the OD17 survey to complement the mobility data. Table~\ref{tab:becc} also shows the average monthly income in the local currency (Brazilian reals) and in US dollars, and the number of trips in the SPMA for each BECC level considering the whole population and only citizens with age between 6 and 18 years that commute for study (see Section~\ref{sec:students}).

% To compare the mobility patterns of different BECC social strata, we bundled the trails in each stratum separately, as shown in Figures~\ref{fig:becc-axd-e}~to~\ref{fig:becc-d-e}. We see significant differences in the mobility patterns between the highest and lowest income levels as shown in Figure~\ref{fig:becc-axd-e}. The $A$ level (on the left-hand side) has a high density in the center of SPMA, which includes the capital downtown surroundings. The highest density is located in the west, southwest, and northeast neighborhoods near downtown. There are density flows between the capital and the cities of Barueri and Cotia, which have high-income residential areas. There are other high dense flows linking the capital to the cities of S\~ao Bernardo do Campo and Santo Andr\'e. Comparing $A$ to the $D$-$E$ level (on the right-hand side of Figure~\ref{fig:becc-axd-e}), we see that $D$-$E$ has the highest dense flows in the capital eastern region. In the $D$-$E$ level map, we can see the absence of high-dense flows in regions that are nearest to the capital downtown; in contrast, these are present in the $A$ level map. We can see more details of $A$ and $D$-$E$ strata in Figures~\ref{fig:becc-a}~and~\ref{fig:becc-d-e}

% \begin{figure}[!htb]
%   \centering
%   \captionsetup{justification=centering}
%   \includegraphics[width=0.98\textwidth]{figures/comparison-axd-e-strata-leg.png}
%   \caption{Comparing density of trails between social strata $A$ (left-hand side) and $D$-$E$ (right-hand side). \label{fig:becc-axd-e}}
% \end{figure}


% \begin{figure}[!htb]
%   \centering
%   \captionsetup{justification=centering}
%   \includegraphics[width=0.98\textwidth]{figures/1-class-a.png}
%   \caption{Density of trails of social stratum $A$. \label{fig:becc-a}}
% \end{figure}

% \begin{figure}[!htb]
%   \centering
%   \captionsetup{justification=centering}
%   \includegraphics[width=0.98\textwidth]{figures/2-class-b1.png}
%   \caption{Density of trails of social stratum $B1$. \label{fig:becc-b1}}
% \end{figure}

% \begin{figure}[!htb]
%   \centering
%   \captionsetup{justification=centering}
%   \includegraphics[width=0.98\textwidth]{figures/3-class-b2.png}
%   \caption{Density of trails of social stratum $B2$. \label{fig:becc-b2}}
% \end{figure}

% \begin{figure}[!htb]
%   \centering
%   \captionsetup{justification=centering}
%   \includegraphics[width=0.98\textwidth]{figures/4-class-c1.png}
%   \caption{Density of trails of social stratum $C1$. \label{fig:becc-c1}}
% \end{figure}

% \begin{figure}[!htb]
%   \centering
%   \captionsetup{justification=centering}
%   \includegraphics[width=0.98\textwidth]{figures/5-class-c2.png}
%   \caption{Density of trails of social stratum $C2$. \label{fig:becc-c2}}
% \end{figure}

% \begin{figure}[!htb]
%   \centering
%   \captionsetup{justification=centering}
%   \includegraphics[width=0.98\textwidth]{figures/6-class-d-e.png}
%   \caption{Density of trails of social strata $D$-$E$. \label{fig:becc-d-e}}
% \end{figure}

% When we compare all maps from the $A$ to the $D$-$E$ level (Figures~\ref{fig:becc-a}~to~\ref{fig:becc-d-e}), we see that the densest flows (red) tend to displace from the capital downtown to the eastern region of the city. The concentration of high-density flows is increasingly spreading from the center to the peripheral regions of the SPMA. Even the less dense flows are increasing and spreading over the SPMA. However, the $D$-$E$ map shows that those flows diminish considerably for these social strata. This may indicate that low-income citizens have less access to the urban mobility system. As a consequence, these people would have less access to the social, educational, health, and cultural services of the SPMA, as those facilities are concentrated in the center regions of the cities. It is worthy to note that those central regions also have more job opportunities. Looking at the $D$-$E$ map, we can see a ``hole" in the capital west downtown. This region (Pinheiros district) concentrates a large number of jobs related to information technology and financial services, which requires workers with high and medium education levels. Thus, the map shows that low-income citizens are not going to that region, which reflects the inequality of opportunities that these citizens face.

% \begin{table}[!htb]
%   \small
%   \newcommand{\hdr}[1]{\bfseries#1}
%   \centering
%   \caption{Trips grouped per BECC income level, social stratum, and traveler age.\label{tab:becc}}
%   \begin{tabular}{>{\footnotesize}c>{\footnotesize}r>{\footnotesize}r>{\footnotesize}r>{\footnotesize}r}
%     \toprule
%     \multirow{2}[2]{*}{\hdr{BECC level}} & \hdr{Monthly income} & \hdr{Monthly income}& \multirow{2}[2]{*}{\hdr{Trips}} & \hdr{Trips of 6 to 18}\\
%     & \hdr{(Brazilian reals (R\$))} & \hdr{(US dollars ($\sim$US\$))} & & \hdr{years old students}\\
%     \midrule
%     A   & 23,345    & 4,245 & 3,062,892  &   184,772\\
%     B1  & 10,386    & 1,888 & 3,854,040  &   260,652\\
%     B2  & 5,363     & 975   & 12,856,182 &   963,242\\
%     C1  & 2,965     & 539   & 11,277,159 &   976,745\\
%     C2  & 1,691     & 307   & 7,852,806  &   721,218\\
%     D-E & 708       & 128   & 2,233,801  &   219,612\\
%     \bottomrule
%   \end{tabular}
% \end{table}

\section{Mobility of young students from different social strata}
\label{sec:students}
% % od17-escola-alta-renda-6-18, od17-escola-baixa-renda-6-18 density1
% To explore even more the mobility patterns showed bundling visualizations, we compared the trips of students from different social strata. We filtered citizens with age between 6 and 18 years whose commuting reason is study. We split them into two groups, the high- to moderate-income, which includes the BECC levels $A$, $B1$, $B2$, and $C1$; and the low-income, which includes levels $C2$ and $D$-$E$. Figures~\ref{fig:students-high}~and~\ref{fig:students-low} show the density maps for both groups.

% \begin{figure}[!htb]
%   \centering
%   \captionsetup{justification=centering}
%   \includegraphics[width=0.98\textwidth]{figures/high-income-density-leg.png}
%   \caption{Density of trails of young students from high-income households. \label{fig:students-high}}
% %\end{figure}
% \vspace*{\floatsep}
% %\begin{figure}[!htb]
%   \centering
%   \captionsetup{justification=centering}
%   \includegraphics[width=0.98\textwidth]{figures/low-income-density-leg.png}
%   \caption{Density of trails of young students from low-income households. \label{fig:students-low}}
% \end{figure}

% The density map of the high- to moderate-income students (Figure~\ref{fig:students-high}) shows a large number of dense flows spread across the central region of SPMA. This part of SPMA concentrates most private schools, universities, and complementary colleges. In addition, high density flows are not as long as flows from other maps with all the data (e.g., Figure~\ref{fig:bundled-graph-density}). This indicates that trips to study are shorter than trips to work. 

% The density map of low-income students (Figure~\ref{fig:students-low}) shows that their mobility is very limited compared to the higher-income students. There are a few dense flows, most of them out of the capital downtown. The high density flows of low-income students are more present in the peripherals of the capital and also in the neighboring cities. There is a concentration of both groups in the southwest region, where are the neighborhoods of the Campo Limpo district. 

% It is worthy to note that the public schools in the SPMA are spread across the central and peripheral parts of the cities. The students are enrolled in these schools according to the proximity of their residences. Thus, they do not have to travel long distances to reach their schools. Also, public schools have lower educational performance than private schools in S\~ao Paulo. Thus, citizens with better financial conditions use to put their children in private schools.

% The scarcity of flows from low-income students may indicate that they do not have equal opportunities to study, being enrolled in schools in their neighborhood. They also do not use to go to the central region of the city and, thus, have less access to universities and complementary colleges. This inequality of opportunities will probably impact these students' jobs and economical conditions.

% We also see that there are many more trails for the high- to moderate-income students (Figure~\ref{fig:students-high}) than for low-income students (Figure~\ref{fig:students-low}). The high-to moderate-income students also travel large distances to study, which indicates that they can choose more flexibly where to study. This fact is corroborated by urban mobility studies that indicate that people with better financial conditions have more mobility than those with poorest conditions\,\cite{carruthers2005,lucas2016}.

\section{Directions at peak hours}
\label{sec:peak-hours}
% %
% As discussed earlier in Sec.~\ref{sec:data}, Figure~\ref{fig:trips_by_hour} shows the distribution of trips by hour of the day, with two main rush-hour peaks (6-9 AM and 5-8 PM). 
% However, this aggregated table does not give us insights in how the rush-hour patterns may differ. To see this, we selected the two rush-hour time intervals and visualized them separately, using directional bundling and color-coding.

% \begin{figure}[!htb]
%   \centering
%   \captionsetup{justification=centering}
%   \includegraphics[width=0.98\textwidth]{figures/peak-hours-6h-to-9h-direction-leg.png}
%   \caption{Directions of trips between 6 to 9 AM \label{fig:peak-hours-6h-9h}}
% %\end{figure}
% \vspace*{\floatsep}
% %\begin{figure}[!htb]
%   \centering
%   \captionsetup{justification=centering}
%   \includegraphics[width=0.98\textwidth]{figures/peak-hours-17h-to-20h-direction-leg.png}
%   \caption{Directions of trips between 5 to 8 PM \label{fig:peak-hours-17h-20h}}
% \end{figure}

% Comparing the peak hours, we can see that morning flows going to the SPMA center (Figure~\ref{fig:peak-hours-6h-9h}, cyan bundle)
% are overall denser and longer than the flows coming from the SPMA center during the afternoon/evening peak (Figure~\ref{fig:peak-hours-17h-20h}, red). This suggests that in the morning people are in a hurry to reach their work, while they are less in a hurry to go back home (or to other destinations like schools or the gym) in the afternoon/evening.

% In Figure~\ref{fig:peak-hours-6h-9h}, we can see that flows in the morning peak going to the capital downtown (cyan bundle coming from the east) are denser than opposite flows (red bundle going to the east). Although flows leaving the capital downtown in the morning are thinner than their opposite ones, they also concentrate a large number of trips, especially to the east and southwest. In Figure~\ref{fig:peak-hours-17h-20h}, the opposite flows seem more equally distributed.

\section{Density by transportation mode}
\label{sec:mode}
% %% on foot, bike, car, metro (sliced, density)
% We next split the OD17 data by transportation mode to compare the flow patterns for four different transportation modes: pedestrians, bicycles, cars, and subway. Figures~\ref{fig:mode-pedestrian}~to~\ref{fig:mode-subway} show the respective visualizations.

% \begin{figure}[!htb]
%   \centering
%   \captionsetup{justification=centering}
%   \includegraphics[width=0.98\textwidth]{figures/mode-pedestrian-density-leg.png}
%   \caption{Density of pedestrian trips \label{fig:mode-pedestrian}}
% \end{figure}

% \begin{figure}[!htb]
%   \centering
%   \captionsetup{justification=centering}
%   \includegraphics[width=0.98\textwidth]{figures/mode-bike-density-leg.png}
%   \caption{Density of bicycle trips \label{fig:mode-bike}}
% \end{figure}

% \begin{figure}[!htb]
%   \centering
%   \captionsetup{justification=centering}
%   \includegraphics[width=0.98\textwidth]{figures/mode-car-density-leg.png}
%   \caption{Density of car trips \label{fig:mode-car}}
% \end{figure}

% \begin{figure}[!htb]
%   \centering
%   \captionsetup{justification=centering}
%   \includegraphics[width=0.98\textwidth]{figures/mode-subway-density-leg.png}
%   \caption{Density of subway trips \label{fig:mode-subway}}
% \end{figure}

% Pedestrian trails (Figure~\ref{fig:mode-pedestrian}) form several low-density `islands' spread across the SPMA, with the densest one (red in figure) being in the capital downtown. Most trails are quite short, which is expected (pedestrians). However, we see a few longer bundles between the capital downtown and the south and north regions of the city. Dense flows are also present in the neighboring cities of Diadema, Tabo\~ao da Serra, Osasco, Guarulhos, Po\'a, and Mogi das Cruzes. Upon examination, we found these dense flows to match the cities' downtown and commercial areas. This information could be useful to find places that could deserve the attention of local governments to provide improvements for pedestrians.

% As most of the pedestrian trips are short, the bundling technique forms a few flows over the SPMA.
% Using bundling for those short trips result in low-density trails, which is less useful compared to long trips. Thus, in these cases it may not be necessary to use bundling. In the upper left area of Figure~\ref{fig:mode-pedestrian}, we can see the OD trails without using bundling, which are near identical to the main bundled area.
% %For pedestrian trails, it would be interesting to apply bundling over small areas. \textbf{AT: Actually, you can do this by running CUBu simply using a much smaller kernel size. But, if we do not do that, I think we should omit the above sentence.}

% Bicycle trips (Figure~\ref{fig:mode-bike}) exhibit similar patterns to pedestrian ones. They are shorter than three kilometers on average. In this figure, we see some thin flows in the capital downtown area. There are also some more salient flows in the capital northeast and in the neighboring cities of Suzano and Guarulhos.
% However, comparing Figure~\ref{fig:mode-bike} with all other transportation means, we immediately see that bicycle trips are by far the least numerous, and exhibit a far sparser pattern, with few star-shaped `hubs' where many trails meet. This suggests that the cycling infrastructure is quite limited, and fragmented. Figure~\ref{fig:mode-bike} also shows trails without using bundling in the upper left corner.

% The car trips (Figure~\ref{fig:mode-car}) show a pattern similar to the one displaying the entire dataset, \emph{i.e.}, all transportation modes (see \emph{e.g.} Figure~\ref{fig:bundled-graph-density}). For a start, this tells that cars are \emph{the} dominant form of transportation in the SPMA, accounting for the main traffic patterns. The highest-density flows occur in the capital downtown. There are several high-density flows linking the downtown area to the other regions of the capital, and also coming and going from the cities of Guarulhos, Barueri, Cotia, S\~ao Bernardo do Campo, Santo Andr\'e, Mau\'a, and Mogi das Cruzes. Compared to all other transportation modes, cars show a far more `spread out' pattern that covers very large areas, indicating that cars are the prevalent transportation mode in most parts of the SPMA.

% Finally, subway trips (Figure~\ref{fig:mode-subway}) show a strong star-shaped pattern, with very high density bundles that connect the capital with the neighboring cities, due to the integration of the subway system with the train system. Compared to all other transportation modes, subways show a clearer, simpler, trip pattern structure.

\section{Different trip reasons}
\label{sec:dist_reasons}
% We next aim to study whether trips done for different reasons exhibit distinct trip patterns. For this, we create bundled visualizations from the OD17 data with trips grouped by work, health, education, and shopping. Figures~\ref{fig:reason-work}~to~\ref{fig:reason-shopping} show the results.

% Work-related trips (Figure~\ref{fig:reason-work}) are overall longer than the other trip reasons, and also cover a larger area (see the central agglomeration in the figure). Interestingly, the longest trips, between the east side and the city center (red bundle), are similar in pattern to the longest trips for health and education. 
% Trips for health reasons are sparser than work-related ones, and also show a more star-like pattern, with long bundles connecting to the central area. This may indicate that peripheral regions are not well served by health services. Trips for studying reasons (Figure~\ref{fig:reason-education}) have the largest distances between the northeast and the western regions of the SPMA. Their pattern is somewhere in-between the work and health trips. Interestingly, education trips show several `loops' in the center of the SPMA. Finally, shopping trips (Figure~\ref{fig:reason-shopping}) show the least dense, and overall also shortest, patterns, apart from a few outliers like the red (important) bundle connecting the center to the northeast. This tells that, unlike health, education, and work, shopping facilities (which are actually provided by private companies) are better distributed over the SPMA. This outlines that bundled visualizations are useful not only when they show the \emph{presence} of certain data, \emph{e.g.} trails linking far-apart regions; the \emph{absence} of patterns is also insightful, as in the case of the lack of long shopping trips.

% \begin{figure}[!htb]
% \centering
% \captionsetup{justification=centering}
% \includegraphics[width=0.98\textwidth]{figures/reason-work-leg.png}
% \caption{Distance of trips for work reasons.\label{fig:reason-work}}
% \end{figure}
  
% \begin{figure}[!htb]
% \centering
% \captionsetup{justification=centering}
% \includegraphics[width=0.98\textwidth]{figures/reason-health-leg.png}
% \caption{Distance of trips for health-related reasons.\label{fig:reason-health}}
% \end{figure}

% \begin{figure}[!htb]
% \centering
% \captionsetup{justification=centering}
% \includegraphics[width=0.98\textwidth]{figures/reason-school-leg.png}
% \caption{Distance of trips for education reasons.\label{fig:reason-education}}
% \end{figure}

% \begin{figure}[!htb]
% \centering
% \captionsetup{justification=centering}
% \includegraphics[width=0.98\textwidth]{figures/reason-shopping-leg.png}
% \caption{Distance of trips for shopping reasons.\label{fig:reason-shopping}}
% \end{figure}



\chapter{Considerações Finais}
\label{cap:plano-de-trabalho}

Neste trabalho, exploramos o uso do \emph{bundling} para criar
visualizações de vários atributos dos dados de mobilidade urbana da Região
Metropolitana de São Paulo. Nossas análises sobre as características da pesquisa
OD 2017 mostram que o \emph{bundling} pode ser usado para identificar e comparar
diferentes padrões de mobilidade implícitos em diferentes subconjuntos de dados
e subconjuntos de atributos disponíveis. Combinando adequadamente a filtragem
(para reduzir a quantidade de dados e/ou atributos a serem explorados) com o
\emph{bundling} (para simplificar as visualizações criadas e reduzir a oclusão
visual) e com os canais visuais disponíveis (opacidade, cor, direção),
destacamos diferentes padrões no conjunto de dados OD17 que não teriam sido
facilmente obtido por ferramentas clássicas de mineração e visualização de
dados.

Nossos resultados sobre a visualização das trajetórias com \emph{bundling}
destacaram sua estrutura centralizada sobre a área estudada da RMSP. Além disso,
essa estrutura condiz com a infraestrutura metroviária e ferroviária de São
Paulo. Embora não tenha sido uma surpresa, a correlação sugere que nossos
parâmetros foram bem ajustados para a visualização na escala metropolitana.
Nossa metodologia para reduzir a complexidade do conjunto de dados de 42 milhões
de viagens para menos de um milhão, e também nossa adaptação de um
\emph{framework} de uso geral (\emph{CUBu}) para agrupar subconjuntos específicos de
dados e/ou atributos foram os pontos principais que tornaram essa análise
possível. Desta forma conseguimos responder a nossa questão de pesquisa, pois
conseguimos obter um método de visualização de grandes conjuntos de dados de
mobilidade sob diferentes perspectivas e explorar seus múltiplos atributos.

Como trabalho futuro, pretendemos explorar melhorias no uso e usabilidade de
visualizações agregadas com \emph{bundling}. Do ponto de vista da visualização,
melhorias no mapa para mostrar as divisões das regiões no topo das trajetórias,
conforme proposto em \citet{Klein2014}, podem ajudar a identificar melhor as
conexões entre as regiões. Além disso, seria interessante testar uma abordagem
diferente para transmitir a informação de densidade dos \emph{bundles} alterando
a espessura das trajetórias proporcionalmente ao fator de expansão de cada
registro da OD17 em vez de usar cores, de forma semelhante a
\citet{lhuillier-fft:17}. Desta maneira não haveria a necessidade de replicar as
trajetórias como fizemos e, portanto, isso reduziria significativamente o
tamanho do conjunto de dados. Essas são etapas importantes para permitir o uso
de \emph{bundling} para análise em tempo real.

Da perspectiva da aplicação da técnica de \emph{bundling}, existem ainda muitas
outras possibilidades para análise da mobilidade urbana usando dados da própria
pesquisa OD e ainda a possibilidades de agregar outros conjuntos de dados, como
aqueles de empresas privadas de mobilidade, dispositivos IoT e sistemas de
compartilhamento de bicicletas. Há ainda espaço para se fazer uma análise
temporal mais profunda comparando a evolução da mobilidade urbana utilizando
dados das pesquisas OD anteriores. Por último, mas não menos importante, também
seria interessante complementar os resultados de nosso estudo com a avaliação de
usuários reais, como gestores de tráfego, urbanistas e planejadores urbanos,
para verificar a utilidade das visualizações com \emph{bundling}.

% \section{Esforço Técnico}

% As colaborações deste trabalho envolveram também resultados técnic





\par
