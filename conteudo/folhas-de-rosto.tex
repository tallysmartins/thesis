%!TeX root=../tese.tex
%("dica" para o editor de texto: este arquivo é parte de um documento maior)
% para saber mais: https://tex.stackexchange.com/q/78101/183146

% Apague as duas linhas abaixo (elas servem apenas para gerar um
% aviso no arquivo PDF quando não há nenhum dado a imprimir) e
% copie, com as alterações necessárias, o conteúdo do arquivo
% conteudo-exemplo/folhas-de-rosto.tex
%%%%%%%%%%%%%%%%%%%%%%%%%%%%%%%%%%%%%%%%%%%%%%%%%%%%%%%%%%%%%%%%%%%%%%%%%%%%%%%%
%%%%%%%%%%%%%%%%%%%%%%%%%%%%% METADADOS DA TESE %%%%%%%%%%%%%%%%%%%%%%%%%%%%%%%%
%%%%%%%%%%%%%%%%%%%%%%%%%%%%%%%%%%%%%%%%%%%%%%%%%%%%%%%%%%%%%%%%%%%%%%%%%%%%%%%%

% Define o texto da capa e da referência que vai na página do resumo;
% "masc" ou "fem" definem se serão usadas palavras no masculino ou feminino
% (Mestre/Mestra, Doutor/Doutora, candidato/candidata). O segundo parâmetro
% é opcional e determina que se trata de exame de qualificação.
\mestrado[masc]
%\mestrado[fem][quali]
%\doutorado[masc]
%\doutorado[masc][quali]

% Se "\title" está em inglês, você pode definir o título em português aqui
\tituloport{Visualização de Fluxos de Mobilidade Urbana com \emph{Bundling}}

% Se "\title" está em português, você pode definir o título em inglês aqui
\tituloeng{Visualizing Urban Mobility Structure with Bundling}

% Se o trabalho não tiver subtítulo, basta remover isto.
% \subtitulo{um subtítulo}

% Se isto não for definido, "\subtitulo" é utilizado no lugar
% \subtituloeng{a subtitle}

\orientador[masc]{Prof. Dr. Fabio Kon}

% Se isto não for definido, "\orientador" é utilizado no lugar
\orientadoreng{Prof. Dr. Fabio Kon}

% Se não houver, remova
% \coorientador[masc]{Dr. Higor Amario de Souza}

% Se isto não for definido, "\coorientador" é utilizado no lugar
%\coorientadoreng{Prof. Dr. Ciclano}

\programa{Ciência da Computação}

% Se isto não for definido, "\programa" é utilizado no lugar
\programaeng{Computer Science}

% Se não houver, remova
\apoio{Esta pesquisa foi desenvolvida no contexto do INCT da Internet do Futuro
para Cidades Inteligentes com apoio do CNPq proc. 465446/2014-0, Coordenação de
Aperfeiçoamento de Pessoal de Nível Superior – Brasil (CAPES) – código
financeiro 001 e FAPESP procs. 14/50937-1 e 15/24485-9.}

% Se isto não for definido, "\apoio" é utilizado no lugar
% \apoioeng{During this work, the author was supported by XXX}

\localdefesa{São Paulo}

\datadefesa{25 de Janeiro de 2021}

% Se isto não for definido, "\datadefesa" é utilizado no lugar
% \datadefesaeng{August 10th, 2017}

% Necessário para criar a referência do documento que aparece
% na página do resumo
\ano{2019}

\banca{
  \begin{itemize}
    \item Profª. Drª. Nome Completo (orientadora) - IME-USP [sem ponto final]
    \item Prof. Dr. Nome Completo - IME-USP [sem ponto final]
    \item Prof. Dr. Nome Completo - IMPA [sem ponto final]
  \end{itemize}
}

% Se isto não for definido, "\banca" é utilizado no lugar
\bancaeng{
  \begin{itemize}
    \item Prof. Dr. Nome Completo (advisor) - IME-USP [sem ponto final]
    \item Prof. Dr. Nome Completo - IME-USP [sem ponto final]
    \item Prof. Dr. Nome Completo - IMPA [sem ponto final]
  \end{itemize}
}

% Palavras-chave separadas por ponto e finalizadas também com ponto.
\palavraschave{\emph{Bundling}, visualização de dados, mobilidade urbana, dados de mobilidade, cidades inteligentes.}

\keywords{\emph{Bundling}, data visualization, urban mobility, mobility data, smart cities.}

% Se quiser estabelecer regras diferentes, converse com seu
% orientador
\direitos{Autorizo a reprodução e divulgação total ou parcial
deste trabalho, por qualquer meio convencional ou
eletrônico, para fins de estudo e pesquisa, desde que
citada a fonte.}

% Isto deve ser preparado em conjunto com o bibliotecário
%\fichacatalografica{
% nome do autor, título, etc.
%}

%%%%%%%%%%%%%%%%%%%%%%%%%%% CAPA E FOLHAS DE ROSTO %%%%%%%%%%%%%%%%%%%%%%%%%%%%%

% Embora as páginas iniciais *pareçam* não ter numeração, a numeração existe,
% só não é impressa. O comando \mainmatter (mais abaixo) reinicia a contagem
% de páginas e elas passam a ser impressas. Isso significa que existem duas
% páginas com o número "1": a capa e a página do primeiro capítulo. O pacote
% hyperref não lida bem com essa situação. Assim, vamos desabilitar hyperlinks
% para números de páginas no início do documento e reabilitar mais adiante.
\hypersetup{pageanchor=false}

% A capa; o parâmetro pode ser "port" ou "eng" para definir a língua
\capaime[port]
%\capaime[eng]

% Se você não quiser usar a capa padrão, você pode criar uma outra
% capa manualmente ou em um programa diferente. No segundo caso, é só
% importar a capa como uma página adicional usando o pacote pdfpages.
%\includepdf{./arquivo_da_capa.pdf}

% A página de rosto da versão para depósito (ou seja, a versão final
% antes da defesa) deve ser diferente da página de rosto da versão
% definitiva (ou seja, a versão final após a incorporação das sugestões
% da banca). Os parâmetros podem ser "port/eng" para a língua e
% "provisoria/definitiva" para o tipo de página de rosto.
%\pagrostoime[port]{definitiva}
\pagrostoime[port]{provisoria}
%\pagrostoime[eng]{definitiva}
%\pagrostoime[eng]{provisoria}

%%%%%%%%%%%%%%%%%%%% DEDICATÓRIA, RESUMO, AGRADECIMENTOS %%%%%%%%%%%%%%%%%%%%%%%

% A definição deste ambiente está no pacote imeusp; se você não
% carregar esse pacote, precisa cuidar desta página manualmente.
%\begin{dedicatoria}
%Esta seção é opcional e fica numa página separada; ela pode ser usada para
%uma dedicatória ou epígrafe.
%\end{dedicatoria}

% Após a capa e as páginas de rosto, começamos a numerar as páginas; com isso,
% podemos também reabilitar links para números de páginas no pacote hyperref.
% Isso porque, embora contagem de páginas aqui começe em 1 e no primeiro
% capítulo também, o fato de uma numeração usar algarismos romanos e a outra
% algarismos arábicos é suficiente para evitar problemas.
\pagenumbering{roman}
\hypersetup{pageanchor=true}

% Agradecimentos:
% Se o candidato não quer fazer agradecimentos, deve simplesmente eliminar
% esta página. A epígrafe, obviamente, é opcional; é possível colocar
% epígrafes em todos os capítulos. O comando "\chapter*" faz esta seção
% não ser incluída no sumário.
\chapter*{Agradecimentos}
\epigrafe{Do. Or do not. There is no try.}{Mestre Yoda}

Um trabalho de pesquisa sempre é fruto direta ou indiretamente da colaboração de
muitas pessoas. Agradecê-las nominalmente é um risco por ser possível deixar
indevidamente alguém de fora. Assim, deixo registrado um sincero agradecimento a
todas e todos que contribuíram para essa minha conquista. Minha família, crucial
e indispensável nesse período, me deu todas as forças de que eu precisei. Um
imenso obrigado do fundo da minha alma.

À senhora Dona Marieta e senhor Chagas, agradeço pelos incentivos e cuidados
para com a minha saúde, pela preocupação com meu sucesso e principalmente pelo
apoio na fase final, quando cheguei a duvidar de mim mesmo. Vocês são anjos na
minha vida. 

Ao meu querido mentor da vida, Paulo Meirelles, cujas palavras não são
suficientes para expressar o quão grato eu sou pela sua ajuda em vários
momentos. Seus conselhos sempre clareiam meus caminhos.

Ao Higor Amário, meu grande guia neste trabalho, agradeço imensamente pelo forte
apoio técnico adentro das várias discussões sobre o tema, pela dedicação e
orientações, indispensáveis para o andamento deste trabalho. Sou profundamente
agradecido por toda sua doação e empenho a este trabalho.

Ao professor Alexandru Telea, agradeço pela enorme atenção e disponibilidade ao
esclarecer todas as minhas dúvidas, por sua prestatividade ao colaborar com o
desenvolvimento deste trabalho e nos fornecer toda a base necessária. Você foi
diretamente responsável para que essa pesquisa se materializasse e gerasse bons
frutos. Obrigado.

Ao Nelson Lago, agradeço por me agraciar com inúmeras doses de conhecimento
sobre os mais variados temas, por me encher-me de exemplos quando não entendia
um assunto qualquer e por estar sempre disponível para nos socorrer quando a
impressora do laboratório emperrava. É uma honra dividir esse tempo e espaço com
você.

Aos meus queridos amigos que a computação me proporcionou, Arthur, Athos, Dylan,
Lucas K., Lucas M., Marcos, Rodrigo Siqueira, agradeço por todo o companheirismo
nessa jornada, dividindo cada dor e cada glória. Com vocês ao redor tudo se
tornou mais fácil.

À Marina Guedes, Lorrayne, Chico, Meire, Renata e Kaccnny, um imenso e especial
agradecimento por cada momento de companheirismo e por todas as vezes que
nos deliciamos em divertidos almoços e jantares. Obrigado por
dividirem comigo essa aventura de sonhar em São Paulo.

Aos professores Alfredo, Daniel e Kelly, membros do grupo de pesquisa de
sistemas, agradeço por cada momento de informação durante as nossas reuniões.
Pouco a pouco, a partir daqueles momentos, foi se construindo o que hoje se
tornou formalmente meu projeto de pesquisa.

Finalmente, agradeço meu grande orientador Fabio Kon, que foi um grande pivô
nesta batalha e não mediu esforços para alcançarmos nossos objetivos. Obrigado
por cada puxão de orelhas, por se empenhar em prol da ciência brasileira e por
ser uma pessoa de grande coração. Vou lembrar-me de cada apresentação, cada vez
que tive que me esforçar no inglês, cada artigo. Fizemos uma longa jornada e
agradeço por me ajudar a escalar mais um degrau na vida.
%
%Texto texto texto texto texto texto texto texto texto texto texto texto texto
%texto texto texto texto texto texto texto texto texto texto texto texto texto
%texto texto texto texto texto texto texto texto texto texto texto texto texto
%texto texto texto texto. Texto opcional.

%<<<<<<< HEAD
%!TeX root=../tese.tex
%("dica" para o editor de texto: este arquivo é parte de um documento maior)
% para saber mais: https://tex.stackexchange.com/q/78101/183146

% Apague as duas linhas abaixo (elas servem apenas para gerar um
% aviso no arquivo PDF quando não há nenhum dado a imprimir) e
% insira aqui o conteúdo do resumo e abstract do seu trabalho
=======
\begin{resumo}{port}
  Nos últimos anos, houve um crescimento na geração e disponibilização de uma
variedade de dados de mobilidade, como carros, ônibus, aviões, embarcações,
cujo conteúdo revela a dinâmica de fluxo entre diferentes regiões espaciais. No
trânsito, a posição dos veículos é mapeada em registros 2D, geralmente latitude
e longitude, registrando sua trajetória ao longo do tempo. A análise dessas
trajetórias tem diversas aplicações, como detecção de padrões de
origem-destino, monitoramento do fluxo ou a observação de eventos externos,
como mudanças climáticas, e seus impactos no tráfego. Com isso, gestores de uma
cidade conseguem mais informações para auxiliar na tomada de decisão e
planejamento do trânsito. Recentemente, técnicas avançadas para a análise de
trajetórias surgiram, permitindo a visualização de grandes conjuntos de dados
em diferentes níveis de detalhes. Uma dessas técnicas, o \emph{bundling},  ajuda a
reduzir a oclusão na visualização de trajetórias através do agrupamento de
segmentos espacialmente próximos, possibilitando que uma representação visual
comunique de maneira mais clara e intuitiva os padrões geográficos de
origem-destino desses objetos. Neste trabalho exploraremos como o \emph{bundling} pode
ser usado na análise e identificação de padrões de deslocamento de fluxos
provenientes de diferentes regiões de uma cidade e como esses fluxos impactam
em ruas e avenidas com grandes congestionamentos. Para isso, iremos aplicar
algoritmos de \emph{bundling} que favorecem a análise multiescala para obter uma
visão macroscópica e microscópica dos fluxos de deslocamento. Além disso,
investigaremos como a técnica ajuda na comparação da estrutura do tráfego em
condições diversas, como em diferentes dias de semana, dias de chuva e na
ocorrência de eventos, como o fechamento de vias. Nós usaremos o simulador
InterSCSimulator para gerar dados de tráfego de carros, ônibus e metrô na
cidade de São Paulo. A simulação será feita com base na pesquisa de
origem-destino (OD) realizada pela Companhia do Metropolitano de São Paulo
(Metrô). Usaremos também dados reais da frota de ônibus de São Paulo para
comparação da visualização com os dados simulados. Esperamos que o uso de
\emph{bundling} para visualização do tráfego possa colaborar  para o melhor
entendimento do trânsito e dar suporte à tomada de decisão por gestores do
transporte urbano, que precisam destinar os limitados recursos para manutenção
das vias da cidade, o que afeta diretamente a qualidade de vida dos cidadãos.
\end{resumo}
>>>>>>> 3faf015... Adds chapters and content

\begin{resumo}{eng}
  Nos últimos anos, houve um crescimento na geração e disponibilização de uma
variedade de dados de mobilidade, como carros, ônibus, aviões, embarcações,
cujo conteúdo revela a dinâmica de fluxo entre diferentes regiões espaciais. No
trânsito, a posição dos veículos é mapeada em registros 2D, geralmente latitude
e longitude, registrando sua trajetória ao longo do tempo. A análise dessas
trajetórias tem diversas aplicações, como detecção de padrões de
origem-destino, monitoramento do fluxo ou a observação de eventos externos,
como mudanças climáticas, e seus impactos no tráfego. Com isso, gestores de uma
cidade conseguem mais informações para auxiliar na tomada de decisão e
planejamento do trânsito. Recentemente, técnicas avançadas para a análise de
trajetórias surgiram, permitindo a visualização de grandes conjuntos de dados
em diferentes níveis de detalhes. Uma dessas técnicas, o bundling,  ajuda a
reduzir a oclusão na visualização de trajetórias através do agrupamento de
segmentos espacialmente próximos, possibilitando que uma representação visual
comunique de maneira mais clara e intuitiva os padrões geográficos de
origem-destino desses objetos. Neste trabalho exploraremos como o bundling pode
ser usado na análise e identificação de padrões de deslocamento de fluxos
provenientes de diferentes regiões de uma cidade e como esses fluxos impactam
em ruas e avenidas com grandes congestionamentos. Para isso, iremos aplicar
algoritmos de bundling que favorecem a análise multiescala para obter uma
visão macroscópica e microscópica dos fluxos de deslocamento. Além disso,
investigaremos como a técnica ajuda na comparação da estrutura do tráfego em
condições diversas, como em diferentes dias de semana, dias de chuva e na
ocorrência de eventos, como o fechamento de vias. Nós usaremos o simulador
InterSCSimulator para gerar dados de tráfego de carros, ônibus e metrô na
cidade de São Paulo. A simulação será feita com base na pesquisa de
origem-destino (OD) realizada pela Companhia do Metropolitano de São Paulo
(Metrô). Usaremos também dados reais da frota de ônibus de São Paulo para
comparação da visualização com os dados simulados. Esperamos que o uso de
bundling para visualização do tráfego possa colaborar  para o melhor
entendimento do trânsito e dar suporte à tomada de decisão por gestores do
transporte urbano, que precisam destinar os limitados recursos para manutenção
das vias da cidade, o que afeta diretamente a qualidade de vida dos cidadãos.
\end{resumo}

<<<<<<< HEAD
%!TeX root=../tese.tex
%("dica" para o editor de texto: este arquivo é parte de um documento maior)
% para saber mais: https://tex.stackexchange.com/q/78101/183146

% Apague as duas linhas abaixo (elas servem apenas para gerar um
% aviso no arquivo PDF quando não há nenhum dado a imprimir) e
% insira aqui o conteúdo do resumo e abstract do seu trabalho
=======
\begin{resumo}{port}
  Nos últimos anos, houve um crescimento na geração e disponibilização de uma
variedade de dados de mobilidade, como carros, ônibus, aviões, embarcações,
cujo conteúdo revela a dinâmica de fluxo entre diferentes regiões espaciais. No
trânsito, a posição dos veículos é mapeada em registros 2D, geralmente latitude
e longitude, registrando sua trajetória ao longo do tempo. A análise dessas
trajetórias tem diversas aplicações, como detecção de padrões de
origem-destino, monitoramento do fluxo ou a observação de eventos externos,
como mudanças climáticas, e seus impactos no tráfego. Com isso, gestores de uma
cidade conseguem mais informações para auxiliar na tomada de decisão e
planejamento do trânsito. Recentemente, técnicas avançadas para a análise de
trajetórias surgiram, permitindo a visualização de grandes conjuntos de dados
em diferentes níveis de detalhes. Uma dessas técnicas, o \emph{bundling},  ajuda a
reduzir a oclusão na visualização de trajetórias através do agrupamento de
segmentos espacialmente próximos, possibilitando que uma representação visual
comunique de maneira mais clara e intuitiva os padrões geográficos de
origem-destino desses objetos. Neste trabalho exploraremos como o \emph{bundling} pode
ser usado na análise e identificação de padrões de deslocamento de fluxos
provenientes de diferentes regiões de uma cidade e como esses fluxos impactam
em ruas e avenidas com grandes congestionamentos. Para isso, iremos aplicar
algoritmos de \emph{bundling} que favorecem a análise multiescala para obter uma
visão macroscópica e microscópica dos fluxos de deslocamento. Além disso,
investigaremos como a técnica ajuda na comparação da estrutura do tráfego em
condições diversas, como em diferentes dias de semana, dias de chuva e na
ocorrência de eventos, como o fechamento de vias. Nós usaremos o simulador
InterSCSimulator para gerar dados de tráfego de carros, ônibus e metrô na
cidade de São Paulo. A simulação será feita com base na pesquisa de
origem-destino (OD) realizada pela Companhia do Metropolitano de São Paulo
(Metrô). Usaremos também dados reais da frota de ônibus de São Paulo para
comparação da visualização com os dados simulados. Esperamos que o uso de
\emph{bundling} para visualização do tráfego possa colaborar  para o melhor
entendimento do trânsito e dar suporte à tomada de decisão por gestores do
transporte urbano, que precisam destinar os limitados recursos para manutenção
das vias da cidade, o que afeta diretamente a qualidade de vida dos cidadãos.
\end{resumo}
>>>>>>> 3faf015... Adds chapters and content

\begin{resumo}{eng}
  Nos últimos anos, houve um crescimento na geração e disponibilização de uma
variedade de dados de mobilidade, como carros, ônibus, aviões, embarcações,
cujo conteúdo revela a dinâmica de fluxo entre diferentes regiões espaciais. No
trânsito, a posição dos veículos é mapeada em registros 2D, geralmente latitude
e longitude, registrando sua trajetória ao longo do tempo. A análise dessas
trajetórias tem diversas aplicações, como detecção de padrões de
origem-destino, monitoramento do fluxo ou a observação de eventos externos,
como mudanças climáticas, e seus impactos no tráfego. Com isso, gestores de uma
cidade conseguem mais informações para auxiliar na tomada de decisão e
planejamento do trânsito. Recentemente, técnicas avançadas para a análise de
trajetórias surgiram, permitindo a visualização de grandes conjuntos de dados
em diferentes níveis de detalhes. Uma dessas técnicas, o bundling,  ajuda a
reduzir a oclusão na visualização de trajetórias através do agrupamento de
segmentos espacialmente próximos, possibilitando que uma representação visual
comunique de maneira mais clara e intuitiva os padrões geográficos de
origem-destino desses objetos. Neste trabalho exploraremos como o bundling pode
ser usado na análise e identificação de padrões de deslocamento de fluxos
provenientes de diferentes regiões de uma cidade e como esses fluxos impactam
em ruas e avenidas com grandes congestionamentos. Para isso, iremos aplicar
algoritmos de bundling que favorecem a análise multiescala para obter uma
visão macroscópica e microscópica dos fluxos de deslocamento. Além disso,
investigaremos como a técnica ajuda na comparação da estrutura do tráfego em
condições diversas, como em diferentes dias de semana, dias de chuva e na
ocorrência de eventos, como o fechamento de vias. Nós usaremos o simulador
InterSCSimulator para gerar dados de tráfego de carros, ônibus e metrô na
cidade de São Paulo. A simulação será feita com base na pesquisa de
origem-destino (OD) realizada pela Companhia do Metropolitano de São Paulo
(Metrô). Usaremos também dados reais da frota de ônibus de São Paulo para
comparação da visualização com os dados simulados. Esperamos que o uso de
bundling para visualização do tráfego possa colaborar  para o melhor
entendimento do trânsito e dar suporte à tomada de decisão por gestores do
transporte urbano, que precisam destinar os limitados recursos para manutenção
das vias da cidade, o que afeta diretamente a qualidade de vida dos cidadãos.
\end{resumo}



%%%%%%%%%%%%%%%%%%%%%%%%%%% LISTAS DE FIGURAS ETC. %%%%%%%%%%%%%%%%%%%%%%%%%%%%%

% Como as listas que se seguem podem não incluir uma quebra de página
% obrigatória, inserimos uma quebra manualmente aqui.
\makeatletter
\if@openright\cleardoublepage\else\clearpage\fi
\makeatother

% Todas as listas são opcionais; Usando "\chapter*" elas não são incluídas
% no sumário. As listas geradas automaticamente também não são incluídas
% por conta das opções "notlot" e "notlof" que usamos mais acima.

% Normalmente, "\chapter*" faz o novo capítulo iniciar em uma nova página, e as
% listas geradas automaticamente também por padrão ficam em páginas separadas.
% Como cada uma destas listas é muito curta, não faz muito sentido fazer isso
% aqui, então usamos este comando para desabilitar essas quebras de página.
% Se você preferir, comente as linhas com esse comando e des-comente as linhas
% sem ele para criar as listas em páginas separadas. Observe que você também
% pode inserir quebras de página manualmente (com \clearpage, veja o exemplo
% mais abaixo).
\newcommand\disablenewpage[1]{{\let\clearpage\par\let\cleardoublepage\par #1}}

% Nestas listas, é melhor usar "raggedbottom" (veja basics.tex). Colocamos
% a opção correspondente e as listas dentro de um par de chaves para ativar
% raggedbottom apenas temporariamente.
{
\raggedbottom

%%%%% Listas criadas manualmente

%\chapter*{Lista de Abreviaturas}
\disablenewpage{\chapter*{Lista de Abreviaturas}}

\begin{tabular}{rl}
  CCEB         & Critério de Classificação Econômica Brasileira \\
	KDE          & Estimadores de Densidade por Núcleo (\emph{Kernel Density-Estimation})\\
	KDEEB        & \emph{Bundling} por Estimadores de Densidade (\emph{Kernel Density-Estimation Edge Bundling})\\
  RMSP         & Região Metropolitana de São Paulo
\end{tabular}

%\chapter*{Lista de Símbolos}
%\disablenewpage{\chapter*{Lista de Símbolos}}
%
%\begin{tabular}{rl}
%        $\omega$    & Frequência angular\\
%        $\psi$      & Função de análise \emph{wavelet}\\
%        $\Psi$      & Transformada de Fourier de $\psi$\\
%\end{tabular}

% Quebra de página manual
% \clearpage

%%%%% Listas criadas automaticamente

%\listoffigures
\disablenewpage{\listoffigures}

%\listoftables
\disablenewpage{\listoftables}

% Esta lista é criada "automaticamente" pela package float quando
% definimos o novo tipo de float "program" (em utils.tex)
%\listof{program}{\programlistname}
%\disablenewpage{\listof{program}{\programlistname}}

%\listof{xablau}{\xablaulistname}
\disablenewpage{\listof{xablau}{\xablaulistname}}

% Sumário (obrigatório)
\tableofcontents

} % Final de "raggedbottom"

% Referências indiretas ("x", veja "y") para o índice remissivo (opcionais,
% pois o índice é opcional). É comum colocar esses itens no final do documento,
% junto com o comando \printindex, mas em alguns casos isso torna necessário
% executar texindy (ou makeindex) mais de uma vez, então colocar aqui é melhor.
\index{Inglês|see{Língua estrangeira}}
\index{Figuras|see{Floats}}
\index{Tabelas|see{Floats}}
\index{Código-fonte|see{Floats}}
\index{Subcaptions|see{Subfiguras}}
\index{Sublegendas|see{Subfiguras}}
\index{Equações|see{Modo Matemático}}
\index{Fórmulas|see{Modo Matemático}}
\index{Rodapé, notas|see{Notas de rodapé}}
\index{Captions|see{Legendas}}
\index{Versão original|see{Tese/Dissertação, versões}}
\index{Versão corrigida|see{Tese/Dissertação, versões}}
\index{Palavras estrangeiras|see{Língua estrangeira}}
\index{Floats!Algoritmo|see{Floats, Ordem}}
