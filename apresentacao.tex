% Authors: Nelson Lago and Fernanda Magano
% This file is distributed under the MIT Licence

%%%%%%%%%%%%%%%%%%%%%%%%%%%%%%%%%%%%%%%%%%%%%%%%%%%%%%%%%%%%%%%%%%%%%%%%%%%%%%%%
%%%%%%%%%%%%%%%%%%%%%%%%%%%%%%%%% PREÂMBULO %%%%%%%%%%%%%%%%%%%%%%%%%%%%%%%%%%%%
%%%%%%%%%%%%%%%%%%%%%%%%%%%%%%%%%%%%%%%%%%%%%%%%%%%%%%%%%%%%%%%%%%%%%%%%%%%%%%%%

% aspectratio default é 4:3;
% as mais úteis são 169 (16:9), 1610 (16:10) e 149 (14:9)
% A língua padrão é a última citada
\documentclass[xcolor={hyperref,svgnames,x11names,table},brazil,english,12pt,aspectratio=149]{beamer}

% Vários pacotes e opções de configuração genéricos
\input{extras/basics}
% A fonte precisa ser definida depois que o tema metropolis foi carregado
%\input{extras/fonts}
%%%%%%%%%%%%%%%%%%%%%%%%%%%%%%%%%%%%%%%%%%%%%%%%%%%%%%%%%%%%%%%%%%%%%%%%%%%%%%%%
%%%%%%%%%%%%%%%%%%%%%%%%%%%%% FIGURAS / FLOATS %%%%%%%%%%%%%%%%%%%%%%%%%%%%%%%%%
%%%%%%%%%%%%%%%%%%%%%%%%%%%%%%%%%%%%%%%%%%%%%%%%%%%%%%%%%%%%%%%%%%%%%%%%%%%%%%%%

% Permite importar figuras. LaTeX "tradicional" só é capaz de trabalhar com
% figuras EPS. Hoje em dia não há nenhuma boa razão para usar essa versão;
% pdfTeX, XeTeX, e LuaTeX podem usar figuras nos formatos PDF, JPG e PNG; EPS
% também pode funcionar em algumas instalações mas não é garantido, então é
% melhor evitar.
\usepackage{graphicx}

% A package float é amplamente utilizada; ela permite definir novos tipos
% de float e também acrescenta a possibilidade de definir "H" como opção de
% posicionamento do float, que significa "aqui, incondicionalmente". No
% entanto, ela tem algumas fragilidades e não é atualizada desde 2001.
% floatrow é uma versão aprimorada e com mais recursos da package "float",
% mas também não é atualizada desde 2009. Aqui utilizamos alguns recursos
% disponibilizados por ambas e é possível escolher qual delas utilizar.
%\usepackage{float}
\usepackage{floatrow}

% Por padrão, LaTeX prefere colocar floats no topo da página que
% onde eles foram definidos; vamos mudar isso. Este comando depende
% do pacote "floatrow", carregado logo acima.
\floatplacement{table}{htbp}
\floatplacement{figure}{htbp}

% Garante que floats (tabelas e figuras) só apareçam após as seções a que
% pertencem. Por padrão, se a seção começa no meio da página, LaTeX pode
% colocar a figura no topo dessa página
\usepackage{flafter}
% Às vezes um float pode ser adiado por muitas páginas; é possível forçar
% LaTeX a imprimir todos os floats pendentes com o comando \clearpage.
% Esta package acrescenta o comando \FloatBarrier, que garante que floats
% definidos anteriormente sejam impressos e garante que floats subsequentes
% não apareçam antes desse ponto. A opção "section" faz o comando ser
% aplicado automaticamente a cada nova seção. "above" e "below" desabilitam
% a barreira quando os floats estão na mesma página.
\usepackage[section,above,below]{placeins}

% LaTeX escolhe automaticamente o "melhor" lugar para colocar cada float.
% Por padrão, ele tenta colocá-los no topo da página e depois no pé da
% página; se não tiver sucesso, vai para a página seguinte e recomeça.
% Se esse algoritmo não tiver sucesso "logo", LaTeX cria uma página só
% com floats. É possível modificar esse comportamento com as opções de
% posicionamento: "tp", por exemplo, instrui LaTeX a não colocar floats
% no pé da página, e "htbp" o instrui para tentar "aqui" como a primeira
% opção. O pacote "floatrow" acrescenta a opção "H", que significa "aqui,
% incondicionalmente".
%
% A escolha do "melhor" lugar leva em conta os parâmetros abaixo, mas é
% possível ignorá-los com a opção de posicionamento "!". Dado que os
% valores default não são muito bons para floats "grandes" ou documentos
% com muitos floats, é muito comum usar "!" ou "H". No entanto, modificando
% esses parâmetros o algoritmo automático tende a funcionar melhor. Ainda
% assim, vale ler a discussão a respeito na seção "Limitações do LaTeX"
% deste modelo.

% Fração da página que pode ser ocupada por floats no topo. Default: 0.7
\renewcommand{\topfraction}{.85}
% Idem para documentos em colunas e floats que tomam as 2 colunas. Default: 0.7
\renewcommand{\dbltopfraction}{.66}
% Fração da página que pode ser ocupada por floats no pé. Default: 0.3
\renewcommand{\bottomfraction}{.7}
% Fração mínima da página que deve conter texto. Default: 0.2
\renewcommand{\textfraction}{.15}
% Numa página só de floats, fração mínima que deve ser ocupada. Default: 0.5
\renewcommand{\floatpagefraction}{.66}
% Idem para documentos em colunas e floats que tomam as 2 colunas. Default: 0.5
\renewcommand{\dblfloatpagefraction}{.66}
% Máximo de floats no topo da página. Default: 2
\setcounter{topnumber}{9}
% Idem para documentos em colunas e floats que tomam as 2 colunas. Default: 2
\setcounter{dbltopnumber}{9}
% Máximo de floats no pé da página. Default: 1
\setcounter{bottomnumber}{9}
% Máximo de floats por página. Default: 3
\setcounter{totalnumber}{20}

% Define o ambiente "\begin{landscape} -- \end{landscape}"; o texto entre
% esses comandos é impresso em modo paisagem, podendo se estender por várias
% páginas. A rotação não inclui os cabeçalhos e rodapés das páginas.
% O principal uso desta package é em conjunto com a package longtable: se
% você precisa mostrar uma tabela muito larga (que precisa ser impressa em
% modo paisagem) e longa (que se estende por várias páginas), use
% "\begin{landscape}" e "\begin{longtable}" em conjunto. Note que o modo
% landscape entra em ação imediatamente, ou seja, "\begin{landscape}" gera
% uma quebra de página no local em que é chamado. Na maioria dos casos, o
% que se quer não é isso, mas sim um "float paisagem"; isso é o que a
% package rotating oferece (veja abaixo).
\usepackage{pdflscape}

% Define dois novos tipos de float: sidewaystable e sidewaysfigure, que
% imprimem a figura ou tabela sozinha em uma página em modo paisagem. Além
% disso, permite girar elementos na página de diversas outras maneiras.
\usepackage[figuresright,clockwise]{rotating}

% Captions com fonte menor, indentação normal, corpo do texto
% negrito e nome do caption itálico
\usepackage[
  font=small,
  format=plain,
  labelfont=bf,up,
  textfont=it,up]{caption}

% Em geral, a package caption é capaz de "adivinhar" se o caption
% está acima ou abaixo da figura/tabela, mas isso não funciona
% corretamente com longtable. Aqui, forçamos a package a considerar
% que os captions ficam abaixo das tabelas.
\captionsetup[longtable]{position=bottom}

% Sub-figuras (e seus captions) - observe que existe uma package chamada
% "subfigure", mas ela é obsoleta; use esta no seu lugar.
\usepackage{subcaption}

% Permite criar imagens com texto ao redor
\usepackage{wrapfig}

% Permite incorporar um arquivo PDF como uma página adicional. Útil se
% for necessário importar uma imagem ou tabela muito grande ou ainda
% para definir uma capa personalizada.
\usepackage{pdfpages}

% Caixas de texto coloridas
%\usepackage{tcolorbox}

\floatstyle{ruled}

\ifcsdef{chapter}
    % O novo ambiente se chama "program" ("\begin{program}") e a extensão
    % temporária é "lop"
    {\newfloat{file}{htbp}{lop}[chapter]}
    {\newfloat{file}{htbp}{lop}}

\captionsetup[file]{style=ruled}

\addto\extrasbrazil{
  \floatname{file}{Arquivo}
  \gdef\filelistname{Lista de Arquivos}
}

\addto\extrasenglish{
  \floatname{file}{File}
  \gdef\programlistname{List of Files}
}

% Retorna o estilo dos floats para o padrão
\floatstyle{plain}


%%%%%%%%%%%%%%%%%%%%%%%%%%%%%%%%%%%%%%%%%%%%%%%%%%%%%%%%%%%%%%%%%%%%%%%%%%%%%%%%
%%%%%%%%%%%%%%%%%%%%%%%%%%%%%%%%%% TABELAS %%%%%%%%%%%%%%%%%%%%%%%%%%%%%%%%%%%%%
%%%%%%%%%%%%%%%%%%%%%%%%%%%%%%%%%%%%%%%%%%%%%%%%%%%%%%%%%%%%%%%%%%%%%%%%%%%%%%%%

% Tabelas simples são fáceis de fazer em LaTeX; tabelas com alguma sofisticação
% são trabalhosas, pois é difícil controlar alinhamento, largura das colunas,
% distância entre células etc. Ou seja, é muito comum que a tabela final fique
% "torta". Por isso, em muitos casos, vale mais a pena gerar a tabela em uma
% planilha, como LibreOffice calc ou excel, transformar em PDF e importar como
% figura, especialmente se você quer controlar largura/altura das células
% manualmente etc. No entanto, se você quiser fazer as tabelas em LaTeX para
% garantir a consistência com o tipo e o tamanho das fontes, é possível e o
% resultado é muito bom. Aqui há alguns pacotes que incrementam os recursos de
% tabelas do LaTeX e alguns comandos pré-prontos que podem facilitar um pouco
% seu uso.

% LaTeX por padrão não permite notas de rodapé dentro de tabelas;
% este pacote acrescenta essa funcionalidade.
\usepackage{tablefootnote}

% Por padrão, cada coluna de uma tabela tem a largura do maior texto contido
% nela, ou seja, se uma coluna contém uma célula muito larga, LaTeX não
% força nenhuma quebra de linha e a tabela "estoura" a largura do papel. A
% solução simples, nesses casos, é inserir uma ou mais quebras de linha
% manualmente, o que além de deselegante não é totalmente trivial (é preciso
% usar \makecell).
% Esta package estende o ambiente tabular para permitir definir um tamanho
% fixo para uma ou mais colunas; nesse caso, LaTeX quebra as linhas se uma
% célula é larga demais para a largura definida. Encontrar valores "bons"
% para as larguras das colunas, no entanto, também é um trabalho manual
% um tanto penoso. As packages tabularx e tabulary permitem configurar
% algumas colunas como "largura automática", evitando a necessidade da
% definição manual. Finalmente, ltxtable permite utilizar tabularx e
% longtable juntas. Neste modelo, não usamos tabularx/tabulary, mas você
% pode carregá-las se quiser.
\usepackage{array}

% Se você quer ter um pouco mais de controle sobre o tamanho de cada coluna da
% tabela, utilize estes tipos de coluna (criados com base nos recursos do pacote
% array). É só usar algo como M{número}, onde "número" (por exemplo, 0.4) é a
% fração de \textwidth que aquela coluna deve ocupar. "M" significa que o
% conteúdo da célula é centralizado; "L", alinhado à esquerda; "J", justificado;
% "R", alinhado à direita. Obviamente, a soma de todas as frações não pode ser
% maior que 1, senão a tabela vai ultrapassar a linha da margem.
\newcolumntype{M}[1]{>{\centering}m{#1\textwidth}}
\newcolumntype{L}[1]{>{\RaggedRight}m{#1\textwidth}}
\newcolumntype{R}[1]{>{\RaggedLeft}m{#1\textwidth}}
\newcolumntype{J}[1]{m{#1\textwidth}}

% Permite alinhar os elementos de uma coluna pelo ponto decimal
\usepackage{dcolumn}

% Define tabelas do tipo "longtable", similares a "tabular" mas que podem ser
% divididas em várias páginas. "longtable" também funciona corretamente com
% notas de rodapé. Note que, como uma longtable pode se estender por várias
% páginas, não faz sentido colocá-las em um float "table". Por conta disso,
% longtable define o comando "\caption" internamente.
\usepackage{longtable}

% Permite agregar linhas de tabelas, fazendo colunas "compridas"
\usepackage{multirow}

% Cria comando adicional para possibilitar a inserção de quebras de linha
% em uma célula de tabela, entre outros
\usepackage{makecell}

% Às vezes a tabela é muito larga e não cabe na página. Se os cabeçalhos da
% tabela é que são demasiadamente largos, uma solução é inclinar o texto das
% células do cabeçalho. Para fazer isso, use o comando "\rothead".
\renewcommand{\rothead}[2][60]{\makebox[11mm][l]{\rotatebox{#1}{\makecell[c]{#2}}}}

% Se quiser criar uma linha mais grossa no meio de uma tabela, use
% o comando "\thickhline".
\newlength\savedwidth
\newcommand\thickhline{
  \noalign{
    \global\savedwidth\arrayrulewidth
    \global\arrayrulewidth 1.5pt
  }
  \hline
  \noalign{\global\arrayrulewidth\savedwidth}
}

% Modifica (melhora) o layout default das tabelas e acrescenta os comandos
% \toprule, \bottomrule, \midrule e \cmidrule
\usepackage{booktabs}

% Permite colorir linhas, colunas ou células
\usepackage{colortbl}

% Você também pode se interessar pelo ambiente "tabbing", que permite
% criar tabelas simples com algumas vantagens em relação a "tabular",
% ou por esta package, que permite criar tabulações.
%\usepackage{tabto-ltx}

% index não é necessário, mas deixamos aqui para usar os mesmos
% passos de compilação que a tese
\input{extras/index}
\input{extras/hyperlinks}
\input{extras/source-code}
%%%%%%%%%%%%%%%%%%%%%%%%%%%%%%%%%%%%%%%%%%%%%%%%%%%%%%%%%%%%%%%%%%%%%%%%%%%%%%%%
%%%%%%%%%%%%%%%%%%%%%%%%%%%% OUTROS PACOTES ÚTEIS %%%%%%%%%%%%%%%%%%%%%%%%%%%%%%
%%%%%%%%%%%%%%%%%%%%%%%%%%%%%%%%%%%%%%%%%%%%%%%%%%%%%%%%%%%%%%%%%%%%%%%%%%%%%%%%

% Você provavelmente vai querer ler a documentação de alguns destes pacotes
% para personalizar algum aspecto do trabalho ou usar algum recurso específico.

% A classe Book inclui o comando \appendix, que (obviamente) permite inserir
% apêndices no documento. No entanto, não há suporte similar para anexos. Esta
% package (que não é padrão do LaTeX, foi criada para este modelo) define o
% comando \annex. Ela deve ser carregada depois de hyperref.
\dowithsubdir{extras/}{\usepackage{annex}}

% Formatação personalizada das listas "itemize", "enumerate" e
% "description", além de permitir criar novos tipos de listas
%\usepackage{paralist}
% Esta package tem a mesma finalidade, mas é mais "moderna" e tem mais
% recursos:
%  * É possível personalizar labels, espaçamento etc.
%  * É possível definir novos tipos de lista, que podem ou não ser
%    baseados nos tipos padrão
%  * É possível "interromper" uma lista e retornar a ela depois sem
%    perder a numeração
%  * Com a opção "inline", a package define os ambientes "itemize*",
%    "description*" e "enumerate*", que fazem os itens da lista como
%    parte de um único parágrafo
%\usepackage[inline]{enumitem}

% Sublinhado e outras formas de realce de texto
\usepackage{soul}
\usepackage{soulutf8}

% Notação bra-ket
%\usepackage{braket}

%\num \SI and \SIrange. For example, \SI{10}{\hertz} \SIrange{10}{100}{\hertz}
\usepackage[binary-units]{siunitx}

% Citações melhores; se você pretende fazer citações de textos
% relativamente extensos, vale a pena ler a documentação. biblatex
% utiliza recursos deste pacote.
\usepackage{csquotes}

\usepackage{url}
% URL com fonte sem serifa ao invés de teletype
\urlstyle{sf}

% Permite inserir comentários, muito bom durante a escrita do texto;
% você também pode se interessar pela package pdfcomment.
\usepackage[textsize=scriptsize,colorinlistoftodos,textwidth=2.5cm]{todonotes}
\presetkeys{todonotes}{color=orange!40!white}{}

% Comando para fazer notas com highlight no texto correspondente:
% \hltodo[texto][opções]{comentário}
\makeatletter
\if@todonotes@disabled
  \NewDocumentCommand{\hltodo}{O{} O{} +m}{#1}
\else
  \NewDocumentCommand{\hltodo}{O{} O{} +m}{
    \ifstrempty{#1}{}{\texthl{#1}}%
    \todo[#2]{#3}{}%
  }
\fi
\makeatother

% Vamos reduzir o espaçamento entre linhas nas notas/comentários
\makeatletter
\xpatchcmd{\@todo}
  {\renewcommand{\@todonotes@text}{#2}}
  {\renewcommand{\@todonotes@text}{\begin{spacing}{0.5}#2\end{spacing}}}
  {}
  {}
\makeatother

% Símbolos adicionais, como \celsius, \ohm, \perthousand etc.
%\usepackage{gensymb}

% Símbolos adicionais, como \textrightarrow, \texteuro etc.
\usepackage{textcomp}

% Permite criar listas como glossários, listas de abreviaturas etc.
% https://en.wikibooks.org/wiki/LaTeX/Glossary
%\usepackage{glossaries}

% Permite formatar texto em colunas
\usepackage{multicol}

% Gantt charts; útil para fazer o cronograma para o exame de
% qualificação, por exemplo.
\usepackage{pgfgantt}

% Na versão 5 do pacote pgfgantt, a opção "compress calendar"
% deixou de existir, sendo substituída por "time slot unit=month".
% Aqui, um truque para funcionar com ambas as versões.
\makeatletter
\@ifpackagelater{pgfgantt}{2018/01/01}
  {\ganttset{time slot unit=month}}
  {\ganttset{compress calendar}}
\makeatother

% Estes parâmetros definem a aparência das gantt charts e variam
% em função da fonte do documento.
\ganttset{
    time slot format=isodate-yearmonth,
    vgrid,
    x unit=1.7em,
    y unit title=3ex,
    y unit chart=4ex,
    % O "strut" é necessário para alinhar o baseline dos nomes dos meses
    title label font=\strut\footnotesize,
    group label font=\footnotesize\bfseries,
    bar label font=\footnotesize,
    milestone label font=\footnotesize\itshape,
    % "align=right" é necessário para \ganttalignnewline funcionar
    group label node/.append style={align=right},
    bar label node/.append style={align=right},
    milestone label node/.append style={align=right},
    group incomplete/.append style={fill=black!50},
    bar/.append style={fill=black!25,draw=black},
    bar incomplete/.append style={fill=white,draw=black},
    % Não é preciso imprimir "0%"
    progress label text=\ifnumequal{#1}{0}{}{(#1\%)},
    % Formato e tamanho dos elementos
    title height=.9,
    group top shift=.4,
    group left shift=0,
    group right shift=0,
    group peaks tip position=0,
    group peaks width=.2,
    group peaks height=.3,
    milestone height=.4,
    milestone top shift=.4,
    milestone left shift=.8,
    milestone right shift=.2,
}

% Em inglês, tanto o nome completo quanto a abreviação do mês de maio
% são "May"; por conta disso, na tradução em português LaTeX erra a
% abreviação. Como talvez usemos o nome inteiro do mês em outro lugar,
% ao invés de forçar a tradução para "Mai" globalmente, fazemos isso
% apenas em ganttchart.
\AtBeginEnvironment{ganttchart}{\deftranslation[to=Portuguese]{May}{Mai}}

% Ilustrações, diagramas, gráficos etc. criados diretamente em LaTeX.
% Também é útil se você quiser importar gráficos gerados com GnuPlot.
\usepackage{tikz}

% Gráficos gerados diretamente em LaTeX; é possível usar tikz para
% isso também.
\usepackage{pgfplots}
% sobre níveis de compatibilidade do pgfplots, veja
% https://tex.stackexchange.com/a/81912/183146
%\pgfplotsset{compat=1.14} % TeXLive 2016
%\pgfplotsset{compat=1.15} % TeXLive 2017
%\pgfplotsset{compat=1.16} % TeXLive 2019
\pgfplotsset{compat=newest}

% Importação direta de arquivos gerados por gnuplot com o
% driver/terminal "lua tikz"; esta package não faz parte da
% instalação padrão do LaTeX, mas sim do gnuplot.
%\usepackage{gnuplot-lua-tikz}

% O formato pdf permite anexar arquivos ao documento, que aparecem
% na página como ícones "clicáveis"; esta package implementa esse
% recurso em LaTeX.
%\usepackage{attachfile}

% Os comandos \TeX e \LaTeX são nativos do LaTeX; esta package acrescenta os
% comandos \XeLaTeX e \LuaLaTeX. Você provavelmente não precisa desse recurso
% e, portanto, pode removê-la.
\usepackage{metalogo}
\providecommand{\ConTeXt}{\textsc{Con\TeX{}t}}
% Outros logos da família TeX; você também pode remover estas linhas:
\usepackage{hologo}
\renewcommand{\ConTeXt}{\hologo{ConTeXt}}


% Diretórios onde estão as figuras; com isso, não é preciso colocar o caminho
% completo em \includegraphics (e nem a extensão).
\graphicspath{{figuras/},{logos/}}

% Comandos rápidos para mudar de língua:
% \en -> muda para o inglês
% \br -> muda para o português
% \texten{blah} -> o texto "blah" é em inglês
% \textbr{blah} -> o texto "blah" é em português
\babeltags{br = brazil, en = english}

% Espaçamento simples
\singlespacing


%%%%%%%%%%%%%%%%%%%%%%%%%%%% APARÊNCIA DO BEAMER %%%%%%%%%%%%%%%%%%%%%%%%%%%%%%%

\usepackage{appendixnumberbeamer}

% Tema metropolis com algumas modificações
\dowithsubdir{extras/}{\usetheme{imeusp}}

% O padrão usa um tom de vermelho escuro como cor principal; a opção
% "greeny" troca essa cor por um tom de verde; a opção "sandy" usa o
% mesmo tom de verde mas modifica a cor padrão dos blocos para um
% tom amarelado.
\dowithsubdir{extras/}{\usecolortheme[greeny]{imeusp}}
% Desabilita a cor de rodapé
\setbeamercolor{footline}{fg=,bg=}

%%%%%%%%%%%%%%%%%%%%%%%%%% COMANDOS PARA O USUÁRIO %%%%%%%%%%%%%%%%%%%%%%%%%%%%%

\newcommand\col{\column{.5\textwidth}}

% A cada nova seção, recapitula o sumário
% Para desabilitar, é só comentar este trecho
\AtBeginSection[]{
  \begin{frame}<beamer>{Agenda}
    \intermezzo
  \end{frame}
}

% Blocos de cor personalizada
\newenvironment{coloredblock}[2]%
{
    \setbeamercolor{block title}{fg=white,bg=#1!80!white}
    \setbeamercolor{block body}{fg=darkgray,bg=#1!20!white}
    \setbeamercolor{local structure}{fg=darkgray,bg=#1!20!white}
    \begin{block}{#2}
}{
    \end{block}
}


%%%%%%%%%%%%%%%%%%%%%%%%%%%%%%% BIBLIOGRAFIA %%%%%%%%%%%%%%%%%%%%%%%%%%%%%%%%%%%

% O estilo de citação e de lista de referências usado por biblatex
\PassOptionsToPackage{
  % Estilo similar a plainnat
  %bibstyle=extras/plainnat-ime,
  %citestyle=extras/plainnat-ime,
  % Estilo similar a alpha
  %bibstyle=alphabetic,
  %citestyle=alphabetic,
  % O estilo numérico é comum em artigos
  %bibstyle=numeric,
  %citestyle=numeric,
  % Um estilo que busca ser compatível com a ABNT:
  %bibstyle=abnt,
  %citestyle=abnt,
  style=bwl-FU,
}{biblatex}

% Para personalizar outros aspectos da bibliografia e citações ou
% para utilizar bibtex, modifique este arquivo.
\input{extras/bibconfig}

% A configuração do arquivo acima foi definida para teses/dissertações;
% vamos mudar algumas opções
\ExecuteBibliographyOptions{
  backref=false,
  maxbibnames=2,
}

\AtBeginBibliography{
  % Numa apresentação, a bibliografia pode ficar em tamanho menor
  \footnotesize
}

% O arquivo com os dados bibliográficos; você pode usar este comando
% mais de uma vez para acrescentar múltiplos arquivos
\addbibresource{bibliografia.bib}

% Este comando permite acrescentar itens à lista de referências sem incluir
% uma referência de fato no texto (pode ser usado em qualquer lugar do texto)
%\nocite{bronevetsky02,schmidt03:MSc, FSF:GNU-GPL, CORBA:spec, MenaChalco08}
% Com este comando, todos os itens do arquivo .bib são incluídos na lista
% de referências
%\nocite{*}


%%%%%%%%%%%%%%%%%%%%%%%%%%%%%%%%%%%%%%%%%%%%%%%%%%%%%%%%%%%%%%%%%%%%%%%%%%%%%%%%
%%%%%%%%%%%%%%%%%%%%%%%%%%%%%% INÍCIO DO CONTEÚDO %%%%%%%%%%%%%%%%%%%%%%%%%%%%%%
%%%%%%%%%%%%%%%%%%%%%%%%%%%%%%%%%%%%%%%%%%%%%%%%%%%%%%%%%%%%%%%%%%%%%%%%%%%%%%%%

\begin{document}

%%!TeX root=../apresentacao.tex
%("dica" para o editor de texto: este arquivo é parte de um documento maior)
% para saber mais: https://tex.stackexchange.com/q/78101/183146

% Apague as duas linhas abaixo (elas servem apenas para gerar um
% aviso no arquivo PDF quando não há nenhum dado a imprimir) e
% insira aqui o conteúdo do seu trabalho. Tome por base o
% arquivo corpo-apresentacao.tex do diretório conteudo-exemplo
% para a definição do título, autoria etc. e estrutura da
% apresentação.

\input{extras/aviso-conteudo}
\avisoApresentacao
%%%%%%%%%%%%%%%%%%%%%%%%%%%%%%%%% METADADOS %%%%%%%%%%%%%%%%%%%%%%%%%%%%%%%%%%%%

\title[The shortened title]{Visualização Dinâmica Multiescala\\ de Dados do Tráfego
de Veículos}
% \subtitle{The (optional) subtitle}

\author[Tallys Martins]{Tallys Martins}

%\institute[USP]{\textbf{Workshop Name} \\ Computer Science Department \\ IME USP}
\institute[USP]{
 \textbf{Orientador:} Fabio Kon \\
 \textbf{Co-Orientador:} Higor Amario de Souza \\
 Computer Science Department \\ IME USP
}

\date{Março de 2019}

% Coloca a imagem no fundo da página de título
\bgimage{\includegraphics[width=\paperwidth]{fundo_predios_e_grafo}}

% Logotipos no rodapé da página de título
\logos{
  \hfil\hfil\includegraphics[width=.1\textwidth]{usp-logo}\hfil%
  \raisebox{-.0103\paperheight}{\includegraphics[height=.0932\paperheight]{interscity-logo}}\hfil%
  \raisebox{-.033\paperheight}{\includegraphics[width=.07\textwidth,trim=0 230 0 0,clip]{ime-logo}}\hfil\hfil
}

%\logos{
%  \hfil\hfil\includegraphics[width=.1\textwidth]{usp-logo}\hfil%
%  \raisebox{-.0103\paperheight}{\includegraphics[height=.0932\paperheight]{interscity-logo}}\hfil%
%  \raisebox{-.00517\paperheight}{\includegraphics[height=.057\paperheight]{cnpq-logo}}\hfil%
%  \raisebox{-.0342\paperheight}{\includegraphics[height=.1035\paperheight]{capes-logo}}\hfil%
%  \includegraphics[height=.044\paperheight]{fapesp-logo}\hfil\hfil
%}

% Usado para criar o qrcode com o endereço da apresentação
\presentationurl{http://interscity.org}

% Inclui o qrcode no sumário da apresentação
\includeqrcodeintoc

% O slide de sumário pode ser dividido em colunas; o parâmetro
% determina após qual o número da seção fazer a quebra de coluna
% (use zero para uma coluna ou simplesmente omita este comando).
\toccolumns{4}


%%%%%%%%%%%%%%%%%%%%%%%%%%%%%%%%%%%%%%%%%%%%%%%%%%%%%%%%%%%%%%%%%%%%%%%%%%%%%%%%
%%%%%%%%%%%%%%%%%%%%%%%%%%%% INÍCIO DA APRESENTAÇÃO %%%%%%%%%%%%%%%%%%%%%%%%%%%%
%%%%%%%%%%%%%%%%%%%%%%%%%%%%%%%%%%%%%%%%%%%%%%%%%%%%%%%%%%%%%%%%%%%%%%%%%%%%%%%%

% É complicado colocar uma imagem de fundo, os logos das agências e
% o conteúdo "normal" do slide de título sem que as coisas fiquem
% bagunçadas, então definimos um comando para gerar o slide de título
\customtitlepage

% Slide com o qrcode
\showqrcode

\begin{frame}{Agenda}
  \overview
\end{frame}

\section{Introdução}

\begin{frame}{Contexto}
  ``Tráfego em uma via é o fluxo ou a passagem de veículos e pedestres por um
caminho que pode ser em regiões urbanas, mares, no ar e até no subsolo''

\hfill \citep{Chen2015}
\end{frame}

\begin{frame}{Contexto}
  \textbf{Dados de tráfego}
  \begin{itemize}
    \item Têm crescido com o avanço de sistemas de transporte
    \item Complexos e volumosos 
  \end{itemize}
\end{frame}

\begin{frame}{Contexto}
  \textbf{Visualizar esses dados}
  \begin{itemize}
    \item Trazem novas abstrações para entender seu comportamento
    \item Relação entre deslocamentos entre regiões
  \end{itemize}
\end{frame}

%\begin{frame}[plain]
%  \includegraphics[width=\textwidth]{interscity-logo}
%\end{frame}

\begin{frame}[standout]
  Bundling
\end{frame}

\begin{frame}{Objetivos}
  \begin{block}{Explorar o \emph{bundling} na visualização dos fluxos de origem-destino}
    \begin{enumerate}
      \item Como o bundling pode ser usado para identificar os fluxos de origem e des-
tino no trânsito em diferentes escalas?

      \item É possível utilizar o bundling para identificar padrões de fluxos de origem
e destino no trânsito?

      \item O bundling é eficiente para gerar uma visualização de uma grande quanti-
dade de dados do trânsito?
    \end{enumerate}
  \end{block}
\end{frame}

\section{Conceitos}

\begin{frame}{Leiaute}
\end{frame}

\begin{frame}{Bundling}
\end{frame}

\begin{frame}{InterSCSimulator}
\end{frame}


\section{Trabalhos Relacionados}

\begin{frame}{Trabalhos Relacionados}
\end{frame}

\section{Metodologia}

\begin{frame}{Metodologia}
\end{frame}

\section{Considerações}

\begin{frame}{Validação}
\end{frame}

\begin{frame}{Plano de Trabalho}
\end{frame}

\section{References}

\begin{frame}[allowframebreaks]{References}
  \nocite{bronevetsky02, schmidt03:MSc, FSF:GNU-GPL, CORBA:spec, MenaChalco08, natbib, biblatex, eco:09}
  \printbibliography
\end{frame}

% Recapitulando
\begin{frame}{\insertshorttitle}
  \overview

  % \begin{center} acrescenta espaço vertical;
  % como possivelmente temos bem pouco espaço aqui,
  % vamos usar centering
  {%
    \centering\noindent%
    \url{https://gitlab.com/link-of-your-repository}\par
  }

\end{frame}

\showqrcode

\appendix

\begin{frame}{Extra info}
  \begin{itemize}
    \item It is often useful to have some extra slides addressing likely questions from the audience at the end of the presentation
    \item By putting them after the ``appendix'' command, they are not counted in the page count indicator
  \end{itemize}
\end{frame}

%!TeX root=../apresentacao.tex
%("dica" para o editor de texto: este arquivo é parte de um documento maior)
% para saber mais: https://tex.stackexchange.com/q/78101/183146

% Apague as duas linhas abaixo (elas servem apenas para gerar um
% aviso no arquivo PDF quando não há nenhum dado a imprimir) e
% insira aqui o conteúdo do seu trabalho. Tome por base o
% arquivo corpo-apresentacao.tex do diretório conteudo-exemplo
% para a definição do título, autoria etc. e estrutura da
% apresentação.

\input{extras/aviso-conteudo}
\avisoApresentacao
%%%%%%%%%%%%%%%%%%%%%%%%%%%%%%%%% METADADOS %%%%%%%%%%%%%%%%%%%%%%%%%%%%%%%%%%%%

\title[The shortened title]{Visualização Dinâmica Multiescala\\ de Dados do Tráfego
de Veículos}
% \subtitle{The (optional) subtitle}

\author[Tallys Martins]{Tallys Martins}

%\institute[USP]{\textbf{Workshop Name} \\ Computer Science Department \\ IME USP}
\institute[USP]{
 \textbf{Orientador:} Fabio Kon \\
 \textbf{Co-Orientador:} Higor Amario de Souza \\
 Computer Science Department \\ IME USP
}

\date{Março de 2019}

% Coloca a imagem no fundo da página de título
\bgimage{\includegraphics[width=\paperwidth]{fundo_predios_e_grafo}}

% Logotipos no rodapé da página de título
\logos{
  \hfil\hfil\includegraphics[width=.1\textwidth]{usp-logo}\hfil%
  \raisebox{-.0103\paperheight}{\includegraphics[height=.0932\paperheight]{interscity-logo}}\hfil%
  \raisebox{-.033\paperheight}{\includegraphics[width=.07\textwidth,trim=0 230 0 0,clip]{ime-logo}}\hfil\hfil
}

%\logos{
%  \hfil\hfil\includegraphics[width=.1\textwidth]{usp-logo}\hfil%
%  \raisebox{-.0103\paperheight}{\includegraphics[height=.0932\paperheight]{interscity-logo}}\hfil%
%  \raisebox{-.00517\paperheight}{\includegraphics[height=.057\paperheight]{cnpq-logo}}\hfil%
%  \raisebox{-.0342\paperheight}{\includegraphics[height=.1035\paperheight]{capes-logo}}\hfil%
%  \includegraphics[height=.044\paperheight]{fapesp-logo}\hfil\hfil
%}

% Usado para criar o qrcode com o endereço da apresentação
\presentationurl{http://interscity.org}

% Inclui o qrcode no sumário da apresentação
\includeqrcodeintoc

% O slide de sumário pode ser dividido em colunas; o parâmetro
% determina após qual o número da seção fazer a quebra de coluna
% (use zero para uma coluna ou simplesmente omita este comando).
\toccolumns{4}


%%%%%%%%%%%%%%%%%%%%%%%%%%%%%%%%%%%%%%%%%%%%%%%%%%%%%%%%%%%%%%%%%%%%%%%%%%%%%%%%
%%%%%%%%%%%%%%%%%%%%%%%%%%%% INÍCIO DA APRESENTAÇÃO %%%%%%%%%%%%%%%%%%%%%%%%%%%%
%%%%%%%%%%%%%%%%%%%%%%%%%%%%%%%%%%%%%%%%%%%%%%%%%%%%%%%%%%%%%%%%%%%%%%%%%%%%%%%%

% É complicado colocar uma imagem de fundo, os logos das agências e
% o conteúdo "normal" do slide de título sem que as coisas fiquem
% bagunçadas, então definimos um comando para gerar o slide de título
\customtitlepage

% Slide com o qrcode
\showqrcode

\begin{frame}{Agenda}
  \overview
\end{frame}

\section{Introdução}

\begin{frame}{Contexto}
  ``Tráfego em uma via é o fluxo ou a passagem de veículos e pedestres por um
caminho que pode ser em regiões urbanas, mares, no ar e até no subsolo''

\hfill \citep{Chen2015}
\end{frame}

\begin{frame}{Contexto}
  \textbf{Dados de tráfego}
  \begin{itemize}
    \item Têm crescido com o avanço de sistemas de transporte
    \item Complexos e volumosos 
  \end{itemize}
\end{frame}

\begin{frame}{Contexto}
  \textbf{Visualizar esses dados}
  \begin{itemize}
    \item Trazem novas abstrações para entender seu comportamento
    \item Relação entre deslocamentos entre regiões
  \end{itemize}
\end{frame}

%\begin{frame}[plain]
%  \includegraphics[width=\textwidth]{interscity-logo}
%\end{frame}

\begin{frame}[standout]
  Bundling
\end{frame}

\begin{frame}{Objetivos}
  \begin{block}{Explorar o \emph{bundling} na visualização dos fluxos de origem-destino}
    \begin{enumerate}
      \item Como o bundling pode ser usado para identificar os fluxos de origem e des-
tino no trânsito em diferentes escalas?

      \item É possível utilizar o bundling para identificar padrões de fluxos de origem
e destino no trânsito?

      \item O bundling é eficiente para gerar uma visualização de uma grande quanti-
dade de dados do trânsito?
    \end{enumerate}
  \end{block}
\end{frame}

\section{Conceitos}

\begin{frame}{Leiaute}
\end{frame}

\begin{frame}{Bundling}
\end{frame}

\begin{frame}{InterSCSimulator}
\end{frame}


\section{Trabalhos Relacionados}

\begin{frame}{Trabalhos Relacionados}
\end{frame}

\section{Metodologia}

\begin{frame}{Metodologia}
\end{frame}

\section{Considerações}

\begin{frame}{Validação}
\end{frame}

\begin{frame}{Plano de Trabalho}
\end{frame}

\section{References}

\begin{frame}[allowframebreaks]{References}
  \nocite{bronevetsky02, schmidt03:MSc, FSF:GNU-GPL, CORBA:spec, MenaChalco08, natbib, biblatex, eco:09}
  \printbibliography
\end{frame}

% Recapitulando
\begin{frame}{\insertshorttitle}
  \overview

  % \begin{center} acrescenta espaço vertical;
  % como possivelmente temos bem pouco espaço aqui,
  % vamos usar centering
  {%
    \centering\noindent%
    \url{https://gitlab.com/link-of-your-repository}\par
  }

\end{frame}

\showqrcode

\appendix

\begin{frame}{Extra info}
  \begin{itemize}
    \item It is often useful to have some extra slides addressing likely questions from the audience at the end of the presentation
    \item By putting them after the ``appendix'' command, they are not counted in the page count indicator
  \end{itemize}
\end{frame}

% Um parágrafo em LaTeX termina com uma linha vazia; como não é possível ter
% certeza que um arquivo incluído (neste caso, "corpo-apresentacao") terminou
% com uma linha vazia, é recomendável usar o comando "par" após "input" para
% garantir que o último parágrafo do arquivo incluído realmente terminou.
\par

% Como o local da bibliografia em uma apresentação pode variar, os comandos
% relacionados estão em corpo-apresentacao.tex e não aqui.

\end{document}
